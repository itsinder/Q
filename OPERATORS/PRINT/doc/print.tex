\documentclass[letterpaper]{article}

%%%%%%%%%%%%%%%%%%%%%%%%%%%%%%%%%%%%%%%%%%%%%%%%%%%%%%%%%%%%%%%%%%%%%%%%%%%%%%%%%%
% Special for e-unibus doc commands

\newcommand{\ForLater}{
\begin{center}
{\bf NOT FOR CURRENT VERSION}
\end{center}
}
\newcommand{\TBC}{\framebox{\textbf{TO BE COMPLETED}}}
\newcommand{\DISCUSS}{\Ovalbox {\bf \textcolor{red}{FOR DISCUSSION}}}
\newcommand{\Input}{\framebox{\textsf{in}}}
\newcommand{\Output}{\framebox{\textsf{out}}}
\newcommand{\debug}[1]{\textbf{debug start} #1 \textbf{debug finish}}
\newcommand{\inx}[1]{\emph{#1}}
\newtheorem{notation}{Notation}
\newtheorem{definition}{Definition}
\newtheorem{problem_statement}{Problem Statement}
\newtheorem{invariant}{Invariant}
\newtheorem{assumption}{Assumption}
\newtheorem{resource_string}{Resource String}
\newtheorem{testcase}{Test Case}
\newtheorem{note}{Note}
\newtheorem{specification}{Specification}
\newtheorem{caution}{Caution}
\newtheorem{prereq}{Pre-requisite}
\newtheorem{action}{Action}
\newtheorem{query}{Query}
\newcommand{\be}{\begin{enumerate}}
\newcommand{\ee}{\end{enumerate}}
\newcommand{\bi}{\begin{itemize}}
\newcommand{\ei}{\end{itemize}}
\newcommand{\bv}{\begin{verbatim}}
\newcommand{\ev}{\end{verbatim}}
\newcommand{\bd}{\begin{description}}
\newcommand{\ed}{\end{description}}
\newcommand{\bpre}{\begin{prereq}}
\newcommand{\epre}{\end{prereq}}
\newcommand{\bact}{\begin{action}}
\newcommand{\eact}{\end{action}}
\newcommand{\bs}{\begin{specification}}
\newcommand{\es}{\end{specification}}
\newcommand{\btc}{\begin{testcase}}
\newcommand{\etc}{\end{testcase}}
\newcommand{\bc}{\begin{caution}}
\newcommand{\ec}{\end{caution}}
\newcommand{\la}{\leftarrow}
\newcommand{\IpArgs}{\subsection{Input Arguments}}
\newcommand{\PreReqs}{\subsection{Pre-requisities}}
\newcommand{\Actions}{\subsection{Actions}}
\newcommand{\Coverage}{{\bf To test coverage.}}

%%%%%%%%%%%%%%%%%%%%%%%%%%%%%%%%%%%%%%%%%%%%%%%%%%%%%%%%%%%%%%%%%%%%%%%%%%%


\newtheorem{theorem}{Theorem}[section]
\newtheorem{lemma}[theorem]{Lemma}
\newtheorem{proposition}[theorem]{Proposition}
\newtheorem{corollary}[theorem]{Corollary}

\newenvironment{proof}[1][Proof]{\begin{trivlist}
\item[\hskip \labelsep {\bfseries #1}]}{\end{trivlist}}
\newenvironment{intuition}[1][Intuition]{\begin{trivlist}
\item[\hskip \labelsep {\bfseries #1}]}{\end{trivlist}}
%% \newenvironment{definition}[1][Definition]{\begin{trivlist}
%% \item[\hskip \labelsep {\bfseries #1}]}{\end{trivlist}}
\newenvironment{example}[1][Example]{\begin{trivlist}
\item[\hskip \labelsep {\bfseries #1}]}{\end{trivlist}}
\newenvironment{remark}[1][Remark]{\begin{trivlist}
\item[\hskip \labelsep {\bfseries #1}]}{\end{trivlist}}

\newcommand{\qed}{\nobreak \ifvmode \relax \else
      \ifdim\lastskip<1.5em \hskip-\lastskip
      \hskip1.5em plus0em minus0.5em \fi \nobreak
      \vrule height0.75em width0.5em depth0.25em\fi}

%%%%%%%%%%%%%%%%%%%%%%%%%%%%%%%%%%%%%%%%%%%%%%%%%%%%%%%%%%%%%%%%%
% \newcommand{\Alogon}{\mbox{\fontfamily{ptm}\selectfont {\large \selectfont A} \hspace{-1.2ex} {\large \selectfont L} \hspace{-2.3ex} \raisebox{0.45ex}{ {\footnotesize \selectfont O} } \hspace{-1.80ex} {\large \selectfont G} \hspace{-1.80ex} \raisebox{-0.33ex}{ {\large \selectfont O} } \hspace{-1.8ex} {\large \selectfont N}}}


\usepackage{times}
\usepackage{helvet}
\usepackage{courier}
\usepackage{hyperref}
\usepackage{fancyheadings}
\pagestyle{fancy}
\usepackage{pmc}
\usepackage{graphicx}
\setlength\textwidth{6.5in}
\setlength\textheight{8.5in}
\begin{document}
\title{Print to CSV}
\author{ Ramesh Subramonian }
\maketitle
\thispagestyle{fancy}
\lfoot{{\small Data Analytics Team}}
\cfoot{}
\rfoot{{\small \thepage}}

\section{Introduction}

This document describes how we print data in one or more columns into a CSV
file. 

\subsection{Invocation}

\begin{verbatim}
print(<table of columns|column>, [filter], [destination])
\end{verbatim}
So, the first argument to print could be just \(x\) or 
it could be \(\{x, y\}\)

\subsection{Print Filters}

There are a few ways in which we can instruct the print to {\bf not}
print all the elements of the column. We can use a range (Section~\ref{range})
or a ``where clause'' represented as a bit column, Section~\ref{bit_column}.

\subsubsection{Range}
\label{range}

Support
\verb+lb:ub+ where {\tt lb} is the lower bound (inclusive) and {\tt
ub} is the upper bound (exclusive). So a range of \(2:4\) would mean printing
out the \(2^{th}\) element and the \(3^{th}\) element. 
Recall that we do C style indexing, so the first element is the \(0^{th}\)
element. 
Some time in the future --- support Python's range specification style. 

\subsubsection{Bit column}
\label{bit_column}
Prints only those rows where the corresponding bit is set.

\subsection{Print Destination}

The destination of a print can be either 
\be
\item nil, in this case print\_csv returns a string. 
In following two cases, print will return true.  This is not meant to be used
when the output is large. Its  a good aid for quick and dirty testing.
\item stdout (user does not specify a file name)
\item a text file 
\ee

It is important to note that the output of print should be consumable by the
load csv function. For example, if the value to be printed contains a special
character --- comma, double quote, eoln or backslash --- then it must be
suitable escaped and possibly quoted as well. A null value will be printed out
as two consescutive dquote characters.

\section{Printing a single column}
\label{single_column}

Let's start with describing how a single column should be printed. As
described in the data loading specification, every type that is
registered with Q should have a C function that (i) converts an ascii
string into the C type and (ii) converts a fixed set of bytes into an ascii
string. We simply iterate through the elements and apply the text converter to
each consecutive block. For example, we would apply {\t fprintf} with the
\verb+%lf+ format on every successive 8 bytes if the field type were
{\tt double}

\section{Printing multiple columns}
\label{multiple_columns}

We require that 
\be
\item 
the first argument is a Lua table where each element is a single column. 
\item all columns have the same length
\ee

\subsection{Extensions}
In this section, we describe what happens when the constraints imposed above are
selectively relaxed.
\subsubsection{Mismatched number of elements}
In this case, the column with insufficient elements starts printing out null
values once it has exhausted all its values. So, if \(x = \{1, 2,3\}\) and \(y
= \{4, 5, 6, 7\}\), then we would get
\begin{verbatim}
1,4
2,5
3,6
"",7
\end{verbatim}

\subsubsection{Scalar not column}
If the table looked like \(\{1, x\}\) where \(x = \{2, 3, 4 \}\), then we would
get
\begin{verbatim}
1,2
1,3
1,4
\end{verbatim}

\end{document}
