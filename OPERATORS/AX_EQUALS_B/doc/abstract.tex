\documentclass{article}
\usepackage{amsmath}

\title{A Simple Solver for Matrix Equations}
\author{Andrew Winkler, Ph.~D.}

\begin{document}

\maketitle

\section{Abstract}

In the companion paper to this post, we derive a solver for linear systems \begin{math}{}Ax=b\end{math}, which works for real, complex, rational,
and certain other scalars. It is related to the Cholesky decomposition, but sidesteps creating a factorization
to directly construct the solution.

Here we exhibit a C implementation of the most important case, that where
real numbers are represented as double precision numbers and the matrix \begin{math}A\end{math} is symmetric and positive semi-definite,
that is \begin{math}{}x^tAx >=0 \end{math} for all \begin{math}{}x\end{math}.

\section{Implementation Notes}
The algorithm proceeds recursively. It is not tail-recursive; but at each step
the only information which is retained for post-processing is the current column
of \begin{math}{}A\end{math}, and the current element of 
\begin{math}{}b\end{math}.

This allows us to overwrite the not-yet-processed elements of 
\begin{math}{}A\end{math} and \begin{math}{}b\end{math}, with the transforms
which the recursion employs. In this way, the memory required for the matrix
never grows. On the other hand, both 
\begin{math}{}A\end{math} and \begin{math}{}b\end{math} are destroyed in the process, so must be copied if any further use of them is desired.

Because the matrix \begin{math}A\end{math} is symmetric, we need only an upper or lower triangle; accordingly we allocate a vector of size \begin{math}n-i\end{math} to represent column number \begin{math}i\end{math}, starting with the diagonal element and descending the column.

\begin{verbatim}
$ cat positive_solver.h 
extern
double * positive_solver(double ** A, double * b, int n);

$ cat positive_solver.c
/* 
 * Andrew Winkler

This code solves the equation Ax=b, where A is a positive
semi-definite matrix, so that x^t A x >= 0 for all x,
provided a solution exists. If A is positive definite,
it will be an isomorphism, but in general conditions
must be imposed on b, for it to lie in the range of A.

It has the virtue of dramatic simplicity - there's no need
to explicitly construct the Cholesky decomposition, no need
to do the explicit back-substitutions.  Yet it's essentially
equivalent to that more labored approach, so its
performance/stability/memory, etc. should be at least as good.

There are two kinds of checks, both of which are disabled.
This means that a solution will be generated; it is up to
the caller to verify that the equation Ax=b is satisfied to
the desired precision. If it does not, then no solution exists.

The first kind of check is that Avec is 0 whenever A[0][0]
is 0, which will always be true if A is positive semi-definite,
but which could become false because of any small perturbation
caused by roundoff error upstream in the computation of A.
The second kind of check is that b[0] is 0 whenever A[0][0] is 0,
which is necessary for b to lie in the range of A. The checks are
disabled because the code does not deal with real numbers,
but rather floating point numbers; it's up to the user to decide
what is "close enough" to zero.

*/

#include <stdlib.h>
#include <stdio.h>
#include "positive_solver.h"

void _consistency_checker(double * v, int n) {
    for (int i=0; i<n; i++) {
        if ( v[i] != 0.0 ) exit(-1);
    }
}

void _positive_solver(
    double ** A, 
    double * b, 
    double * x, 
    int n
    ) 
{
  if (n < 1) exit(-1);

  if (n == 1) {
    if (A[0][0] == 0.0) {
        /* _consistency_checker(b, 1); */
        x[0] = 0.0;
        return;
    }
    x[0] = b[0] / A[0][0];
    return;
  }

  double * bvec = b + 1;
  double * Avec = A[0] + 1;
  double ** Asub = A + 1;
  double * xvec = x + 1;

  int m = n -1;

  if (A[0][0] == 0.0) {
      /*
      _consistency_checker(b, 1);
      _consistency_checker(Avec, m);
      */
  } else {
      for(int j=0; j < m; j++){
        bvec[j] -= Avec[j] * b[0] / A[0][0];
        for(int i=0; i < m - j; i++)
          Asub[i][j] -= Avec[i] * Avec[i+j] / A[0][0];
      }
  }

  _positive_solver(Asub, bvec, xvec, m);

  if (A[0][0] == 0.0) {
      x[0] = 0.0;
      return;
  }

  double p = 0; for(int k=0; k<m; k++) p += Avec[k] * xvec[k];

  x[0] = (b[0] - p) / A[0][0];

  return;
}

#include <malloc.h>
double * positive_solver(
    double ** A, 
    double * b, 
    int n
    ) 
{
  double * x = (double *) malloc(n * sizeof(double));
  _positive_solver(A, b, x, n);
  return x;
}

\end{verbatim}
\end{document}
