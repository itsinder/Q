\startreport{Q --- Grammar}
\reportauthor{Ramesh Subramonian}

\newcommand{\YES}{\checkmark}
\newcommand{\KW}{\bf{KW}\!:\!}
\newcommand{\Range}{{\cal{I}}}
\newcommand{\Args}{\({\cal{A}}\)}
\newcommand{\Pragmas}{{\cal{P}}}
\newcommand{\Multiple}{{\cal{F}}}
\newcommand{\cfld}{{\sf{b}}}
\newcommand{\NumericTypes}{\(\{I1, I2, I4, I8, F4, F8\}\)}
\newcommand{\IntegerTypes}{\(\{I1, I2, I4, I8\}\)}
\newcommand{\RealTypes}{\(\{F4, F8\}\)}

\section{Introduction}

Q is a vector language, designed for efficient implementation
of counting, sorting and data transformations. It uses a single data
structure --- a table.

\subsection{Motivation}

I will motivate the need for Q by quoting from two of my Gods --- Codd and
Iverson. I could be accused of quoting scripture for my purpose (see
below) and it is true that I am being selective in my extracts.
However, that does not detract from their essential verity.

\begin{verse}
The devil can cite Scripture for his purpose. \\ 
An evil soul producing holy witness \\ 
Is like a villain with a smiling cheek, \\
A goodly apple rotten at the heart:
\end{verse}

\subsubsection{Extracts from Codd}

{\tt 
The most important motivation for the research work that resulted in the
relational model was the objective of providing a sharp and clear
boundary between the logical and physical aspects of database
management. We call this the {\em data independence objective}.

A second objective was to make the model structurally simple, so that
all kinds of users and programmers could have a common understanding of
the data, and could therefore communicate with one another about the
database. We call this the {\em communicability objective}.

A third objective as to introduce high level language concepts *but not
specific syntax) to enable users to express operations upon large chunks
of information at a time. This entailed providing a foundation for
set-oriented processing (i.e., the ability to express in a single
statement the processing of multiple sets of records at a time). We
call this the {\em set-processing objective}.

To satisfy these three objectives, it was necessary to discard all those
data structuring concepts (e.g., repeating groups, linked structures)
that were not familiar to end users and to take a fresh look at
the addressing of data.
}

We have deviated from Codd's preference for the relational model.
Instead, we choose to drop down one level to the table. As Codd writes:

{\tt 
  Tables are at a lower level of abstraction than relations, since they
    give the impression that positional (array-type) addressing is
    applicable (which is not true of \(n\)-ary relations), and they fail
    to show that the information content of a table is independent of
    row order. Nevertheless, even with these minor flaws, tables are the
    most important conceptual representation of relations, because they
    are universally understood.

}

Lastly, in designing Q, we wanted it to be a data model as Codd defines
one

{\tt
A data model is a combination of at least three components:
  \be
  \item A collection of data structure types (the building blocks);
\item A collection of operators or rules of inference, which can be
  applied to any valid instances of the data types listed in (1), to
  retrieve, derive, or modify data from any parts of those structures in
  any combinations desired;
\item A collection of general  integrity rules, which implicitly or
  explicitly define the set of consistent database states or changes of
  states or both
  \ee

}

\subsubsection{Extracts from Iverson}

The importance of language has been stated over the centuries. Iverson
quotes the following from Whitehead:

{\tt By relieving the brain of all unnecessary work, a good notation
  sets it free to concentrate on more advanced problems, and in effect
    increases the mental power of the race}

In the same vein, he quotes Babbage:

{\tt The quantity of meaning compressed into small space by algebraic
  signs, is another circumstance that facilitates the reasonings we are
    accustomed to carry on by their aid}

I would hesitate to claim that Q meets any of the following criteria
that Iverson lays down for good notation. But it is  definitely the
guiding principle and aspirational goal for Q.

{\tt 
\be
\item Ease of expressing constructs arising in problems. If it is to be
effective as a tool of thought, a notation must allow convenient
expression not only of notions arising directly from a problem, but also
of those arising in subsequent analysis, generalization and
specialization.

\item Suggestivity. A notation will be said to be suggestive if the
forms of the expressions arising in one set of problems suggests related
expressions which find application in other problems.

\item Ability to subordinate detail. Brevity is achieved by
subordinating detail, and we will consider three important ways of doing
this
\bi
\item the use of arrays
\item the assignment of names to functions and variables
\item the use of operators
\ei

\item Economy. Economy requires that a large number of ideas be
expressible in terms of a relatively small vocabulary. 

\item Amenability to formal proofs

\ee
}

\section{Notations and Conventions}

\subsection{Auxiliary Fields}
\label{aux_flds}

An auxiliary field is a field that belongs to a primary field. It is used to
create some property for the priomary field 
\be
\item NullFld. This is a boolean field which tells us whether the corresponding
entry in the primary field is null or not
\item LenFld. This is an integer field which tells us the length of the
corresponding entry of the primary field, whose FldType is SV
\item OffFld. The data for an SV field is stored as a continuous sequence of
bytes. The value of this field tells us how far the text for this cell is from
the start.
\ee


\begin{definition}
A boolean field is an I1 field with 
\(\mathrm{MinVal} \geq 0\) and 
\(\mathrm{MaxVal} \leq 1\)
\end{definition}


\subsection{Environment Variables}

\bd
\item [Q\_META\_DATA] Name of file where meta data is stored
\item [Q\_DATA\_DIR] Name of directory where data is stored. 
Needs to be unique for each table space. 
\item [Q\_RUN\_TIME\_CHECKS] If it is set, then we execute run time.
\ed

%-----------------------------------------------------

\subsection{Notations}
\bi
\item The upside down T symbol \(\bot\) means null
\item Properties are of the form \verb+PROP=foobar+ where \verb+foobar+
is the desired property. Foobar must be alphabetic.
\item Operators are of the form \verb+OP=foobar+ where \verb+foobar+
is the desired operator. Foobar must be alphabetic.
\item Args are of the form \verb+ARGS={valid json}+. Limitations on the
characters permitted in the JSON? \TBC 
\item Wherever it is possible to have an I1 field, \(f\), on the RHS of a Q
statement, one can have the nn field, \(NN(f)\). Are there any
exceptions to this rule? \TBC
\item When a string is used for user input (like a regular expression
    match), it is uuencoded on the way in so that there is no conflict
with other non-alpha numeric characters which have special meaning to
the parser
\item Lower bounds are inclusive and upper bounds are exclusive.
\item If an operation is expected to create a table \(T\) and a table
by that name exists, then the original table is first deleted. Examples
of operations that create tables are \TBC
\item 
If an operation is expected to create a field with name \(f\) and a
field with such a name exists, then the old field is deleted {\bf after}
the operation is performed and replaced by the newly created field. 
\item A table has a non-null, unique name.
\item A field has a non-null name that is unique within a table.
\item If a field, {\tt foo} has null values, then property {\bf HasNullFld} is true and a boolean field {\\tt .nn.foo} exists.
\item Most operations cannot be performed on the auxiliary field of a primary
field. When this is possible, we will say so explicitly. 

\ei

\subsection{Field Types}
\label{fld_types}
See Table~\ref{tbl_fld_types}. Details below
\bd
\item [LABELS] Assume \(T.f\) is a string field. However, assume that the number of
unique strings is relatively small and our use of this string is just as a
label.  In such cases, we will create a ``lookup table'', \(T_L\), that 
has a field {\bf txt}. We will set \(T.f\) to be an integer field, 
serving as a foreign key to \(T_L\). 
Hence, \(0 \leq T.f[i] < |T_L|\), 
We set property
{\tt LkpTbl} tp \(T_L\) and proeprty {\tt LkpFld} to {\tt txt}.
For purposes of comparison, we use integer comparisons, which is much faster.
However, when we have to print out the value, we need an indirection step to
access the string corresponding to the integer.

\item [SC] We store one byte (for null character) more than what user specifies as length.  Can have a null field. Verify this \TBC
\item [SV] \TBC
\ed
\begin{table}
\centering
\begin{tabular}{|l|l|l|} \hline \hline
{\bf Value} & {\bf Description} & {\bf Status} \\ \hline \hline
I1 & 1 byte signed integer & \YES \\ \hline 
I2 & 2 byte signed integer & \YES \\ \hline 
I4 & 4 byte signed integer & \YES \\ \hline 
I8 & 8 byte signed integer & \YES \\ \hline 
F4 & 4 byte floating point & \YES \\ \hline 
F8 & 8 byte floating point & \YES \\ \hline 
SC & constant length string& \YES \\ \hline 
F8 & variable length string& \YES \\ \hline 
B  & bit field  & WIP\\ \hline 

\hline
\hline
\end{tabular}
\caption{Types supported by Q}
\label{tbl_fld_types}
\end{table}
%---------------------------------------------------------------
\subsection{Some Useful Enumerations}

\begin{table}
\centering
\begin{tabular}{|l|l|} \hline \hline
{\bf Value} \\ \hline \hline
Unknown \\ \hline
Ascending \\ \hline
Descending \\ \hline
Unsorted \\ \hline
\hline
\end{tabular}
\caption{Return Values for SortType} 
\label{srt_types}
\end{table}
%------------------------------------------------------------
\begin{table} \centering
\begin{tabular}{|l|l|} \hline \hline
{\bf Abbreviation} & {\bf Explanation} \\ \hline 
NumRows  & number of rows \\ \hline
Fields  & lists fields in table \\ \hline
Exists   & whether it exists or not \\ \hline
RefCount & number of tables using it \\ \hline
\hline
\end{tabular}
\caption{Properties of Tables}
\label{tbl_props}
\end{table}
%------------------------------------------------------------
\begin{table}
\centering
\begin{tabular}{|l|l|} \hline \hline
{\bf Abbreviation} & {\bf Explanation} \\ \hline 
HasNullFld  & does it have a null field  \\ \hline
HasLenFld & does it have a length field \\ \hline
HasoffFld & does it have an offset field \\ \hline
FldType & one of values in Section~\ref{fld_types} \\ \hline
SortType & one of values in Table~\ref{srt_types} \\ \hline
LkpTbl   & lookup table \\ \hline
Lkpfld   & lookup field \\ \hline
Width  & width for FldType = SC \\ \hline
\hline
\end{tabular}
\caption{Properties of Fields}
\label{fld_props}
\end{table}

\begin{invariant}
\(\mathrm{FldType} = SC \Leftrightarrow \mathrm{Width} \neq \bot \)
\end{invariant}

\begin{invariant}
\(\mathrm{FldType} = SV \Leftrightarrow \mathrm{HasOffFld} = true\)
\end{invariant}

\begin{invariant}
\(\mathrm{FldType} = SV \Leftrightarrow \mathrm{HasLenFld} = true\)
\end{invariant}

\begin{invariant}
\(\mathrm{LkpTbl} \neq \bot \Leftrightarrow \mathrm{LkpFld} \neq \bot\)
\end{invariant}

\begin{invariant}
\(\mathrm{LkpTbl} \neq \bot \Rightarrow \mathrm{FldType} \in \{I1, I2, I4, I8\}\)
\end{invariant}
%------------------------------------------------------------

\subsection{PTO --- Partial Table Operations}
\label{PTO}
There are three ways in whicha  partial table can be specified. They are all
writte as \(T_1|(\ldots)\) where the \(\ldots\) can be one of 
\be
\item a boolean field, \(f_c\).
In this case, we consider only rows of \(T_1\) where \(f_c = 1\)
\item \(n_1, n_2\) where \( 0 \leq n_1 < n_2 < n_R\), where \(n_R\) is the number
of rows in the table. 
In this case, we consider only those rows whose indexes
are \(\geq n_1 \wedge < n_2\)
boolean (I1 to be precise) field has value 1 
\item \(T_2, f_{lb}, f_{ub}\). In this case, \(f_{lb}, f_{ub}\) are I8 fields in
\(T_2\) such that \(f_{lb} < f_{ub}\). The \(i^{th}\) row of 
\(T_1\)  is considered only if \(\exists j:~ T_2[j].f_{lb} \leq i <
T_2[j].f_{ub}\)

\ee

%------------------------------------------------------------
\subsection{Pragmas}

There are times when we wish to 
\be
\item modify the normal behavior of an operation 
\item provide hints to the execution engine as to which algorithm to use
\ee
These are provided as {\bf pragmas} in the command, which take the form
{\tt PRAGMA=JSON STRING}


\section{Commands}

See Table~\ref{Q_Commands}
\begin{center}
\tablefirsthead{\hline {\bf Command} & {\bf Summary} & {\bf Details} \\ \hline}
\tablehead{\hline {\bf Command} & {\bf Summary} & {\bf Details} \\ \hline}
\tabletail{\hline {\small\sl Q Commands} & & {\small\sl continued on next page} \\ \hline}
\tablelasttail{\hline}
\bottomcaption{Q Commands}
\label{Q_Commands}
\label{tbl_Q_Commands}
\begin{supertabular}{|l|l|l|}

\(+ \mathrm{~Scope~}  S\) & Start Scope \(S\) & \\ \hline
\(- \mathrm{~Scope~} \) & Stop Current Scope & \\ \hline
\(? \mathrm{~Scope~}\) & Print Scope Hierarchy & \\ \hline
\( + \mathrm{~Compound~}\) & Start Compound Expression & \\ \hline
\( - \mathrm{~Compound~}\) & Stop Compound Expression & \\ \hline
\hline
{\bf OP}=none & no op & \\  \hline
{\bf OP}=dump FileName & save meta data in file & Section~\ref{dump} \\ \hline
%------------------------------------------
%------------------------------------------
\(\# T\) & number of rows in table & \\ 
\(? T\) & does table exist & \\ 
\(? T *\) & all meta data for table & \\ 
\(? T \) {\bf PROP=x}  & describe property \(x\) of table & Table~\ref{tbl_props} \\ \hline
%----------------------------------------------------
\(? T.f\) &  is field in table & \\ 
\(? T.f *\) & all meta data for field & \\ 
\(? T.f  \) {\bf PROP=x} & describe property \(x\) of field \(T.f\) & Table~\ref{fld_props} \\ \hline
%---------------------------------------------------------
\(+ T.f\) {\bf PROP=x} \(v\) & set property \(x\) of field \(T.f\) to \(v\)
& Table~\ref{fld_props} \\  
\(+ T_1.f_1\) {\bf PROP=nnfld} \(T_2.f_2\) & THINK & THINK \\ \hline

\(- T.f\) {\bf PROP=x} & unset property \(x\) of field \(T>f\) of table \(T\) & \\ \hline

\(T := \) {\bf OP=New} \(n\) & add empty table with \(n\) rows & \\ 
\(T := \) {\bf OP=LoadCSV} \Args & load from CSV & Section~\ref{load_tbl} \\

\(T := \) {\bf OP=LoadBin} \Args&  load from binary & Section~\ref{load_tbl} \\ 
\(T := \) {\bf OP=LoadHDFS} \Args & load from hdfs & Section~\ref{load_tbl} \\ \hline
%------------------------------------------------------------
\(T.f := \) {\bf OP=LoadBin} \Args & load single field &
Section~\ref{load_fld} \\  
\(T.f := \) {\bf OP=LoadCSV} \Args & load single field & Section~\ref{load_fld} \\  \hline
%---------
\(- T\) & delete table &  \\ 
\(- \{T_1, T_2, \ldots\} \) & delete tables &  \\ 
\(- T.f\) & delete field &  \\
\(- T.\{f_1, f_2, \ldots\}\) & delete fields & \\
\(- T.f\) {\bf PROP=NNFld} &  THINK & THINK \\ \hline
%-----------------------------------------------------------
\(T.f := \) {\bf OP=} \(\oplus\) \Args & scalar to field & Section~\ref{s_to_f} \\  \hline
{\bf OP=} \(\oplus\) \(T.f\) & field to scalar & Section~\ref{f_to_s} \\
{\bf OP=} \(\oplus\) \(T|(\ldots).f\) & field to scalar & Section~\ref{f_to_s} \\
% THINK +pr_fld+ & \\ \hline 
%--------
\(T_1 \leftarrow T_2\) & rename table & Section~\ref{rename_tbl} \\  
\(T_1 := T_2\) & copy table & Section~\ref{copy_tbl}  \\ 
\(T_1 := T_2|(..)\) & copy partial table & Section~\ref{copy_tbl}  \\  \hline
%---------
\(T_1 \stackrel{+}{=} T_2\) & append table \(T_2\) to \(T_1\) & Section~\ref{app_tbl} \\ \hline
%---------
\(T_1.f_1 \leftarrow T_2.f_2\) & replace \(f_1\) in \(T_1\) with
\(f_2\) in \(T_2\) & \\ 
                           &     and delete \(f_2\) in \(T_2\) & \\ \hline
%-------------
\(T_1.f_1 := T_2.f_2\) & copy field & Section~\ref{copy_fld}  \\ 
\(T_1.f_1 := T_2|(\ldots).f_2\) & copy partial field & Section~\ref{copy_fld}  \\
\hline
%---------
\(T_1(f_1 := f_2)\) & duplicate field & \\ \hline
\(T_1(f_1 <= f_2)\)  & rename field    & After, no \(T_1.f_2\) \\ \hline
\(T_1.f_1 := T_2.f_2\)  & duplicate field    & After, no \(T_2.f_2\) \\ \hline
\(T_1.f_1 <= T_2.f_2\)  & move field    & \\ \hline
\(T_1.NN(f_1) := T_2.f_2\) & make null field & THINK \\ \hline
\(T_1.f_1 := T_2.NN(f_2)\) & break null field & THINK \\ \hline
\(T_1 (f_1 := \) {\bf OP=} \(\oplus~ f_2)\) \Args & Create \(f_1\) from \(f_2\) & Section~\ref{f1opf2} \\ \hline
\(T_1 (f_1 := \) {\bf OP=} \(\oplus~\) \Args \( f_2)\) 
& Create \(f_1\) from scalar and \(f_2\) & Section~\ref{f1s1opf2} \\ \hline

\(T_1.f_1 := \) {\bf OP=} \(\oplus~ T_2.f_2\) \Args & Create \(T_1.f_1\) from
\(T_2.f_2\)  & \\ \hline
% subsample stride 
\(T_1 (f_1 := f_2 \) {\bf OP=} \(\oplus~ f_3)\) \Args & Create \(f_1\) form
\(f_2, f_3\)  & Section~\ref{f1f2opf3} \\ \hline
%--------
%--------
\(T_1(w := ?\!:~~x~~y~~z)\) & ternary operator & Section~\ref{ternary_op} \\ 
\(T_1(w := ?\!:~~lb~~ub~~y~~z)\) & ternary operator & Section~\ref{ternary_op} \\ 
\(T_1(w := ?\!:~~lb~~ub~~val)\) & set value & Section~\ref{ternary_op} \\  \hline

%--------
\(T_1.\{f_I, f_1\} := \) {\bf OP=Permute} \(T_2.f_2\) & permute & 
Section~\ref{xfer} \\ \hline
%-----
\(T_1.f_0 := \) {\bf OP=Pack~} \(T_1.\{f_1, f_2, f_3, \ldots\}\) \Args 
& pack many fields into one Section~\ref{pack}  \\ 
%-----
\(T_1.\{f_1, f_2, \ldots\} := \) {\bf OP=UnPack} \(T_1.f_0\) \Args & 
unpack one field into many & Section~\ref{unpack} \\ \
%-----
\(T_1.\{f_1, f_2\} := \) {\bf OP=UnConcat} \(T_2.f_0\) & 
unpack 2 fields into one & Section~\ref{f1opf2f3} \\ \hline
\(T_1.f_1 := \) {\bf OP=} \(\oplus~T_2.f_2~~T_3.f_3\) & \(\cup, \cap, -\)
& Section~\ref{t1f1t2f2opt3f3} \\ \hline
%-----
{\bf OP=Sort}  \(T.f\) \Args & in place sort & Section~\ref{fop}  \\ 
{\bf OP=Saturate}  \(T.f\) \Args & in place saturate & \\ 
{\bf OP=Permute}  \(T.f\) \Args & in place random permute & \\ \hline
%--------
{\bf OP=Sort} \(T.\{f_1, f_2\}\) \Args & in place joint sort & 
Section~\ref{sortf1f2} \\ \hline
%----
\(T_1.\{f_L, f_U, f_n, f_{\mu}\}  := \) {\bf OP=Quantiles} \(T_2.f_2\) \Args & Calculate quantiles & Section~\ref{percentiles} \\ 
 & Calculate approximate quantiles & \\ \hline
%----
\(T_1.\{f_L, f_U, f_C\} := \) {\bf OP=NumInRange} \(T_2.f_2 \) & 
count number in range & Section~\ref{num_in_range} \\  \hline
%----
\(T_1.\{f_v, f_n\} := \) {\bf OP=CountValues} \(T_2.f_2\) \Args & 
count number of occurrences & Section~\ref{Counting} \\ 
\(T_1.\{f_v, f_n\} := \) {\bf OP=CountValues} \(T_2|(\ldots).f_2\) \Args & 
count number of occurrences & Section~\ref{Counting} \\ 
% TODO & \verb+count_ht+ & & \\ 
% TODO & \verb+approx_unique+ & & \\ 
% TODO & \verb+countf+ & & \\  \hline
%-------------------
\(T_D.\{l_D, f_D\} := \) {\bf OP=Join} \(T_S.\{f_S, l_S\}\) \Args & 
join & Section~\ref{join} \\  \hline
%-------------------
\(T_D.f_D := \) {\bf OP=Top} \(T_S.f_S\) \Args & top n & Section~\ref{top_n}
\\
\(T_D.f_D := \) {\bf OP=Top} \(T_S|(\ldots).f_S\) \Args & top n & Section~\ref{top_n} \\ \hline
% TODO & \verb+approx_frequent+ & & \\ \hline
 \(T_D.\{f_V, f_X\} := \) {\bf OP=ExistsIn} \(T_S.f_V\) &
exists in  & Section~\ref{is_a_in_b} \\ \hline

\end{supertabular}
\end{center}
\newpage

%------------------------------------------------------------
\subsection{From Field to Scalar}
\label{f_to_s}

{\bf \textcolor{red}{Supports PTO}}

Note that there is an overlap between what is a field property
(Table~\ref{fld_props}) and what is produced by reducing a field to a scalar. 

\begin{table}
\centering
\begin{tabular}{|l|l|l|} \hline \hline
{\bf Abbreviation} & {\bf Explanation} & \Args \\ \hline 
Sum & sum & --- \\ \hline
Min & minimum & --- \\ \hline
Max & maximum & --- \\ \hline
Avg & average & --- \\ \hline
SD & standard deviation & --- \\ \hline
Print & prints values & --- \\ \hline
NDV  & number of distinct values & --- \\ \hline
NumNull & number of null values & \\ \hline
ApproxNDV & approx number of distinct values & \TBC \\ \hline
ValAtIdx & value at specified index & Index \\ \hline
IdxAtVal & first index with specified value & Value \\ \hline
\hline
\end{tabular}
\caption{Reduce operators}
\label{tbl_f_to_s}
\end{table}
Partial Table Specification now allowed for
\be
\item NumNull
\item ValAtIdx. If no index with specified Value, returns -1
\item IdxAtVal. Index must be within bounds of table size.
\ee

%------------------------------------------------------------
\subsection{Load Field from External Source}
\label{load_fld}
Valid values for {\bf OP} are as follows
\bd
\item [LoadCSV] \Args\ to contain
\be
\item DataFile. Name of Data File. MANDATORY
\item DataDirectory. Name of directory where data file is to be found. OPTIONAL.
Default is current working directory.
\item FldType
\ee

\item [LoadBin] Same as above
\ed
%------------------------------------------------------------
\subsection{Loading Table from External Sources}
\label{load_tbl}

Valid values for {\bf OP} are as follows.
\bd
\item [LoadCSV] In this case \Args\ contains
\be
\item MetaDataFile. See Section~\ref{Format of Meta Data File}. MANDATORY
\item DataFile. Name of Data File. MANDATORY
\item DataDirectory. Name of directory where data file is to be found. OPTIONAL.
Default is current working directory.
\item IgnoreHeader. Can have values true or false. Default is false. Ignores
first line if set to true
\item FldSep. Can have values COMMA, TAB. Default is COMMA.
\item \ldots \TBC
\ee
\item [LoadBin] In this case \Args\ contains
\be
\item DataFile. Name of Data File. MANDATORY
\item DataDirectory. Name of directory where data file is to be found. OPTIONAL.
\item FldList. Comma separated list of field names e.g., {\tt foo,bar}
\item FldSpec. Comma separated list of field types e.g., {\tt I4,I8}
Things to note
\be
\item Number of fields must match number of format specifications
\item Formats SC and SV not allowed for LoadBin
\item Null values not allowed
\ee
\ee
\item [LoadHDFS] \TBC
\ed
%-------------------------------------------------------------
\subsection{From Scalars to Field}
\label{s_to_f}
In all cases, \Args\ must contain {\bf FldType}. Values provided must be
consistent with field type. 
Valid values for {\bf OP} are in Table~\ref{tbl_s_to_f}. Details below. 
\begin{table}
\centering
\begin{tabular}{|l|l|l|l|} \hline \hline
{\bf OP} \\ \hline \hline
Constant \\ \hline
Sequence \\ \hline
Period \\ \hline
Random \\ \hline
\hline
\end{tabular}
\caption{stof: Valid values of {\bf OP} for creating field from scalar}
\label{tbl_s_to_f}
\end{table}

\bd
\item [Constant] In this case, \Args\ must contain {\bf Value}
\item [Sequence] In this case, \Args\ must contain {\bf Start}, \(s\), and {\bf Increment},
\(\delta\). Values set to 
\(f_0 = s, f_1 = s+ \delta, f_i = (i-1) \times \delta\)
\item [Period] Like Sequence but \Args\ must also contain {\bf Period}, 
\(T\), which tells
us how often to reset to initial value. For example, \(s = 2, \delta = 3, T =
4\) means we get values like \(\{2, 5, 8, 11, 2, 5, 8, 11, \ldots\}\)
\item [Random] \Args\ must contain {\bf Distribution}. If distribution is
``Uniform'', \Args\ must contain {\bf MinVal} and {\bf MaxVal}. 
\ed

%------------------------------------------------------------

\subsection{Creating One Field from Another}
\label{f1opf2}

Consider the creation of \(f_1\) from \(f_2\). 
See Table~\ref{tbl_f1opf2} for possible values of {\bf OP}. 
If the field type of
the newly created field, \(f_1\), is the same as that of the source field,
\(f_2\),  then a checkmark is placed in the column {\bf InType = OutType}.

%------------------------------------------------------------
\begin{table}
\centering
\begin{tabular}{|l|l|l|} \hline \hline
{\bf Abbreviation} & \(FldType(f_2)\) & {\bf Details}  \\ \hline \hline
Shrink & \IntegerTypes & Section~\ref{f1opf2_shrink} \\ \hline
Cast & \NumericTypes & Section~\ref{f1opf2_cast} \\ \hline
Convert & \NumericTypes & Section~\ref{f1opf2_conv} \\ \hline
BitCount & \IntegerTypes & Section~\ref{f1opf2_bitcount} \\ \hline
Sqrt & \RealTypes & square root \\ \hline
Abs  & \NumericTypes & absolute value \\ \hline
CRC32  & \(\{I4, I8\}\) & crc32 hash \\ \hline
\(!\)  & \(\{I1, I2, I4, I8\}\) & Section~\ref{f1opf2_not} \\ \hline
\(++\)  & \(\{I1, I2, I4, I8\}\) & increment \\ \hline
\(--\)  & \(\{I1, I2, I4, I8\}\)  & decrement \\ \hline
\verb+~+  & \(\{I1, I2, I4, I8\}\) & bit-wise complement  \\ \hline
Reciprocal  & \(\{F4, F8\}\) &  \\ \hline
Accumulate  & \NumericTypes & Section~\ref{f1opf2_accumulate}  \\ \hline
Smear  & I1 & Section~\ref{f1opf2_smear}  \\ \hline
Mix  & I4, I8 & Section~\ref{f1opf2_mix} \\ \hline
IdxWithReset  & I1 & Section~\ref{f1opf2_idx_with_reset}  \\ \hline
\hline
\end{tabular}
\caption{f1opf2: Valid values of {\bf OP} for creating one field from another}
\label{tbl_f1opf2}
\end{table}

\subsubsection{Shrink}
\label{f1opf2_shrink}

Converts to smallest integer type that will not cause loss of precision. Valid
input types are \(\{I2, I4, I8\}\). Valid output types are \(\{I1, I2, I4,
I8\}\)

\subsubsection{Cast}
\label{f1opf2_cast}

Dangerous operation since it simply interprets the bytes differently. 
Valid transformations are in
Table~\ref{tbl_f1opf2_cast}

%------------------------------------------------------------
\begin{table}
\centering
\begin{tabular}{|l|l|l|} \hline \hline
{\bf Old FldType} & {\bf New FldType} \\ \hline 
I4 & F4 \\ \hline
F4 & I4 \\ \hline
I8 & F8 \\ \hline
F8 & I8 \\ \hline 
\hline
\end{tabular}
\caption{f1opf2: Valid Casts}
\label{tbl_f1opf2_cast}
\end{table}


\subsubsection{Convert}
\label{f1opf2_conv}

\bi
\item \Args\ must contain {\bf NewFldType}
\item If min/max of \(f_2\) will cause overflow based on
\(FldType(f_1)\), then operation fails
\item Loss of precision in conversions is for user to worry about
\ei

\subsubsection{BitCount}
\label{f1opf2_bitcount}

Counts the number of bits. Output type is I1.

\subsubsection{Logical Not}
\label{f1opf2_not}

\(!\) is the logical not operator.

\subsubsection{Accumulate}
\label{f1opf2_accumulate}

\bi
\item \(f_1[0] = f_2[0]\)
\item \(f_1[j] = \sum_{i=0}^{i=j} f_2[i]\)
\ei

Output type same as  input type, except is over-ruled by user using
NewFldType in \Args

\subsubsection{Smear}
\label{f1opf2_smear}

User needs to specify NumAfter, \(n_R\), and NumBefore, \(n_L\), in \Args.
This operation ``smears'' the selection, specified by source field \(f_2\) by \(n_R\) to the right and
\(n_L\) to the left i.e., 
\bi
\item 
\(f_1[i] = 1 \Rightarrow \forall j: 1 \leq j \leq n_R, f_2[i+j] \leftarrow 1\)
 \item 
\(f_1[i] = 1 \Rightarrow \forall j: 1 \leq j \leq n_L, f_2[i-j] \leftarrow 1\) 
\item \(n_R \geq 0\), \(n_L \geq 0\), \(n_R\) and \(n_L\) cannot both be
0
\item \(FldType(f_1) \leftarrow FldType(f_2) = I1\)
\item \(IsNulLFld(f_1) = false\)
\ei


\subsubsection{Mix}
\label{f1opf2_mix}

\TBC

\subsubsection{Index with Reset}
\label{f1opf2_idx_with_reset}

\TBC


\subsection{Creating One Field from a Scalar and a Field}
\label{f1s1opf2}
Here, we create a new field in a table by using an existing field and
a scalar. See Table~\ref{tbl_f1s1opf2} for possible values of {\bf OP}.
\Args must contain {\bf SCALAR}.
\begin{table}[hb]
\centering
\begin{tabular}{|l||l|l|l|l|l|l|l|l|}  \hline \hline
{\bf OP} & {\bf I1} & {\bf I2} & {\bf I4} & {\bf I8}
& {\bf F4 } & {\bf F8}  & {\bf SC } & {\bf Details} \\ \hline \hline
\verb=+=      & \YES & \YES & \YES & \YES & \YES & & &  \\ \hline
\verb+-+      &      &      & \YES & \YES & \YES & & &  \\ \hline
\verb+*+      &      &      & \YES & \YES & \YES & & &  \\ \hline
\verb+/+      &      &      & \YES & \YES & \YES & & &  \\ \hline
\verb+%+      &      &      & \YES & \YES &      & & &  \\ \hline
\verb+&+      & \YES &      & \YES & \YES &      & & &  \\ \hline
\verb+|+      & \YES &      & \YES & \YES &      & & &  \\ \hline
\verb+^+      & \YES &      & \YES & \YES &      & & &  \\ \hline
\verb+>+      & \YES &      & \YES & \YES & \YES & & &  \\ \hline
\verb+<+      & \YES &      & \YES & \YES & \YES & & &  \\ \hline
\verb+>=+     & \YES &      & \YES & \YES & \YES & & &  \\ \hline
\verb+<=+     & \YES &      & \YES & \YES & \YES & & &  \\ \hline
\verb+!=+     & \YES &      & \YES & \YES & \YES & & &  \\ \hline
\verb+==+     & \YES & \YES & \YES & \YES & \YES & & &  \\ \hline
\verb+<<+     &      &      & \YES & \YES &      & & &  \\ \hline
\verb+>>+     &      &      & \YES & \YES &      & & &  \\ \hline
\verb+<=||>=+ &      &      & \YES &      &      & & &  \\ \hline
\verb+>&&<+   &      &      & \YES &      &      & & &  \\ \hline
\verb+>=&&<=+ &      &      & \YES &      &      & & &  \\ \hline
{\tt regex}   &      &      & \YES &      &      & & \YES &
Section~\ref{regex} \\ \hline
\hline
\end{tabular}
\caption{Supported Operations for f1s1opf2}
\label{tbl_f1s1opf2}
\end{table}

\subsubsection{Regex Match}
\label{regex}

\Args\ must contain {\bf MatchType} which is either {\bf Exact} or {\bf Regex}. If the first, then its a straight equality comparison. Otherwise, it is considered a regex match. FldType of source field, \(f_2\), is SC. FldType of newly created field is I1, values being 0 or 1.

\subsection{Ternary Operator}
\label{ternary_op}

There are three variants
\be
\item \(T_1(w := ?\!: x~~y~~z)\). In this case,
\bi
\item \(FldType(x) = I1\)
\item \(HasNulLFld(x) = false\)
\item \(y, z\) can be either a scalar or a field. 
\item If both are fields, there field types must be the same. 
\(FldType(w)\) is set to the field type of \(y\) or \(z\)
\item If one of them is a field, the scalar for the other should be compatible
with the FldType of the field
\(FldType(w)\) is set to the FldType of the argument that is a field
\item If both are scalars, the FldType is chosen by the system to be smallest
that will suffice.
\item No matter how chosen, \(FldType(w) \in \{I1, I2, I4, I8, F4, F8\}\)
\ei

\item \(T_1(w := ?\!: lb~~ub~~y~~z)\) 
\bi
\item Constraints on \(y, z\) same as above
\item lb, ub are integers are specify a valid index range with \(lb < ub\)
\item Sets \(w\) in \(T\) to field \(y\) for all rows specified and to field
\(z\) for all other rows. 
\ei

\item \(T_1(w := ?\!: lb~~ub~~val)\) 
\bi
\item lb, ub are integers are specify a valid index range with \(lb < ub\)
\item Field \(w\) must exist 
\item \(val\) must be consistent with \(FldType(w)\)
\item Sets \(w\) in \(T\) to scalar \(s\) for all rows specified. 
\item \(FldType(w) \in \{I1, I2,  I4, I8, F4, F8, SC\}\)
\ei
\ee


\subsection{f1f2opf3}
\label{f1f2opf3}

Valid values for {\bf OP} are in Table~\ref{tbl_f1f2opf3}. 
Assume that we create \(f_3 \leftarrow f_1 \oplus f_2\).
\bi
\item \(FldType(f_1) = FldType(f_2)\)
\item \(FldType(f_3) \leftarrow FldType(f_1)\), except for concatenate. In that
case, 
\be
\item \(FldType(f_1) = I1 \Rightarrow FldType(f_3) = I2\)
\item \(FldType(f_1) = I2 \Rightarrow FldType(f_3) = I4\)
\item \(FldType(f_1) = I4 \Rightarrow FldType(f_3) = I8\)
\ee
\ei
%------------------------------------------------------------
\begin{table}
\centering
\begin{tabular}{|l|l|l|l|l|l|l|l|l|} \hline \hline
{\bf Abbreviation} & {\bf Explanation} & 
{\bf I1 } & {\bf I2 } & {\bf I4 } & {\bf I8 } & {\bf F4 } & {\bf F8} \\ \hline
\verb=+=   & & \YES & \YES & \YES & \YES & \YES & \YES \\ \hline
\verb+-+   & & \YES & \YES & \YES & \YES & \YES & \YES \\ \hline
\verb+*+   & & \YES & \YES & \YES & \YES & \YES & \YES \\ \hline
\verb+/+   & & \YES & \YES & \YES & \YES & \YES & \YES \\ \hline
\verb+%+   & & \YES & \YES & \YES & \YES &      &      \\ \hline
\verb+&&+  & & \YES & \YES & \YES & \YES & & \\ \hline
\verb+||+  & & \YES & \YES & \YES & \YES & & \\ \hline
\verb+>+   & & \YES & \YES & \YES & \YES & \YES & \YES \\ \hline
\verb+<+   & & \YES & \YES & \YES & \YES & \YES & \YES \\ \hline
\verb+>=+  & & \YES & \YES & \YES & \YES & \YES & \YES \\ \hline
\verb+<=+  & & \YES & \YES & \YES & \YES & \YES & \YES \\ \hline
\verb+!=+  & & \YES & \YES & \YES & \YES & \YES & \YES \\ \hline
\verb+==+  & & \YES & \YES & \YES & \YES & \YES & \YES \\ \hline
\verb+..+  & concatenate & \YES & \YES & \YES & & & \\ \hline
\verb+||!+ & or not & \YES & \YES & \YES & \YES & & \\ \hline
\verb+&&!+ & and not & \YES & \YES & \YES & \YES & & \\ \hline
\verb+&+   & & \YES & \YES & \YES & \YES & & \\ \hline
\verb+|+   & & \YES & \YES & \YES & \YES & & \\ \hline
\verb+^+   & & & & & & & \\ \hline
\verb+<<+  & left shift & \YES & \YES & \YES & \YES & & \\ \hline
\verb+>>+  & right shift & \YES & \YES & \YES & \YES & & \\ \hline
\hline
\end{tabular}
\caption{f1f2opf3}
\label{tbl_f1f2opf3}
\end{table}



\section{xfer}
\label{xfer}
\bi
\item \(Fldtype(f_I) \in {I4, I8}\) 
\item \(Fldtype(f_1) \leftarrow Fldtype(f_2)\) 
\item \(Fldtype(f_2) \in \{I1, I2, I4,  I8, F4, F8, SC\}\)
\item \(0 \leq min(f_I) \leq max(f_I) < |T_2|\)
\ei

\subsection{pack}
\label{pack}

\be
\item \(FldType(f_0) \leftarrow \{I4, I8\}\) 
\item Options must be specified as set of integer ranges such that 
\be
\item ranges are non-overlapping 
\item minimum value of any range is 0
\item maximum value of any range is 31 for I4 and 62 for I8
\item All input fields must be non-negative
\item Each input field must be able to fit into the range allocated for it.
\ee

Let option be  \(\{(l_1, u_1), (l_2, u_2), \ldots\}\). This means that
\be
\item field \(f_1\) will be shifted to occupy bits \([l_1, u_1]\) of \(f_0\) 
\item field \(f_2\) will be shifted to occupy bits \([l_2, u_2]\) of \(f_0\) 
\item \ldots
\ee

\ee


\subsection{fop}
\label{fop}

Following operations are supported
\be
\item sort ascending 
\item sort decending 
\item permute (random)
\item saturate --- set minimum or maximum value (or both) to user-specified
\ee

\subsection{sortf1f2}
\label{sortf1f2}

Following operations are supported
\be
\item sort first field ascending, second field don't care 
\item sort first field ascending, second field ascending
\item sort first field descending, second field don't care 
\item sort first field descending, second field descending
\ee

%----------------------------------------------------
\subsection{Append one table to another}
\label{app_tbl}
\TBC
%----------------------------------------------------
\subsection{Percentiles}
\label{percentiles}
\TBC
%----------------------------------------------------
\subsection{Number in Range}
\label{num_in_range}
\TBC
%----------------------------------------------------
\subsection{t1f1t2f2opt3f3} 
\label{t1f1t2f2opt3f3} 
\TBC
%----------------------------------------------------
%----------------------------------------------------
\subsection{Copy Table}
\label{copy_tbl} 
\TBC
%----------------------------------------------------
\subsection{Copy Field}
\label{copy_fld} 
\TBC
%----------------------------------------------------
\subsection{Un-concatenate}
\label{f1opf2f3} 
\TBC
%----------------------------------------------------
%----------------------------------------------------
\subsection{Create Field by ``striding'' another Field}
\label{stride} 

\bi
\item 
Options are 
\be
\item {\bf op} set to {\tt stride} 
\item {\bf start} set to integer in \([0, |T_2|-1]\)
\item {\bf incr} set to positive integer \(< |T_2|/16\)
\ee
\item \(FldType(f_2) \in \) \NumericTypes
\item \(FldType(f_1) \leftarrow FldType(f_2)\)
\item \(f_1[i] = f_2[(start + i \times incr) \mathrm{~mod~} |T_2|]\) 
\ei
%----------------------------------------------------
\subsection{Create Field by ``sub-sampling'' another Field}
\label{subsample} 

\bi
\item 
Options are 
\be
\item {\bf op} set to {\tt subsample} 
\item {\bf nR} set to positive integer, \(0 < n_R < |T_2|/16\)
\item {\bf replacement} set to {\tt true} or {\tt false}
\ee
\item  If \(T_1\) exists, \(n_R = |T_1|\)
\item \(FldType(f_2) \in \) \NumericTypes
\item \(FldType(f_1) \leftarrow FldType(f_2)\)
\item \(f_1\) is created by picking items at random from \(f_2\).
Whether picks are made with or without replacement, depends on user
specification
\ei

\subsection{Counting}
\label{Counting}

\subsubsection{Counting (large NDV)}
\verb+count_vals+
\subsubsection{Counting (small NDV)}
\verb+count_ht+
\subsubsection{Counting (partial tables)}
\verb+countf+

\subsection{Join}
\label{join}
\TBC
%--------------------------------------------------------------
\subsection{dump}
\label{dump}

Arguments are
\be
\item name of file into which tables are dumped
\item name of file into which fields are dumped
\ee

Creates CSV files of meta-data.

TODO: Deal with reference count incremeneting, checking, decrementing

TODO: Deal with external fields from scope creation

TODO: Setting foreign keys woith SetMeta

TODO: Scopes and all the mess that goes with it, including locking, ...

