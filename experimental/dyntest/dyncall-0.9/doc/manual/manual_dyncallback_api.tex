%//////////////////////////////////////////////////////////////////////////////
%
% Copyright (c) 2007,2013 Daniel Adler <dadler@uni-goettingen.de>, 
%                         Tassilo Philipp <tphilipp@potion-studios.com>
%
% Permission to use, copy, modify, and distribute this software for any
% purpose with or without fee is hereby granted, provided that the above
% copyright notice and this permission notice appear in all copies.
%
% THE SOFTWARE IS PROVIDED "AS IS" AND THE AUTHOR DISCLAIMS ALL WARRANTIES
% WITH REGARD TO THIS SOFTWARE INCLUDING ALL IMPLIED WARRANTIES OF
% MERCHANTABILITY AND FITNESS. IN NO EVENT SHALL THE AUTHOR BE LIABLE FOR
% ANY SPECIAL, DIRECT, INDIRECT, OR CONSEQUENTIAL DAMAGES OR ANY DAMAGES
% WHATSOEVER RESULTING FROM LOSS OF USE, DATA OR PROFITS, WHETHER IN AN
% ACTION OF CONTRACT, NEGLIGENCE OR OTHER TORTIOUS ACTION, ARISING OUT OF
% OR IN CONNECTION WITH THE USE OR PERFORMANCE OF THIS SOFTWARE.
%
%//////////////////////////////////////////////////////////////////////////////

\newpage
\section{\emph{Dyncallback} C library API}

This library extends \product{dyncall} with function callback support, allowing
the user to dynamically create a callback object that can be called directly,
or passed to functions expecting a function-pointer as argument.\\
\\
Invoking a \product{dyncallback} calls into a user-defined unified handler that 
permits iteration and thus dynamic handling over the called-back-function's
parameters.\\
\\
The flexibility is constrained by the set of supported types, though.\\
\\
For style conventions and supported types, see \product{dyncall} API section.
In order to use \product{dyncallback}, include {\tt "dyncall\_callback.h"}.

\subsection{Callback Object}

The \emph{Callback Object} is the core component to this library.

\paragraph{Types}

\begin{lstlisting}[language=c]
typedef struct DCCallback DCCallback;
\end{lstlisting}

\paragraph{Details}
The \emph{Callback Object} is an object that mimics a fully typed function
call to another function (a generic callback handler, in this case).\\
\\
This means, a pointer to this object is passed to a function accepting a pointer
to a callback function \emph{as the very callback function pointer itself}.
Or, if called directly, cast a pointer to this object to a function pointer and
issue a call.


\subsection{Allocation}

\paragraph{Functions}

\begin{lstlisting}[language=c]
DCCallback* dcbNewCallback(const char*        signature,
                           DCCallbackHandler* funcptr,
                           void*              userdata);
void dcbFreeCallback(DCCallback* pcb);
\end{lstlisting}

\lstinline{dcbNewCallback} creates and initializes a new \emph{Callback} object,
where \lstinline{signature} is the needed function signature (format is the
one outlined in the language bindings-section of this manual, see Table \ref{sigchar})
of the function to mimic, \lstinline{funcptr} is a pointer to a callback handler,
and \lstinline{userdata} a pointer to custom data that might be useful in the
handler.
Use \lstinline{dcbFreeCallback} to destroy the \emph{Callback} object.\\
\\
As with \capi{dcNewCallVM}/\capi{dcFree}, this will allocate memory using the
system allocators or custom overrides.


\subsection{Callback handler}

The unified callback handler's declaration used when creating a \capi{DCCallback}
is:

\begin{lstlisting}
char cbHandler(DCCallback* cb,
               DCArgs*     args,
               DCValue*    result,
               void*       userdata);
\end{lstlisting}

\capi{cb} is a pointer to the \capi{DCCallback} object in use, \capi{args} allows
for dynamic iteration over the called-back-function's arguments (input) and
\capi{result} is a pointer to a \capi{DCValue} object in order to store the
callback's return value (output, to be set by handler).\\
Finally, \capi{userdata} is a pointer to some user defined data that can be
set when creating the callback object.
The handler itself returns a signature character (see Table \ref{sigchar}) specifying the
data type used for \capi{result}.

