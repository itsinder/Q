%//////////////////////////////////////////////////////////////////////////////
%
% Copyright (c) 2012 Daniel Adler <dadler@uni-goettingen.de>, 
%                    Tassilo Philipp <tphilipp@potion-studios.com>
%
% Permission to use, copy, modify, and distribute this software for any
% purpose with or without fee is hereby granted, provided that the above
% copyright notice and this permission notice appear in all copies.
%
% THE SOFTWARE IS PROVIDED "AS IS" AND THE AUTHOR DISCLAIMS ALL WARRANTIES
% WITH REGARD TO THIS SOFTWARE INCLUDING ALL IMPLIED WARRANTIES OF
% MERCHANTABILITY AND FITNESS. IN NO EVENT SHALL THE AUTHOR BE LIABLE FOR
% ANY SPECIAL, DIRECT, INDIRECT, OR CONSEQUENTIAL DAMAGES OR ANY DAMAGES
% WHATSOEVER RESULTING FROM LOSS OF USE, DATA OR PROFITS, WHETHER IN AN
% ACTION OF CONTRACT, NEGLIGENCE OR OTHER TORTIOUS ACTION, ARISING OUT OF
% OR IN CONNECTION WITH THE USE OR PERFORMANCE OF THIS SOFTWARE.
%
%//////////////////////////////////////////////////////////////////////////////

\subsection{SPARC Calling Convention}

\paragraph{Overview}

The SPARC family of processors is based on the SPARC instruction set architecture, which comes in basically tree revisions,
V7, V8 and V9. The former two are 32-bit whereas the latter refers to the 64-bit SPARC architecture (see next chapter). SPARC is big endian.\\

\paragraph{\product{dyncall} support}

\product{dyncall} fully supports the SPARC 32-bit instruction set (V7 and V8), \product{dyncallback} support is missing, though.

\subsubsection{SPARC (32-bit) Calling Convention}

\paragraph{Register usage}

\begin{itemize}
\item 32 32-bit integer/pointer registers
\item 32 floating point registers (usable as 8 quad precision, 16 double precision or 32 single precision registers)
\item 32 registers are accessible at a time (8 are global ones (g*), whereas the rest forms a register window with 8 input (i*), 8 output (o*) and 8 local (l*) ones)
\item invoking a function shifts the register window, the old output registers become the new input registers (old local and input ones are not accessible anymore)
\end{itemize}

\begin{table}[h]
\begin{tabular*}{0.95\textwidth}{lll}
Name                                 & Alias                   & Brief description\\
\hline
{\bf \%g0}                           &                         & Read-only, hardwired to 0 \\
{\bf \%g1-\%g7}                      &                         & Global \\
{\bf \%o0 and \%i0}                  &                         & Output and input argument 0, return value \\
{\bf \%o1-\%o5 and \%i1-\%i5}        &                         & Output and input argument registers \\
{\bf \%o6 and \%i6}                  &                         & Stack and frame pointer \\
{\bf \%o7 and \%i7}                  &                         & Return address (caller writes to o7, callee uses i7) \\
\end{tabular*}
\caption{Register usage on sparc calling convention}
\end{table}

\paragraph{Parameter passing}
\begin{itemize}
\item Stack parameter order: right-to-left
\item Caller cleans up the stack @@@ really?
\item Stack always aligned to 8 bytes.
\item first 6 integers and floats are passed independently in registers using \%o0-\%o5
\item for every other argument the stack is used
\item @@@ what about floats, 64bit integers, etc.?
\item results are returned in \%i0, and structs/unions pass a pointer to them as a hidden stack parameter (see below)
\end{itemize}

\paragraph{Stack layout}

Stack directly after function prolog:\\

\begin{figure}[h]
\begin{tabular}{5|3|1 1}
\hhline{~-~~}
                                   & \vdots                      &                                &                               \\
\hhline{~=~~}
local data                         & \hspace{4cm}                &                                & \mrrbrace{10}{caller's frame} \\
\hhline{~-~~}
padding                            &                             &                                &                               \\
\hhline{~-~~}
\mrlbrace{7}{parameter area}       & argument x                  & \mrrbrace{3}{stack parameters} &                               \\
                                   & \ldots                      &                                &                               \\
                                   & argument 6                  &                                &                               \\
                                   & input argument 5 spill      & \mrrbrace{3}{spill area}       &                               \\
                                   & \ldots                      &                                &                               \\
                                   & input argument 0 spill      &                                &                               \\
                                   & struct/union return pointer &                                &                               \\
\hhline{~-~~}
register save area (\%i* and \%l*) &                             &                                &                               \\
\hhline{~=~~}
local data and padding             &                             &                                & \mrrbrace{3}{current frame}   \\
\hhline{~-~~}
parameter area                     &                             &                                &                               \\
\hhline{~-~~}
                                   & \vdots                      &                                &                               \\
\hhline{~-~~}
\end{tabular}
\\
\\
\\
\caption{Stack layout on sparc calling convention}
\end{figure}

