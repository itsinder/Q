\documentclass[12pt,timesnewroman,letterpaper]{article}

%%%%%%%%%%%%%%%%%%%%%%%%%%%%%%%%%%%%%%%%%%%%%%%%%%%%%%%%%%%%%%%%%%%%%%%%%%%%%%%%%%
% Special for e-unibus doc commands

\newcommand{\ForLater}{
\begin{center}
{\bf NOT FOR CURRENT VERSION}
\end{center}
}
\newcommand{\TBC}{\framebox{\textbf{TO BE COMPLETED}}}
\newcommand{\DISCUSS}{\Ovalbox {\bf \textcolor{red}{FOR DISCUSSION}}}
\newcommand{\Input}{\framebox{\textsf{in}}}
\newcommand{\Output}{\framebox{\textsf{out}}}
\newcommand{\debug}[1]{\textbf{debug start} #1 \textbf{debug finish}}
\newcommand{\inx}[1]{\emph{#1}}
\newtheorem{notation}{Notation}
\newtheorem{definition}{Definition}
\newtheorem{problem_statement}{Problem Statement}
\newtheorem{invariant}{Invariant}
\newtheorem{assumption}{Assumption}
\newtheorem{resource_string}{Resource String}
\newtheorem{testcase}{Test Case}
\newtheorem{note}{Note}
\newtheorem{specification}{Specification}
\newtheorem{caution}{Caution}
\newtheorem{prereq}{Pre-requisite}
\newtheorem{action}{Action}
\newtheorem{query}{Query}
\newcommand{\be}{\begin{enumerate}}
\newcommand{\ee}{\end{enumerate}}
\newcommand{\bi}{\begin{itemize}}
\newcommand{\ei}{\end{itemize}}
\newcommand{\bv}{\begin{verbatim}}
\newcommand{\ev}{\end{verbatim}}
\newcommand{\bd}{\begin{description}}
\newcommand{\ed}{\end{description}}
\newcommand{\bpre}{\begin{prereq}}
\newcommand{\epre}{\end{prereq}}
\newcommand{\bact}{\begin{action}}
\newcommand{\eact}{\end{action}}
\newcommand{\bs}{\begin{specification}}
\newcommand{\es}{\end{specification}}
\newcommand{\btc}{\begin{testcase}}
\newcommand{\etc}{\end{testcase}}
\newcommand{\bc}{\begin{caution}}
\newcommand{\ec}{\end{caution}}
\newcommand{\la}{\leftarrow}
\newcommand{\IpArgs}{\subsection{Input Arguments}}
\newcommand{\PreReqs}{\subsection{Pre-requisities}}
\newcommand{\Actions}{\subsection{Actions}}
\newcommand{\Coverage}{{\bf To test coverage.}}

%%%%%%%%%%%%%%%%%%%%%%%%%%%%%%%%%%%%%%%%%%%%%%%%%%%%%%%%%%%%%%%%%%%%%%%%%%%


\newtheorem{theorem}{Theorem}[section]
\newtheorem{lemma}[theorem]{Lemma}
\newtheorem{proposition}[theorem]{Proposition}
\newtheorem{corollary}[theorem]{Corollary}

\newenvironment{proof}[1][Proof]{\begin{trivlist}
\item[\hskip \labelsep {\bfseries #1}]}{\end{trivlist}}
\newenvironment{intuition}[1][Intuition]{\begin{trivlist}
\item[\hskip \labelsep {\bfseries #1}]}{\end{trivlist}}
%% \newenvironment{definition}[1][Definition]{\begin{trivlist}
%% \item[\hskip \labelsep {\bfseries #1}]}{\end{trivlist}}
\newenvironment{example}[1][Example]{\begin{trivlist}
\item[\hskip \labelsep {\bfseries #1}]}{\end{trivlist}}
\newenvironment{remark}[1][Remark]{\begin{trivlist}
\item[\hskip \labelsep {\bfseries #1}]}{\end{trivlist}}

\newcommand{\qed}{\nobreak \ifvmode \relax \else
      \ifdim\lastskip<1.5em \hskip-\lastskip
      \hskip1.5em plus0em minus0.5em \fi \nobreak
      \vrule height0.75em width0.5em depth0.25em\fi}

%%%%%%%%%%%%%%%%%%%%%%%%%%%%%%%%%%%%%%%%%%%%%%%%%%%%%%%%%%%%%%%%%
% \newcommand{\Alogon}{\mbox{\fontfamily{ptm}\selectfont {\large \selectfont A} \hspace{-1.2ex} {\large \selectfont L} \hspace{-2.3ex} \raisebox{0.45ex}{ {\footnotesize \selectfont O} } \hspace{-1.80ex} {\large \selectfont G} \hspace{-1.80ex} \raisebox{-0.33ex}{ {\large \selectfont O} } \hspace{-1.8ex} {\large \selectfont N}}}


\usepackage{times}
\usepackage{helvet}
\usepackage{courier}
\usepackage{fancyheadings}
\pagestyle{fancy}
\usepackage{pmc}
\usepackage{graphicx}
\setlength\textwidth{6.5in}
\setlength\textheight{8.5in}
\begin{document}
\title{Logistic Regression in Q}
\author{ Ramesh Subramonian }
\maketitle
\thispagestyle{fancy}
\lfoot{{\small Data Analytics Team}}
\cfoot{}
\rfoot{{\small \thepage}}

\section{Introduction}

\TBC

\section{Data Structures}

\bi
\item \(X\) is an \(N \times M\) matrix containing the input data.
  \(X_{i, j}\) is value of \(j^{th}\) attribute of \(i^{th}\)
  instance. \(X\) is stored as \(M\) columns, where \(X_j\) is
  observations for attribute \(j\)
\item \(y\) is an \(N \times 1\) classification vector. \(y_i\) is
  classification of instance \(i\) and can be 1 or 0.
\item \(\beta\) is an \(M \times 1\) coefficient vector. \(\beta_j\)
  is coefficient for attribute \(j\). This is stored as a Lua table\footnote{In
    this presentation, we skip details such as Lua tables being indexes from 1
  and not 0}
  from 1
\item \(w\) is actually an \(N \times N\) matrix. However, since
  off-diagonal elements are zero, we can represent is as an \(N \times
  1\) vector.  \ei

We start with an initial guess for \(\beta\) and then calculate better
approximations using Equation~\ref{eqn1} until convergence.
\begin{equation}
\label{eqn1}
(X^T W X) \times \beta ^{new} = X^T W ( X \beta + W^{-1}(y - p) )
\end{equation}
To see the solver at the heart of each iteration, re-write Equation~\ref{eqn1} as Equation~\ref{eqn2}
\begin{equation}
\label{eqn2}
A x = b
\end{equation}
where 
\be
\item \(A\) is an \(M \times M\) symmetric matrix, \((X^T W X)\)
\item \(x\) is an \(M \times 1\) vector, which are the new
  coefficients, \(\beta^{new}\), that we solve for in each iteration
\item \(b\) is an \(M \times 1\) vector, \(X^T W ( X \beta + W^{-1}(y - p) )\)
\ee

\section{computations}

Step by step computations in Table~\ref{step_by_step_calc}.
\begin{table}[ht]
\centering
\begin{tabular}{|l|l|l|l|} \hline \hline
  {\bf Name} & {\bf Description} & {\bf Type} & {\bf Code} \\ \hline \hline
  \(t_1\) & \(X \beta\) & \((N \times M) \times (M \times 1)\)  &
  Section~\ref{t1} \\ 
  & & \(= N \times 1\) & \\ \hline
  \(p\) & & \(N \times 1\) & \( p = \mathrm{logit}(x)\) \\ \hline
  \(t_2\) &  \(y - p\) & \(N \times 1\) & \( t_2 = \mathrm{sub}(y, p)\) \\ \hline
\(w\) & & \(N \times 1\) & \( w = \mathrm{fn}(x)\) \\ \hline
  \(t_3\) & \((y-p)/w\) & \(N \times 1\) & \(t_3 = \mathrm{div}(t_2, p)\) \\ \hline
  \(t_4\) & \(X \beta + w^{-1}(y-p)\) & \(N \times 1\) & \(\mathrm{add}(t_1, t_3)\) \\ \hline
  \(t_5\) & \(w(X \beta + w^{-1}(y-p))\) & \(N \times 1\) & \(\mathrm{mul}(w, t_4)\) \\ \hline
  \(b\) & \(X^T w(X \beta + w^{-1}(y-p))\) & \((M\times N) \times (N \times 1)\)
  & \(\forall j_{j=1}^{j=M} b_j = \) \\ 
        & & \( = M \times 1 \) & \(\mathrm{sumprod}(X_j, t_5)\) \\ \hline
  \(A\) & \(X^T W X\) & \((M \times N) \times (N \times N) \times (N \times M)\)
  & \(\forall_{j=1}^{j=M} \forall_{k=j}^{k=M} A_{j, k} = \) \\ 
  & & \(= (M \times M)\) & \(\mathrm{sumprod2}(X_j, w, X_k)\) \\ \hline
  \hline

\hline
\end{tabular}
\caption{Listing of individual steps and intermediate values}
\label{step_by_step_calc}
\end{table}

\begin{table}[hb]
\centering
\begin{tabular}{|l|l|l|l|l|} \hline \hline
  {\bf Name} & {\bf Input Type} & {\bf Arguments} & {\bf Return Value} \\ \hline \hline
  logit & Scalar & \(x\) & \(\frac{e^x}{1 + 1^x}\) \\ \hline
  add & Scalar & \(x, y\) & \(x + y \)  \\ \hline
  sub & Scalar & \(x, y\) & \(x - y \)  \\ \hline
  mul & Scalar & \(x, y\) & \(x \times y \)  \\ \hline
  div & Scalar & \(x, y\) & \(x / y \)  \\ \hline
  fn & Scalar & \(x\) & \(\frac{1}{1 - x}\) \\ \hline
  sumprod & Vector & \(x, y\) & \(\sum_i (x_i \times y_i)\) \\ \hline
  sumprod2 & Vector & \(x, y, z\) & \(\sum_i (x_i \times y_i \times z_i)\) \\ \hline
\hline
\end{tabular}
\caption{Necessary Operators}
\label{tbl_custom_ops}
\end{table}

\subsection{t1}
\label{t1}

\begin{verbatim}
t1 = constant(0, N, "int64_t",) -- N by 1 vector 
for j, beta_j in ipairs(beta) do
  t1 = add(t1, mul(Xj, beta_j) -- 
end
\end{verbatim}

\section{Details}

\subsection{Notes}

\be
\item The calculation of \(A\) is simplified by the fact that the off-diagonal
  elements of \(w\) are 0 and that it is a symmetric matrix
\ee
\subsection{Clarifications needed}

\be
\item 
There was something with ``affine'' which required an additional column for
\(X\) to consist of all 1's? \TBC
\item Initial guess for \(\beta\)
\ee

\end{document}
