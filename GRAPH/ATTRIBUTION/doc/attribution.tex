\startreport{Network Influence Score}
\reportauthor{Ramesh Subramonian}
\section{Introduction}

\TBC

\subsection{Selected Q operators}
See Table~\ref{tbl_q_ops}.
\begin{table}[ht]
\centering
\begin{tabular}{|l|l|l|} \hline \hline
{\bf operator} & {\bf explanation} & {\bf reference} \\ \hline \hline
get\_val\_by\_idx & get value by index & 
Section~\ref{get_val_by_idx} \\ \hline

set\_sclr\_val\_by\_idx & set scalar value by index & 
Section~\ref{set_sclr_val_by_idx} \\ \hline

add\_vec\_val\_by\_idx & Accumulate vector value by index & 
Section~\ref{add_vec_val_by_idx} \\ \hline

vvmul & vector-vector multiply & Section~\ref{vvmul} \\ \hline

vsleq & vector-scalar \(\leq\) comparison & Section~\ref{vsleq} \\ \hline

ifxthenyelsez & ternary operator \verb+ x ? y : z+ & 
Section~\ref{ifxthenyelsez} \\ \hline

where & select rows where condition is true & 
Section~\ref{where} \\ \hline
\hline
\end{tabular}
\caption{Selected Q operators}
\label{tbl_q_ops}
\end{table}

\section{Data Structures}
\subsection{T0}

\(T_0\) has the following columns, where each row refers to a person.
\be
\item \(x\) --- a sequence \(0, 1, \ldots\)
\item \(y\), person who did the referring. 
\item id, original ID 
\item \(r\), dollar amount spent 
\item \(s\), dollar amount attributed, {\em created by algorithm}
\item \(d\), max depth of referral chain ending in this person, {\em created by algorithm}
\ee
Notes:
\bi
\item \(y[i] = j \Rightarrow\) \(i\) was referred by \(j\). 
We also say that \(y\) is the 1-parent of \(i\)
\item 
Assume that \(y[i] = j\) and \(y[j] < 0\). 
This implies \(i\) was referred by \(j\) but nobody referred \(j\)
\item a person can be referred by at most one person 
\ei

\subsection{T1}

\(T_1\) is \(T_0\) restricted to persons who were refered. Specifically,
\bi
\item \(v = \mathbf{Q.vsgeq}(T_0.y, 0)\)
\item \(T_1.x = \mathbf{Q.where}(T_0.x, v)\)
\item \(T_1.y = \mathbf{Q.where}(T_0.y, v)\)
%% DELETE \item \(T_1.r = \mathbf{Q.where}(T_0.r, v)\)
%% DELETE \item \(T_1.d = \mathbf{Q.const}(1)\) 
%% DELETE --- initial assumption on depth of chain ending in \(x\)
\ei

\subsection{Tk}

Tables \(T_k\) for \(k =  2, \ldots\) have the following columns
\be
\item \(x\), person who was referred
\item \(y\), person who did the referring at a distance of \(k\)
\ee

Note that if \(a\) referred \(b\) and \(b\) referred \(c\), then table \(T_2\)
would have a row of the form \((c, a)\)

\subsection{Sample Data}
An example is in Table~\ref{example_extend}. 
\begin{table}[hp]
\centering
\begin{tabular} {|l|l|l|l|l|} \hline \hline
{\bf k} & {\bf x} & {\bf y} & {\bf z} & {\bf w} \\ \hline
0 & 0 & 1  & --- & --- \\ \hline
0 & 1 & 2  & --- & --- \\ \hline
0 & 2 & 3  & --- & --- \\ \hline
0 & 3 & 4  & --- & --- \\ \hline
0 & 4 & -1 & --- & --- \\\hline
0 & 5 & 6  &  --- & --- \\\hline
0 & 6 & 7  &  --- & --- \\\hline
0 & 7 & -1 &  --- & --- \\\hline
0 & 8 & -1 &  --- & --- \\\hline
\hline  \hline
1 & 0 & 1  & 2  & 1 \\ \hline
1 & 1 & 2  & 3  & 1 \\ \hline
1 & 2 & 3  & 4  & 1 \\ \hline
1 & 3 & 4  & -1 & 0 \\ \hline
1 & 5 & 6  & 2  & 1 \\ \hline
1 & 6 & 7  & -1 & 0 \\ \hline
\hline  \hline
2 & 0 & 2 & 3 & 1 \\ \hline
2 & 1 & 3 & 4 & 1 \\ \hline
2 & 2 & 4 & -1 & 0 \\ \hline
2 & 5 & 7 & -1 & 0 \\ \hline
\hline  \hline
3 & 0 & 2 & 4 & 1 \\ \hline
3 & 1 & 3 & -1 & 0 \\ \hline
\hline  \hline
4 & 0 & 2 & -1 & 0 \\ \hline
\end{tabular}
\caption{Sample data}
\label{example_extend}
\end{table}


\section{Algorithm}

The algorithm is presented in Figure~\ref{algo}. Some useful routines in
Section~\ref{aux_algo}
\subsection{Sub-routines}
\label{aux_algo}

\bi
\item The function {\bf Q.extend} is in Figure~\ref{extend}
\item The way the attribution diminishes with distance is in Table~\ref{attenuate}
\ei

\begin{figure}
\centering
\fbox{
\begin{minipage}{12 cm}
\centering
\begin{tabbing} 
\hspace*{0.25in} \= \hspace*{0.25in} \= \hspace*{0.25in} \= \kill
Inputs: \+ \\
\(T_k.x\) \\
\(T_k.y\) \\
\(T_0.y\) \-\\
Operation:  \+ \\
Let \(T_k.x[i] = x', T_k.y[i] = y'\).  \\
Let \(f' = T_0.y[y']\).  \\
If \(f' \geq 0\), add \((x', f')\) to output; else, do nothing \- \\
Implementation in Q:  \+ \\
\(z = \mathbf{Q.get}(T_k.y, T_0.y)\) \\
\(w = \mathbf{Q.vsgeq}(T_k.y, 0)\) \\
\(T_{k+1}.x = \mathbf{Q.where}(T_k.x, w)\) \\
\(T_{k+1}.y = \mathbf{Q.where}(T_k.z, w)\) \- \\
\end{tabbing}
\caption{extend}
\label{extend}
\end{minipage}
}
\end{figure}


\begin{table}[hp]
\centering
\begin{tabular}{|l|l|l|l|l|l|l|} \hline \hline
\(d_{max}\) & \(\alpha_1\)  & \(\alpha_2\)  & \(\alpha_3\)  & \(\alpha_4\)
\\ \hline
1 & 1 & --- & --- & --- \\ \hline
2 & 2/3 & 1/3 & --- & --- \\ \hline
3 & 4/7 & 2/7 & 1/7 & --- \\ \hline
4 & 8/15 & 4/15 & 2/15 & 1/15 \\ \hline
\hline
\end{tabular}
\caption{Attenuation Factors for different \(d, d_{max}\)}
\label{attenuate}
\end{table}


\begin{figure}
\centering
\fbox{
\begin{minipage}{12 cm}
\centering
\begin{tabbing} 
\hspace*{0.25in} \= \hspace*{0.25in} \= \hspace*{0.25in} \= \kill

\(T_0.d = \mathrm{Q.const}(1)\) --- initialize max depth to 1 \\
\(T_0.s = \mathrm{Q.const}(0)\) --- initialize attributed revenue to 0 \\
\(k = 1\) \\
{\bf repeat } \+ \\
\(T_{k+1} = \mathbf{Q.extend}(T_k.x, T_k.y, T_0.y)\) (Figure~\ref{extend})\\
{\bf if} \(T_{k+1} = \bot\) {\bf then break else} \(k = k+1\) {\bf end}\\
--- Update max depth of referral chain ending in each person \\
\(\mathbf{Q.set\_sclr\_val\_by\_idx}(T_{k+1}.x, T0.d)\) \- \\
{\bf until} --- not possible to extend any further \\
\(d_{max} = k\) \\
Load \(alpha\) which contains attribution coefficients (Table~\ref{attenuate}\\
{\bf for} \(k = 1, \ldots k_{max}\) {\bf do} \+ \\
  \(T_k.d       = \mathbf{Q.get}(T_k.x, T_0.d)\)  \\
  \(T_k.r       = \mathbf{Q.get}(T_k.x, T_0.r)\)  \\
  \(T_k.\alpha  = \mathbf{Q.get\_val\_by\_idx}(T_k.d, \alpha[k])\) \\
  \(s           = \mathbf{Q.vvadd}(T[k].r, T[k].\alpha)\) \\
  \(\mathbf{Q.add\_vec\_val\_by\_idx}(s, T_k.y, T0.s)\) \- \\
{\bf endfor} \\
\end{tabbing}
\caption{Pseudo-code}
\label{algo}
  \end{minipage}
}
\end{figure}

\section{Details of Q Operators}
\label{Q_ops}

\subsection{get\_val\_by\_idx}
\label{get_val_by_idx}
\bi
\item Inputs: (1) \(x\), Vector  (2) \(y\), Vector

\item Outputs: \(z\), Vector, same length as \(x\)

\item Operation: \(z[i] \leftarrow y[x_i]\)
\ei

\subsection{set\_sclr\_val\_by\_idx}
\label{set_sclr_val_by_idx}
\bi
\item Inputs: (1) \(x\), Vector  (2) \(s\), Scalar

\item Modified: \(z\), Vector

\item Operation: \(x[i] = j \Rightarrow z[j] \leftarrow s\)
\ei

\subsection{add\_vec\_val\_by\_idx}
\label{add_vec_val_by_idx}
\bi
\item Inputs: (1) \(x\), Vector  (2) \(y\), Vector, same length

\item Modified: \(z\), Vector \(\forall i: 0 \leq y[i] \leq z:\mathrm{length}() - 1\)

\item Operation: \(x[i] = j \Rightarrow z[j] \leftarrow z[j] + y[i]\)
\ei


\subsection{vvmul}
\label{vvmul}
\bi
\item Inputs: (1) \(x\), Vector  (2) \(y\), Vector, same length as \(x\)

\item Outputs: \(z\), Vector, same length as \(x\)

\item Operation: \(z[i] \leq x[i] + y[i]\)
\ei

\subsection{vsleq}
\label{vsleq}
\bi
\item Inputs: (1) \(x\), Vector  (2) \(y\), Scalar

\item Outputs: \(z\), bit-Vector, same length as \(x\)

\item Operation: \(x[i] \leq y \Rightarrow z[i] = 1\); else, \(z[i] = 0\)
\ei

%% \subsection{ainb}
%% \bi
%% \item Inputs: (1) \(x\), Vector  (2) \(y\), Vector
%% 
%% \item Outputs: \(z\), bit-Vector of same length as \(x\)
%% 
%% \item Operation: \(z[i] = 1\) if \(\exists j: x[i] = y[j]\); else, 0
%% 
%% \ei

\subsection{ifxthenyelsez}
\label{ifxthenyelsez}
\bi
\item Inputs: (1) \(x\), bit-Vector (2) \(y\), Scalar (3) \(z\), Vector, all of
the same length

\item Outputs: \(w\) Vector of same length and type as \(y\)

\item Operation: \(x[i] = 1 \Rightarrow w[i] \leftarrow y\); else, \(w[i]
\leftarrow z[i]\)

\ei


\subsection{where}
\label{where}
\bi
\item Inputs: (1) \(x\), Vector  (2) \(y\), bit-Vector, same length as \(x\)

\item Outputs: \(z\), Vector of same type as \(x\)

\item Operation: \(z\) contains \(x[i]\) only if
\(y[i] = 1\). Order of elements in  \(z\) same as \(x\)

\ei

\appendix
Some bogus reference to make bibtex not complain \cite{Hastie2009}.
\bibliographystyle{alpha}
\bibliography{../../../DOC/ref}
