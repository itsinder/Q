\section{Introduction}

It is undeniable that there has been an explosive growth in the 
amount of data available for analysis. Solutions that 
deal with the reliable storage of large amounts of data have seen tremendous
improvements, as also tools to analyze large data sets. What has sometimes been
obscured is the understanding of the actual reality of the life of a data
scientist or analyst. More often than not, the actual amount of data that needs
to be analyzed seldom exceeds a few tera bytes\cite{Dittrich2015}. The user may spend
several months on this sliver of data, augmenting it or eliminating parts of it
as insights emerge. Therefore, the mantra of ``velocity, volume and variety'',
while entirely true for data in general, does not translate as readily to the
sustained, concentrated effort that goes into most work that we have
encountered. The familiarity gained about the nature of the data 
while working on a given problem for a length of time should not be discounted. The
user becomes aware of what approximations can be made, what dynamic range is
truly needed, at what point can a value be saturated, etc.

The fact that we are better off scaling up before scaling out has been suggested
by several authors \cite{Rowstron2012,Dittrich2015}. 
%Kyrola2012
Fault tolerance on large distributed systems is a difficult
problem and adds significant system complexity but is it really needed?
If we consider the mean time between failure of a single server and
the cost of a ``re-do'' in the event of failure,
it is hard to justify that investment. 
Further, the emergence of GPUs --- both their increased compute power and the
increased  bandwidth to the GPU --- pack the single server solution with
considerable punch.

In this paper, we present Q, our answer to both technological trends and actual
user needs. Q is an analytical database that invites the participation of the
user in tuning performance and writing new operators. 
%% Its architecture makes it amenable to integrating it in 
%% other eco-systems such as Jupyter notebooks.

The simplicity and power of Q does not come for free. We have abandoned many
cherished tenets of databases, like fault tolerance, multi-tenancy, concurrency,
etc. While it is hard to argue against these features, our experience leads us
to question as to whether they truly assist the work of the data scientist, the
business analyst, the machine learning model builder. Maybe its time that
analytical databases look more like Jupyter notebooks on steroids.

Due to the page limitation, we have heavily edited code fragments and
have substituted the insights gained from performance experiments instead of
performance graphs.
