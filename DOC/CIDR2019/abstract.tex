\begin{abstract}
The unspoken secret of ``big data'' is that while data warehouses are
exploding in size, the actual amount of data needed for many analytics and
machine learning problems is better characterized as ``medium data''. We define
``medium data'' as data large enough that one must be cognizant of performance but small
enough that a single machine is adequate for interactive usage. This realization has motivated the
design and development of Q, an analytical database, with a rich language of
operators for the manipulation of data.  The language is inspired by
Ousterhout's dichotomy, using a scripting language (Lua) with a fast interpreter
(LuaJIT) as the front-end and a compiled language (C) for the back-end.
A fundamental realization that has influenced Q's design is that data
practitioners often build up tremendous intuition about the data sets that they
are working with. Q encourages the use of this intuition to both 
gain performance and
simplify code.
Q exposes its software architecture, thereby (1) making it easy to integrate Lua
code into the engine when tinkering and (2)
providing performance gains to the more adventurous C programmer 
The other central principle guiding Q's development was that the life of a data
scientist involves a lot of trial and error on large data sets. Providing a
debugging environment, with enough performance to make it interactive, was a
basic driver of this effort.

\end{abstract}
