\begin{abstract}
The unspoken secret of ``big data'' is that while data warehouses are
exploding in size, the actual amount of data needed for many analytics and
machine learning problems is better characterized as ``medium data''. We define
``medium data'' as data large enough that one must be cognizant of performance but small
enough that a single machine is adequate. This realization has motivated the
design and development of Q, an analytical database, with a rich language of
operators for the manipulation of data.  The language is inspired by
Ousterhout's dichotomy, using a scripting language (Lua) with a fast interpreter
(LuaJIT) as the front-end and a compiled language (C) for the back-end.
A fundamental realization that has influenced Q's design is that data
practitioners often build up tremendous intuition about the data sets that they
are working with. Q encourages the use of this intuition to gain performance.
Q exposes its software architecture to the more
adventurous programmer, providing performance gains to the programmer willing to
invest more time in relatively simple tweaks.
By using a simple data model that is accessible to the programmer, Q makes it
easy to integrate high performance custom code when necessary. Lastly, a central
goal of this effort was to make it possible to interactively debuf large data
sets. 

\end{abstract}
