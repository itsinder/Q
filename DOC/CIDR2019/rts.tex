\section{Q's Run Time System}

Q is a high-performance ``almost-relational'' 
analytical, single-node, column-store database. 
\be
\item 
By ``analytical'' we mean that data changes at the user's behest (e.g.,
loading a data set) but is not subject to external events.
\item 
By ``almost-relational'' we mean that it would more correctly
be called a ``tabular model'' \cite{Codd1982}. As Codd states, ``Tables are
at a lower level of abstraction than relations, since they give
the impression that positional (array-type) addressing is applicable
(which is not true of \(n\)-ary relations), and they fail to
show that the information content of a table is independent
of row order. Nevertheless, even with these minor flaws,
tables are the most important conceptual representation of
relations, because they are universally understood.''
\item By ``single-node'', we mean that Q does not distribute computation across
  machines. The flow of execution is inherently single-threaded. However,
  pthreads and OpenMP are heavily used {\em within} individual operations so as
  to exploit multi-core systems as well as GPUs.
\item By ``column-store'', we mean that 
Q provides the Lua programmer with the Vector type, each
individual element of which is a Scalar. A table in Q is a Lua
table of Vectors (Section~\ref{vectors_versus_tables}), where a Lua table is equivalent
to a Python dictionary.

\ee

\subsection{Q as a Lua extension}

While it is useful to think of Q as a functional language, 
it is more accurately described as a C library, embedded within an interpreted
language.
We chose to embed within an existing language for two reasons. One is that we did not have to
write a custom compiler. More importantly, is that it allowed us to leverage a rich eco-system of libraries, debuggers
\ldots
yet seamlessly move between that world and the high performance capabilities of
Q. 

We chose Lua
because it was designed specifically as both an embedding and embedded language
\cite{Lua2011} and it had a wickedly fast interpreter, LuaJIT.
We experimented with several approaches to invoking C from Lua. These included
(a) dyncall \cite{Adler2013} (b) the native Lua C API (c) LuaFFI (d) LuaJIT's
FFI. Taking the native Lua C API as the baseline, dyncall was (surprisingly) the worst (twice
as slow) and LuaJIT was (unsurprisingly) the fastest (50 times as fast!).

Data is either stored in memory and, when necessary, persisted (uncompressed) in
binary format to the file system. This allows us to quickly access data by
mmapp-ing the file. This alllows one to save and restore a session across a
reboot of the server.

\section{Vectors}
\label{vectors}

A vector is essentially a map \(f(i)\) such that given \(i, 0 \leq i < n\), it
returns a value of a given type. The main types that \Q\ supports four variants
of integers (1, 2, 4, and 8 byte) and two variations of floating point (single
and double precision). In addition, it supports bit vectors, constant length
strings. There is limited support for variable length strings, which are used
primarily as dictionaries.

\input{scalars}

\section{Reducers}
\label{reducers}

A reducer is a construct that takes one or more vectors as input and produces
one or more scalars as output. The simplest Reducers are those for min, max, sum
and other associative functions. In this case, the reducer returns 2 values,one
being the value of the function being computed and the other being the number of
non-null values observed. 


\input{glossary}



\section{Reducers}
\label{reducers}

A reducer is a construct that takes one or more vectors as input and produces
one or more scalars as output. The simplest Reducers are those for min, max, sum
and other associative functions. In this case, the reducer returns 2 values,one
being the value of the function being computed and the other being the number of
non-null values observed. 


\input{glossary}
