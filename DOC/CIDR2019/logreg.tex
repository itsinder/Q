\subsection{Logistic Regression}

A key subroutine in logistic regression \cite{Hastie2009} is improving the 
estimates of the
coefficients, \(\beta\), using the Newton-Raphson algorithm.
Our particular implementation requires computation of 
\(\frac{1}{1 + e^{-x}}\) and 
\(\frac{1}{(1 + e^{-x})^2}\), where \(x\) is \TBC 
We now discuss a sequence of increasingly sophisticated way of implementing
these operators in Q.

\subsubsection{The Lua implementation}
The ``quick and dirty'' way is to implement it in Lua. An example is 
in Figure~\ref{logit_in_lua}.
\begin{figure}
\centering
\fbox{
\begin{minipage}{14 cm}
\centering
\verbatiminput{logit.lua}
\caption{Using LuaJIT FFI to map C data to Lua}
\label{logit_in_lua}
\end{minipage}
}
\end{figure}

\subsubsection{The C implementation}
Figure~\ref{logit_in_lua} indicates that precisely one line is specific to the
operator in question, the rest is necessary boiler plate. Therefore, the design
of the run time has been around providing machinery to move data and leaving
the actual operator as a template that can be provided by the user. All the operator
writer needs to (1) pick a template --- in this case it is {\tt v1s1opv2}, indicating that a
vector v1 and an optional scalar s1 are used to produce another vector v2.
(2) provide the code fragment {\tt  v2 = 1.0 / ( 1.0 + exp(-1.0 * v1)); )}.

\subsubsection{Eliminating common sub-expressions}
However, one quickly recognizes that creating separate operators for those 2
functions would lead to 
redundant computations, which can be eliminated as shown below.
\begin{verbatim}
t1 = Q.vsmul(x, Scalar.new(-1, fldtype)):memo(false)
t2 = Q.exp(t1):memo(false)
t3 = Q.incr(t2):memo(false)
t4 = Q.sqr(t3):memo(false)
y  = Q.reciprocal(t3)
z  = Q.reciprocal(t4)
\end{verbatim}

\subsubsection{Lock step evaluation}
However, notice that using the above code introduces a subtle but
critical bug when the number of elements in \(x\) exceeds the chunk
size.  If we were to call {\tt eval()} on \(y\), then we would
end up consuming sucessive chunks of \(t_3\). Now, if we were to call
{\tt eval()} on \(z\), we would fail when requesting the first
chunk of \(t_3\), since it has {\bf not} been memo-ized. One solution
is to ensure that \(y\) and \(z\) are evaluated in lock-step, after they have
been created, as shown below.
\begin{verbatim} 
  cidx = 0  -- chunk index
  repeat 
    ly = y:chunk(cidx)
    lz = z:chunk(cidx)
    cidx = cidx + 1
  until ( ly == 0 )
\end{verbatim}

The problem with this solution is that the burden of lock-step evaluation falls
on the Q programmer. This is remedied by the {\tt conjoin} function.

\begin{verbatim}
y:set_memo(false); z:set_memo(false);  -- no memo-izing needed
conjoin({y, z}) -- indicate that they need to be evaluated together
assert(z:is_eov() == false) -- z not fully evaluated
y:eval() -- evaluate y when needed
assert(z:is_eov() == true) -- z fully evaluated as a consequence of y eval
\end{verbatim}

Need performance numbers \TBC
\be
\item Lua versus C
\item separate operators versus eliminating common sub expr
\ee

\subsubsection{Custom Operators}

As part of the \(\beta\) calculation step, we need to perform \(A \leftarrow x^T
W x\) where (1) 
\(x\) is a \(m \times m\) matrix and (2) \(W\) is a \(n \times n\) matrix where
non-diagonal elements are 0 and can be represented as a vector, \(w\) of legnth
\(n\).
The straight-forward Q code is as follows
\begin{verbatim}
for i, X_i in ipairs(X) do
  for j, X_j in ipairs(X) do
    A[i][j] = Q.sum( Q.vvmul(X_i, Q.vvmul(w, X_j))):eval()
  end
end
\end{verbatim}

However, with a little more effort, we created custom operator, {\tt
Q.sumprod2}, a simplified sketch of which is provided below. The actual code
uses blocking to avoid writing out to the entire temp array before reading it
back in.
For \(n = XX, m =
YY\), we get a performance gain of ZZ \TBC What we hope to illustrate is that a
simple data model allows us to move along the performance versus simplicity
continuum as desired.
\begin{verbatim}
_sum(float *x, int n) {
  float sum = 0;
  for ( i = 0; i < n; i++ ) { 
    sum += x[i];
  }
  return sum;
}
_vvmul(float *x, float *y, float *z, int n) {
  for ( i = 0; i < n; i++ ) { 
    z[i] = x[i] * w[i];
  }
}
for ( i = 0; i < m; i++ ) { 
  _vvmul(X[i], w, temp, n);
  for ( j = i; j < m; j++ ) { 
    _vvmul(X[j], temp, temp);
    A[i][j] = _sum(temp, n);
  }
}
\end{verbatim}

