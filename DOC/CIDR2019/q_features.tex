\section{Power to the People}

It is well accepted that large data systems benefit significantly 
from careful optimization of data movement. Optimizing compilers and query plan
rewriters aim to do this automatically.
Q takes a fundamentally different approach. It is our belief that the
choreography of computations can be left to the database programmer {\bf if} 
they have (i) some understanding of the underyling
system architecture and (ii) relatively simple knobs to influence the run time
system.  An example is provided in Section~\ref{reduce_operator}

In many analytical tasks, one repeatedly performs very similar operations on slowly
changing data. In these cases, 
it is not onerous to maintain (and periodically refresh) such statistics. 
Note that the fidelity demanded of these summary statistics is often low ---
they need to be only as good as the use to which they are put. For example, when
used for load balancing, they need to be good enough to guard against excessive
skew.
An example is discussed in Section~\ref{social_graph}

\subsection{The reduce operator}
\label{reduce_operator}
Consider the case where we want to perform many different reductions (e.g. min,
max, sum, \ldots) over the same vector. The simple way to do this is
\begin{verbatim}
x = Q.sum(w); y = Q.min(w); z = Q.max(w)
\end{verbatim}
However, this necessitates several scans over the vector \(w\). It is more
efficient to evaluate the vector \(w\) a chunk at a time, perform all the
reductions on the chunk, store the partial results and then repeat over
successive chunks. So, we can re-write the above as 
{\tt x, y, z = Q.reduce({ "sum", "min", "max" }, w)}
where {\tt Q.reduce} is described in Figure~\ref{fig_reduce}

\begin{figure}
\centering
\fbox{
\begin{minipage}{14 cm}
\centering
\verbatiminput{reduce.lua}
\caption{The ``reduce'' operator}
\label{fig_reduce}
\end{minipage}
}
\end{figure}
We try to avoid memo-izing vectors when we don't have to because 
of the cost of flushing to disk. In our experiments, for \(n > 2^{25}\), 
memo-izing the input \(w\) and computing min/max/sum sequentially 
is {\ul six} times slower than using the reduce operator.


\newcommand{\Q}{{\tt Q} }
\subsection{Exploiting Slowly Changing Summary Statistics}

In many analytical tasks, one repeatedly performs very similar operations on slowly
changing data. Examples are time series analysis to project the next 30 days
based on the last 90 days, anomaly detection, trend detection, \ldots In such
cases, summary statistics of the underlying data change slowly. Examples of such
statistics are average length of a web session, distribution of page views
across different parts of a website.  
In such cases, it is not onerous to
maintain (and periodically refresh) such statistics. 
Similarly, 
in the process of building and tuning machine, the practitioner ends up
developing tremendous intuition about the nature of the data they work with.
\Q\ encourages 
the programmer to get performance gains by exploiting the intuitions and
information gained.

Consider a simple parallelization of the sort routine. Assume that we 
``magically'' know a set of mutually exclusive and exhaustive intervals such
that each interval would get ``roughly'' the same number of elements of an input
Vector \(x\). Then a simple parallel sort consists of 
(1) partitioning \(n\) elements of \(x\) into \(n_B\) bins so that each bin has
\(n(b)\) elements
(2) sorting each bin in parallel
(3) copying each bin into the right place. Probabilistic gaurantees
of the form \(P[n(b) \geq (1+\beta) \times(n/n_B)] \leq \epsilon\) 
allow us to allocate slightly more space than absoutely necessary and to 
fall back to the sequential sort in the small probability that our estimates are off.

To support this, the {\tt Q.sort} operator accepts ``hints'' in the form of
optional arguments. In this case, the hint is a Vector whose length is the
number of bins and each element is the upper bound of that bin. We report
performance numbers in Table~\ref{XXX}

\begin{table}
\centering
\begin{tabular}{|l|l|l|l|l|} \hline \hline
  {\bf n} & {\bf Speedup} \\ \hline \hline
  \(2^{27}\) & X \\ \hline
  \(2^{28}\) & X \\ \hline
  \(2^{29}\) & X \\ \hline
  \(2^{30}\) & X \\ \hline
  \(2^{31}\) & X \\ \hline
  \hline
\end{tabular}
\caption{Speedup obtained with parallel sort}
\label{tbl_sort_speedup}
\end{table}

\subsection{Exploiting meta data}

To quote George Santayana, ``those who don't remember the past are condemned to
repeat it''.  \Q\ minimizes re-work in several ways
\be
\item 
Remembering basic statistics such as min, max, sum, average, whenever the
corresponding operators are invoked. 
\item the sorted-ness of a vector is
stored as one of (a) unknown (b) ascending (c) descending (d) not sorted. \Q\
uses sort heavily to simplify other operators by converting them into linear
scans. For example, consider {\tt x, y  = Q.count(z)} where \(x\) contains the
unique values of \(z\) and \(y\) the number of occurrences. If \(z\) is not
sorted, then the count operator internally creates a copy of \(z\) which is
sorted, passed to the count sub-routine and then deleted. However, if \(z\) is
sorted, this overhead is eliminated.

\ee

