\subsection{The reduce operator}
\label{reduce_operator}
Consider the case where we want to perform many different reductions (e.g. min,
max, sum, \ldots) over the same vector. The simple way to do this is
\begin{verbatim}
x = Q.sum(w); y = Q.min(w); z = Q.max(w)
\end{verbatim}
However, this necessitates several scans over the vector \(w\). It is more
efficient to evaluate the vector \(w\) a chunk at a time, perform all the
reductions on the chunk, store the partial results and then repeat over
successive chunks. So, we can re-write the above as 
{\tt x, y, z = Q.reduce({ "sum", "min", "max" }, w)}
where {\tt Q.reduce} is described in Figure~\ref{fig_reduce}

\begin{figure}
\centering
\fbox{
\begin{minipage}{14 cm}
\centering
\verbatiminput{reduce.lua}
\caption{The ``reduce'' operator}
\label{fig_reduce}
\end{minipage}
}
\end{figure}
We try to avoid memo-izing vectors when we don't have to because 
of the cost of flushing to disk. In our experiments, for \(n > 2^{25}\), 
memo-izing the input \(w\) and computing min/max/sum sequentially 
is {\ul six} times slower than using the reduce operator.
