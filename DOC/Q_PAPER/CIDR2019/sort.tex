\subsection{The social graph}
\label{social_graph}

One of the problems we investigated at LinkedIn was using the social graph to
guide and power the design of data-driven products.
Assume that a social graph is presented as a 
a table, {\tt E}, with 2 columns, {\tt from} and {\tt to}. Q supports the
creation of auxiliary data structures to enable fast access but, in keeping with
our minimalist philosophy, does not prvoide them out of the box.
\be
\item {\tt f, t = Q.sort2(E.from, E.to, "ascending")}, sorts {\tt E}
with {\tt from} as primary key and {\tt to} as
secondary key (Section~\ref{par_sort})
\item {\tt V.m = Q.unique(f)} creates a Vector of 
member IDs, sorted ascending. 
\item For each member in {\tt V.m}, we determine the contiguous edges out of it
  by 
  \be
\item {\tt V.lb = Q.join(f, V.m, min\_index)}
\item {\tt V.ub = Q.join(f, V.m, max\_index)}
  \ee
Now, we can assert that member
{\tt m = V.m[i]} is connected to members {\tt t[V.lb[i] .. V.ub[i]]}
\ee


\subsubsection{Speeding up sort}
\label{par_sort}
Consider a simple parallelization of the sort routine. Assume that we 
were given a set of bins such
that each bin would get ``roughly'' the same number of elements of an input
Vector \(x\). Then a simple parallel sort consists of 
(1) partitioning \(n\) elements of \(x\) into \(n_B\) bins so that each bin has
\(n(b)\) elements
(2) sorting each bin in parallel
(3) copying each bin into the right place. Probabilistic guarantees
of the form \(P[n(b) \geq (1+\beta) \times(n/n_B)] \leq \epsilon\) 
allow us to allocate slightly more space (specified by \(\beta\)) than absoutely necessary and to 
fall back to the sequential sort in the small probability (specified by
\(\epsilon\))  that our estimates are off.

But where do these bins come from? As in the social graph example above, where
new connections are being created and deleted all the time, the change in these
overall distribution is much, much slower. Each sort uses the previous estimates
and produces a new estimate, good enough for load balancing.
{\tt Q.sort} operator accepts ``hints'' in the form of
optional arguments. In this case, the hint is a Vector whose length is the
number of bins and each element is the upper bound of that bin. For Vectors of
length greater than \(2^{20}\), we obtained linear speedup on an Intel Core i7.

To quote George Santayana, ``those who don't remember the past are condemned to
repeat it''.  Q minimizes re-work in several ways
\be
\item 
Remembering basic statistics such as min, max, sum, average, whenever the
corresponding operators are invoked. 
Such meta-data is often the by product of other operators.
\item the sorted-ness of a vector is
stored as one of (a) unknown (b) ascending (c) descending (d) not sorted. 
Q
uses sort heavily to simplify other operators by converting them into linear
scans. 
%% explain difference between unknown and not sorted
For example, consider {\tt x, y  = Q.count(z)} where \(x\) contains the
unique values of \(z\) and \(y\) the number of occurrences. If \(z\) is not
sorted, then the count operator internally (a) creates a copy, \(z'\), of \(z\)
(b) sorts \(z'\) (c) passes \(z'\) to the core {\tt count} code and (d) \(z'\) gets
garbage collected when the count operator returns. Of course, if \(z\) is
sorted, then this cost of sorting \(z'\) is eliminated.

\ee
