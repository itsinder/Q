\subsection{The reduce operator}
\label{reduce_operator}
Consider the case where we want to perform many different reductions (e.g. min,
max, sum, \ldots) over the same vector. The simple way to do this is
\begin{verbatim}
x = Q.sum(w); y = Q.min(w); z = Q.max(w)
\end{verbatim}
However, this necessitates several scans over the vector \(w\). It is more
efficient to evaluate the vector \(w\) a chunk at a time, perform all the
reductions on the chunk, store the partial results and then repeat over
successive chunks. The reduce operator takes a Vector as input and produces one
or more Scalars as output. It does so, by 
(i) creating a Reducer for each of the desired Scalars,
(ii) repeatedly invokes {\tt next()} on them until no more data
(iii) invokes {\tt value()} on each Reducer and returns that.

We try to avoid memo-izing vectors when we don't have to because 
of the cost of flushing to disk. In our experiments, for \(n > 2^{25}\), 
memo-izing the input \(w\) and computing min/max/sum sequentially 
is {\em six} times slower than using the reduce operator.
