\begin{abstract}
The unspoken secret of ``big data'' is that while data warehouses are
exploding in size, the actual amount of data needed for many analytics and
machine learning problems is better characterized as ``medium data''. We define
``medium data'' as data large enough that one must be cognizant of performance but small
enough that a single machine is adequate for interactive usage. This realization has motivated the
design and development of Q, an analytical database. It has its own language
with an extensible set of operators.  The language is inspired by
Ousterhout's dichotomy, using a scripting language (Lua) with a fast interpreter
(LuaJIT) as the front-end and a compiled language (C) for the back-end.
A fundamental realization that has influenced Q's design is that data
practitioners often build up tremendous intuition about the data sets that they
are working with. Q encourages the use of this intuition to gain performance
at relatively low complexity.
Q exposes its software architecture in a disciplined and layered fashion. By
this we mean that there is a continuum of optimizations that become possible as
one is willing to engage with more of the details of the system. 
The other central principle guiding Q's development was that development and
maintenance of code, (especially when it involves trial and error, changing
requirements, \ldots) is as important as run time performance. 
This mandated providing a programming system and a debugging environment, 
with enough interactive performance to make coding and debugging a pleasure.

\end{abstract}
