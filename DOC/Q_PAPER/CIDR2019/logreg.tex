\subsection{Logistic Regression}

A key subroutine in logistic regression \cite{Hastie2009} is improving the 
estimates of the
coefficients, \(\beta\), using the Newton-Raphson algorithm.
In the course of this, we need to compute
\(\frac{1}{1 + e^{-x}}\) and 
\(\frac{1}{(1 + e^{-x})^2}\).
We now discuss a sequence of increasingly sophisticated ways of implementing
these operators in Q.

\subsubsection{The Lua implementation}
\label{Logit_Lua}
The ``quick and dirty'' way is to implement it in Lua. To do so, we 
(i) use the Vector's {\tt get\_chunk} method to access data, 
(ii) use LuaJIT 's {\tt ffi.cast} to directly access C data from Lua 
(iii) write the core operation in Lua as
\begin{verbatim}
for i=0,n1 do cd2[i]=1/(1+math.exp(-1*cd1[i])) end
\end{verbatim}

\subsubsection{The templatized implementation}
\label{Logit_template}
The design
of the run time has been around providing machinery to move data and leaving
the actual operator as a template that can be provided by the user. All the operator
% NOTE ACTUALLY this is f1s1opf2 not v1s1opv2
writer needs to (1) pick a template --- in this case it is {\tt v1s1opv2}, indicating that a
vector v1 and an optional scalar s1 are used to produce another vector v2
(2) provide the code fragment {\tt  v2 = 1/(1+exp(-1*v1)); )} (3)
guard against improper usage e.g, we wouldn't apply this function to a string.

In our experiments, 
JIT compilation makes the Lua code 25\% faster than single-threaded C code,
although 3 times slower than the multi-threaded C code which shows almost linear
speedup.

\subsubsection{Eliminating common sub-expressions}
However, one quickly recognizes that creating separate operators for those 2
functions would lead to 
redundant computations, which can be eliminated as in Figure~\ref{sub_expr}.
This cuts the time by half.

\begin{figure}
\centering
\fbox{
\begin{minipage}{8 cm}
\centering
\verbatiminput{sub_expr.lua}
\caption{Common sub-expression elimination}
\label{sub_expr}
\end{minipage}
}
\end{figure}


\subsubsection{Lock step evaluation}
However, the code in Figure~\ref{sub_expr} has a subtle but
critical bug when the number of elements in \(x\) exceeds the chunk
size.  If we were to call {\tt eval()} on \(y\), then we would
end up consuming sucessive chunks of \(t_3\). Now, if we were to call
{\tt eval()} on \(z\), we would fail when requesting the first
chunk of \(t_3\), since it has {\bf not} been memo-ized. One solution
is to ensure that \(y\) and \(z\) are evaluated in lock-step, after they have
been created, as in Figure~\ref{lock_step}.
\begin{figure}
\centering
\fbox{
\begin{minipage}{8 cm}
\centering
\verbatiminput{lock_step.lua}
\caption{Lock step evaluation}
\label{lock_step}
\end{minipage}
}
\end{figure}

The problem with this solution is that the burden of lock-step evaluation falls
on the Q programmer. This is remedied by the {\tt conjoin} function
(Figure~\ref{conjoin})

\begin{figure}
\centering
\fbox{
\begin{minipage}{8 cm}
\centering
\verbatiminput{conjoin.lua}
\caption{Conjoined Vectors}
\label{conjoin}
\end{minipage}
}
\end{figure}

%% TODO Normalizing the 
%% TODO \be
%% TODO \item Lua versus C (single  threaded)
%% TODO \item Lua versus C (multi threaded)
%% TODO \item separate operators versus eliminating common sub expr
%% TODO \ee

\subsubsection{Custom Operators}

As part of the \(\beta\) calculation step, we need to compute an \(m \times m\)
matrix \(A\) as  \(x^T W x\) where (1) 
\(x\) is a \(n \times m\) matrix and (2) \(W\) is a \(n \times n\) matrix where
non-diagonal elements are 0 and can be represented as a vector, \(w\) of legnth
\(n\). 
The straight-forward Q code is in Figure~\ref{compute_A_Lua_basic}
\begin{figure}
\centering
\fbox{
\begin{minipage}{8 cm}
\centering
\verbatiminput{compute_A_Lua_basic.lua}
\caption{Q code for \(x^T W x\)}
\label{compute_A_Lua_basic}
\end{minipage}
}
\end{figure}

However, with a little more effort, we created custom operator in C.
The custom code uses
chunking to avoid writing out to the entire temp array before reading it
back in. Setting \(m=32\), we varied \(n\) from \(2^{16}\) to \(2^{24}\). With
chunking, performance increases as we increase \(n\) but saturates after
\(2^{20}\). Without chunking, performance peaks at \(n=2^{18}\) and steadily
declines after that, to about a third of peak performance.
What we hope to illustrate is that a
simple data model allows us to move along the performance versus simplicity
continuum as desired.


This in turn raised the question: could we have ``chunked'' the computation 
without resorting to a custom operator? To do so, (1)
we first set up the reducers --- no actual
computation happens. This is exactly as in Figure~\ref{compute_A_Lua_basic}, except that there is no call to
{\tt eval()}. (2) We evaluate the reducer in chunks, stopping
when the input vectors have no more data as in
Figure~\ref{compute_A_Lua_chunked}.
\begin{figure}
\centering
\fbox{
\begin{minipage}{8 cm}
\centering
\verbatiminput{compute_A_Lua_chunked.lua}
\caption{Chunked Q code for \(x^T W x\)}
\label{compute_A_Lua_chunked}
\end{minipage}
}
\end{figure}

%% \begin{table}
%%   \centering
%%   \begin{tabular}{|l|l|l|} \hline \hline
%%     {\bf Chunking used?}    & {\bf Q} & {\bf Custom Operator} \\ \hline \hline
%%     {\bf Yes} & TODO & TODO \\ \hline
%%     {\bf No} & TODO & TODO \\ \hline
%%     \hline
%%   \end{tabular}
%%   \caption{Rlative times for different computational strategies}
%% \end{table}
