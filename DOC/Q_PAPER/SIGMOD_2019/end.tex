\section{Conclusion}
\label{end}

The development of Q has been motivated by (i) 
Stonebraker's ``one size does not fit all'' dictum and (ii) the
understanding, as data science practitioners, of what is truly needed to be
effective. We posit that engaging the user, rather than shielding them from
system internals, makes for simple, performant code. 
Following Fred Brooks' ``form is liberating'', the single machine,
single-threaded nature of Q's run time still poses many interesting research
problems. 

Lastly, and most provocatively, we wonder whether one day we will look at
systems like Hadoop the way we look at plastic bags --- seductively simple,
largely unnecessary and, in the final analysis, ecologically destructive.

%% Lastly, and most provocatively, we wonder whether the database community, as
%% some of the largest users of data centers, has a social responsibility 
%% to factor in environmental cost into system design.
