\subsection{Reducers}
\label{reducers}

A Reducer is a construct that takes one or more Vectors as input and produces
one or more Scalars as output. In that sense, this is similar to ``Builders ---
declarative objects for producing result e.g. compute a sum'' \cite{Weld2017}.

The simplest Reducers are those for {\tt min},
{\tt max}, {\tt sum} and other associative functions. 
In this case, the Reducer takes 1 Vector as input and 
returns 2 Scalars, one
being the value of the function being computed and the other being the number of
non-null values observed. 

An example of a more advanced Reducer is {\tt minK}. 
Given 2 Vectors, \(d\) and \(g\), minK returns 2
Lua tables of Scalars, \(d'\) and \(g'\) such that \(d'\) contains the \(k\)
smallest values of \(d\) and \(g'\) contains the corresponding values of \(g\).
See Section~\ref{knn} for a use case.

A few notes about Reducers: (a) 
The value of a Reducer can be queried even before it has been evaluated 
completely, a facility that has proven useful while debugging
(Section~\ref{knn})
(b) Given that Reducers can return a table of Scalars,
it is possible to 
convert a Lua table of scalars, \(s1, s2, \ldots\) into a vector by
\(Q.\mathrm{pack}(\{s1, s2, \ldots\})\).
Equivalently, \(S = Q.\mathrm{unpack}(x)\) creates a Lua table of Scalars from a
Vector \(x\).

