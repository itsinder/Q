\section{Power to the People}

It is well accepted that large data systems benefit significantly 
from careful optimization of data movement. Optimizing compilers and query plan
rewriters aim to do this automatically.
\Q\ takes a fundamentally different approach. It is our belief that the
choreography of computations can be left to the database programmer {\bf if} 
they have 
(i) some understanding of the underyling system architecture and (ii) 
relatively simple knobs to influence the run time system.  

In many analytical tasks, one repeatedly performs very similar operations on slowly
changing data. In these cases, 
it is not onerous to maintain (and periodically refresh) summary statistics that
can significantly speed up more complex operations. For example, \Q's sort
routine gets linear speed up on up to 8 cores, when provided with
a rough distribution of the keys. 
Note that the fidelity demanded of these summary statistics is often low ---
they need to be only as good as the use to which they are put. 

\subsection{Building indexes --- the social graph}
\label{social_graph}

At LinkedIn, we used \Q\ to analyze the social graph, given its importance in
extracting insights.
Assume that a social graph is presented as a 
a table, {\tt E}, with 2 columns, {\tt from} and {\tt to}. \Q\ supports the
creation of auxiliary data structures to enable fast access but, in keeping with
our minimalist philosophy, does not provide them out of the box. We describe the
pre-processing performed to provide efficient access.
\be
\item {\tt f, t = Q.sort2(E.from, E.to, "ascending")}, sorts {\tt E}
with {\tt from} as primary key and {\tt to} as
secondary key
\item {\tt V.m = Q.unique(f)} creates a Vector of 
member IDs, sorted ascending. 
\item For each member in {\tt V.m}, we determine the contiguous edges out of it
  by 
  \be
\item {\tt V.lb = Q.join(f, V.m, min\_index)}
\item {\tt V.ub = Q.join(f, V.m, max\_index)}
  \ee
Now, we can assert that member
{\tt m = V.m[i]} is connected to members {\tt t[V.lb[i] .. V.ub[i]]}
\ee

