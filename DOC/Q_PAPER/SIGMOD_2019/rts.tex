\section{Q's Run Time System}

Q is a high-performance ``almost-relational'' 
analytical, single-node, column-store database. 
\be
\item 
By ``analytical'' we mean that data changes at the user's behest (e.g.
loading a data set) but is not subject to external events.
\item 
By ``almost-relational'' we mean that it would more correctly
be called a ``tabular model'' \cite{Codd1982}. As Codd states, ``Tables are
at a lower level of abstraction than relations, since they give
the impression that positional (array-type) addressing is applicable
(which is not true of \(n\)-ary relations), and they fail to
show that the information content of a table is independent
of row order. Nevertheless, even with these minor flaws,
tables are the most important conceptual representation of
relations, because they are universally understood.''
\item By ``single-node'', we mean that Q does not distribute computation across
  machines. The flow of execution is inherently single-threaded. However,
  OpenMP is heavily used {\em within} individual operations so as
  to exploit multi-core systems as well as GPUs.
\item By ``column-store'', we mean that 
Q provides the Lua programmer with the Vector type, each
individual element of which is a Scalar. A table in Q is a Lua
table of Vectors (Section~\ref{vectors_versus_tables}), where a Lua table is an
associative array, like a Python dictionary.

\ee

\subsection{Q as a Lua extension}


It is useful to think of Q as a domain specific language, targeted for data
manipulation. In contrast with Wevers' work \cite{Wevers2014} on
developing a persistent functional language with
a relational database system inside, Q works within the context of Lua, while
inspired by functional programming ideas like memo-ization, lazy evaluation,
etc.
More accurately, Q is a C library, embedded within an interpreted
language.
We chose to embed within an existing language because (i) we did not have to
write a custom compiler (ii) it allowed us to leverage a rich eco-system of
libraries, debuggers, web development environments, allowing the programmer to
blur the distinction between application logic and database programming.

We chose Lua
because it was designed specifically as both an embedding and embedded language
\cite{Lua2011} and it had a wickedly fast interpreter, LuaJIT. The efficiency of
this combination is about 90\% with the overhead being split 90:10 between the
run time system and the Lua glue logic. Among other things, the glue logic
supports operator serves as the bridge between the dynamic typing that Q
supports and the static typing required by the C code underneath.

Data is stored in memory and, when necessary, is persisted (uncompressed) in
binary format to the file system. This allows us to quickly access data by
mmapp-ing the file.  Like kdb+ \cite{Borror2015}, one can think of Q as 
an in-memory database with persistent backing to the file system.

Q's software architecture consists of the following four layers. From bottom to
top with decreasing complexity, they are:
\be
\item The core run time system --- Vectors, Scalars, Reducers
\item operators --- provides core functionality (using C and OpenMP) on
  top of run time
\item Q library developers --- use Q and Lua to create higher level functions on
  top of core operators
\item Q programmers --- use Q much the way  Python programmer uses scikit-learn.
  \ee

\section{Vectors}
\label{vectors}

A vector is essentially a map \(f(i)\) such that given \(i, 0 \leq i < n\), it
returns a value of a given type. The main types that \Q\ supports four variants
of integers (1, 2, 4, and 8 byte) and two variations of floating point (single
and double precision). In addition, it supports bit vectors, constant length
strings. There is limited support for variable length strings, which are used
primarily as dictionaries.

\input{scalars}

\section{Reducers}
\label{reducers}

A reducer is a construct that takes one or more vectors as input and produces
one or more scalars as output. The simplest Reducers are those for min, max, sum
and other associative functions. In this case, the reducer returns 2 values,one
being the value of the function being computed and the other being the number of
non-null values observed. 


\input{glossary}



\section{Reducers}
\label{reducers}

A reducer is a construct that takes one or more vectors as input and produces
one or more scalars as output. The simplest Reducers are those for min, max, sum
and other associative functions. In this case, the reducer returns 2 values,one
being the value of the function being computed and the other being the number of
non-null values observed. 

