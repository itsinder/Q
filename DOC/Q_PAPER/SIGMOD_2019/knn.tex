\subsection{k-Nearest Neighbors}
\label{knn}

A key subroutine of the {\tt knn} algorithm \cite{Hastie2009} is finding the class labels of the
\(k\) points closest to the point \(x\) being classified. 
We assume that we have \(n\) points in \(m\)-dimensional space, represented as a
Lua table, \(T\), of \(m\) Vectors of length \(n\)

Note that the operations are performed in column
order rather than the more intuitive row order. Row ordering means computing the
distance from \(x\) to a point in \(T\) completely before proceeding to the next
point. We invite the reader to judge from Figure~\ref{knn_core} the merits of our claim that the programming burden of being aware of data layout and
operation ordering is not onerous.

Given Q's lazy evaluation in chunks, note that the operation is actually
performed in batches. In other words, had we written 
{\tt d1, g1 = Q.mink(d, g, k)} and then done {\tt g1:next()} instead  of {\tt
g1:eval()}, 
we would have gotten the class labels of the \(k\) points
closest to \(x\) from the {\bf first} \(n_C\) points of \(T\). 

\begin{figure}[hbtp]
\centering
\fbox{
\begin{minipage}{8 cm}
\centering
\verbatiminput{knn.lua}
%% TODO Change mink_reducer to mink in final paper
\caption{core of k-nn algorithm in Lua}
\label{knn_core}
\end{minipage}
}
\end{figure}


