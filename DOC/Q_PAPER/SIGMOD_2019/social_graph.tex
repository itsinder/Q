\subsection{The social graph}
\label{social_graph}

One of the problems we investigated at LinkedIn was using the social graph to
guide and power the design of data-driven products.
Assume that a social graph is presented as a 
a table, {\tt E}, with 2 columns, {\tt from} and {\tt to}. Q supports the
creation of auxiliary data structures to enable fast access but, in keeping with
our minimalist philosophy, does not prvoide them out of the box.
\be
\item {\tt f, t = Q.sort2(E.from, E.to, "ascending")}, sorts {\tt E}
with {\tt from} as primary key and {\tt to} as
secondary key (Section~\ref{par_sort})
\item {\tt V.m = Q.unique(f)} creates a Vector of 
member IDs, sorted ascending. 
\item For each member in {\tt V.m}, we determine the contiguous edges out of it
  by 
  \be
\item {\tt V.lb = Q.join(f, V.m, min\_index)}
\item {\tt V.ub = Q.join(f, V.m, max\_index)}
  \ee
Now, we can assert that member
{\tt m = V.m[i]} is connected to members {\tt t[V.lb[i] .. V.ub[i]]}
\ee

Q minimizes re-work in several ways
\be
\item 
Remembering basic statistics such as min, max, sum, average, whenever the
corresponding operators are invoked. 
Such meta-data is often the by product of other operators.
\item the sorted-ness of a vector is
stored as one of (a) unknown (b) ascending (c) descending (d) not sorted. 
Q
uses sort heavily to simplify other operators by converting them into linear
scans. 
%% explain difference between unknown and not sorted
For example, consider {\tt x, y  = Q.count(z)} where \(x\) contains the
unique values of \(z\) and \(y\) the number of occurrences. If \(z\) is not
sorted, then the count operator internally (a) creates a copy, \(z'\), of \(z\)
(b) sorts \(z'\) (c) passes \(z'\) to the core {\tt count} code and (d) \(z'\) gets
garbage collected when the count operator returns. Of course, if \(z\) is
sorted, then this cost of sorting \(z'\) is eliminated.

\ee
