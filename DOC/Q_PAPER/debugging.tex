\section{Debugging}
\label{Debugging}

Our experience has been that ease of debugging has often been made
subservient to bragging rights on performance. The ability to debug code
interactively was one of the central goals of the \Q\ project. In this section,
we describe common use cases we found ourselves facing and how these are
addressed.

\be
\item Using an Integrated Development Environment.
Since \Q\ is implemented as a Lua package, the CLION IDE from IntelliJ 
with the Lua plugin allows interactive debugging sessions. 
\item Sharing sessions. When working in teams, it is often useful to invite
colleagues to ``look over one's shoulder'' and even make changes
directly. To do so, we have embedded Q in an HTTP server and have provided a command
line interface (CLI). When the CLI starts up, one provides the server and port
to which it should connect. As long as the appropriate ports are open, two
developers could issue commands to the server, the commands being serialized at the
server. 
\item Save and Restore. Data scientists often need to multiplex between projects
and most development projects are not small enough to be solved in a single
session. The functions {\tt save} and {\tt restore} are handy in this case. {\tt
Save}
causes the evaluation of any partial vectors and writes out meta data about
these to a file. {\tt Restore} reads from that file and brings back to life the
vectors and any meta-data associated with them.

\item Printing variables. Most debuggers allow one to print the value of a
variable. ``print'' is less useful for vectors 
which could easily have a billion values. What one typically needs to do
while debugging is to query some {\it property} of the vector. For example, is it
sorted, how many unique values does it have, what is its distribution,
\ldots These are enabled in \Q\ by invoking {\tt save} from the IDE and
{\tt restore} from another session, where these operations can be performed and
then intermediate results (like count of unique values) discarded.

\item Modifying state. In addition to inspecting variables, it is often useful
  to modify them so as to continue the debugging session without having to
  start over. For example, say that we hit a breakpoint and discover  
  that we had forgotten to sort a vector or load a table. We 
  (i) {\tt Save} from the IDE
  (ii) {\tt Restore} in a parallel session 
  (iii) perform the modifications in that session 
  (iv) {\tt Save} and quit the session 
  (v) lastly, {\tt Restore} in the IDE.

\ee
