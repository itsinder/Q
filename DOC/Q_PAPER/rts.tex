\section{Q's Run Time System}

Q is a high-performance ``almost-relational'' 
analytical, single-node, column-store database. 
\be
\item 
By ``analytical'' we mean that data changes at the user's behest (e.g.
loading a data set) but is not subject to external events.
\item 
By ``almost-relational'' we mean that it would more correctly
be called a ``tabular model'' \cite{Codd1982}. As Codd states, ``Tables are
at a lower level of abstraction than relations, since they give
the impression that positional (array-type) addressing is applicable
(which is not true of \(n\)-ary relations), and they fail to
show that the information content of a table is independent
of row order. Nevertheless, even with these minor flaws,
tables are the most important conceptual representation of
relations, because they are universally understood.''
\item By ``single-node'', we mean that Q does not distribute computation across
  machines. The flow of execution is inherently single-threaded. However,
  OpenMP is heavily used {\em within} individual operations so as
  to exploit multi-core systems as well as GPUs.
\item By ``column-store'', we mean that 
Q provides the Lua programmer with the Vector type, each
individual element of which is a Scalar. A table in Q is a Lua
table of Vectors (Section~\ref{vectors_versus_tables}), where a Lua table is an
associative array, like a Python dictionary.

\ee

\subsection{Q as a Lua extension}


It is useful to think of Q as a domain specific language, targeted for data
manipulation. In contrast with Wever's work \cite{Wever2014} on
developing a persistent functional language with
a relational database system inside, Q works within the context of Lua, while
inspired by functional programming ideas like memo-ization, lazy evaluation,
etc.
More accurately, Q is a C library, embedded within an interpreted
language.
We chose to embed within an existing language because (i) we did not have to
write a custom compiler (ii) it allowed us to leverage a rich eco-system of
libraries, debuggers, web development environments, allowing the programmer to
blur the distinction between application logic and database programming.

We chose Lua
because it was designed specifically as both an embedding and embedded language
\cite{Lua2011} and it had a wickedly fast interpreter, LuaJIT.
We experimented with several approaches to invoking C from Lua. These included
(a) dyncall \cite{Adler2013} (b) the native Lua C API (c) LuaFFI (d) LuaJIT's
FFI. Taking the native Lua C API as the baseline, dyncall was (surprisingly) the worst (twice
as slow) and LuaJIT was (unsurprisingly) the fastest (50 times as fast!).

Data is stored in memory and, when necessary, is persisted (uncompressed) in
binary format to the file system. This allows us to quickly access data by
mmapp-ing the file.  Like kdb+ \cite{Borror2015}, one can think of Q as 
an in-memory database with persistent backing to the file system.

How efficient is the combination of Lua and C? 
On a sample workload, when the chunk size 
(discussed in Section~\ref{Vectors})
is set to 1M, the C code that does the real work
accounts for 92\% of the time, the run time takes 1\%, the balance attributable to Lua as glue logic 
When the chunk size drops to 64K, C takes 52\%, the run time takes 2\% and Lau
the rest.

Q's software architecture consists of the following four layers. From bottom to
top with decreasing complexity, they are:
\be
\item The core run time system --- Vectors, Scalars, Reducers
\item operators --- provides core functionality (using C and OpenMP) on
  top of run time
\item Q library developers --- use Q and Lua to create higher level functions on
  top of core operators
\item Q programmers --- use Q much the way  Python programmer uses scikit-learn.
  \ee

\subsection{Vectors}
\label{Vectors}
A vector is essentially a map \(f(i)\) such that given \(i, 0 \leq i < n\), it
returns a value of a given type. The main types that \Q\ supports are four variants
of integers (1, 2, 4, and 8 byte) and two variations of floating point (single
and double precision). In addition, it supports bit vectors and constant length
strings. There is limited support for variable length strings, which are used
primarily as dictionaries. 

Note that Q has 6 types of numbers, in contrast with Lua which uses a 
single type {\tt number}, internally a double precision floating point.
This is because data bandwidth plays a significant role in determining
performance, as illustrated by Nvidia's
introduction of half precision floating point  \cite{nvidia2017}. The user is
encouraged (but not required) to use the smallest type that supports the actual
dynamic range required. It is eye-opening to realize that this is often  less
than what one expects. For example, while analyzing LinkedIn's endorsement feature
(e.g. User A endorses User B's proficiency in Java),
the number of unique endorsements that captured more than 99\% of the traffic
was less than 32K, making a 2-byte integer adequate for the task at hand.

It is relatively easy to add other types to Q, as long as these 
are fixed width types. The real cost is in making sure that all relevant
operators interpret that data meaningfully. For example, currently, {\tt add}
accepts all combinations of number types, but we do not (yet)
support adding a bit to a double.

We
encourage ``inexact'' types where justified. For example, a common use case is
to trace a user's activity through a web session. This requires joining on a
user ID across different log tables in the data warehouse. The user ID is
often provided as a long alphanumeric string. While Q could represent this as a
constant length string, it is better represented as an unsigned 64-bit integer
containing the hash of the string, since join requires only equality comparisons.
The number of distinct keys being hashed is often less than \(2^{32}\). When
hashed to a \(2^{64}\) space, the probability of collisions is vanishingly
small. Most business analyses are insensitive to this level of imprecision.

When a vector is created, we need to specify (i) its type (ii) whether it has
null values (iii) how it will be populuate. We can either (a) ''push'' data to
it, much like writing to a file in append mode or (b) we can provide a generator function,
which is invoked with the chunk index, \(i\), as parameter and which knows how to
generate the \(i^{th}\) chunk.
%% TODO Have not introduced chunk as yet

Vectors are evaluated lazily. Hence, a statement like 
{\tt x = Q.const(\{len = 10, qtype = I4, val = 0\})} does not actually create
ten
4-byte integers with value 0 as one might suspect. Data is populated only when
{\tt eval()} is explicity invoked on the vector or the data is 
implicitly required by some other operator e.g. {\tt Q.print\_csv(x)}

Vectors are processed in chunks. Consider an expression  like \(\sum (a + b\times
c)\), written in Q as {\tt d = Q.sum(Q.add(Q.mul(b,c), a))}.
When {\tt d} is eval'd, computation alternates between the \(+\) operator
and the \(\times\) operator, processing chunks of data at a time until there are
no more.
We had originally chosen co-routines to effect the {\tt resume} and
{\tt yield} capabilities needed. However, Lua's support of closures allowed for
a simpler implementation.
The chunk size, \(n_C\), is chosen large enough that it is amenable to
vectorization and parallelization and small enough that its memory consumption
is low.

Vectors are not mutable (with few exceptions)
and must be produced sequentially. In other words, the \(i^{th}\) element must
be produced before the \({i+1}^{th}\). Vectors
operate in ``chunks'' of a fixed size. Let us say that the chunk size is 64K and
that we have produced 65K elements. In that case, the current chunk would have
only 1K elements. Whether one can get access to an element in the previous chunk
depends on whether the vector has been ``memo-ized''. The default behavior, with
a concomitant performance hit, is to memo-ize. However, when the programmer is
aware that the vector will be consumed in a streaming fashion, they set memo
to false. 

A vector is fully materialized in one of two ways (1) If it was setup with a a generator function and invocation of the function produces 
fewer elements than the chunk size (2) it was set up for streaming and {\tt
eov()} was explicitly signaled.

Memo-izing is done by appending previous chunks in binary format to a file.
Subsequent reads of this vector are done by mmap-ing the file. Not all
algorithms are readily transformed into streaming operations e.g. sort. There
are a few cases where we support modifying a vector after it has been fully
materialized by opening it in write mode and mmap-ing it.

Mmap-ing gives us the illusion of a linear address space. This is useful to
incoporate algorithms and libraries that have not been written with streaming in
mind. For example, we have used this to call functions in the GNU Scientific
Library and LAPACK from Q.

Q's run time is an alternate approach to ``stream fusion''
\cite{Mainland2017}. In
that paper, the authors identify this as a technique that allows a compiler to
`` cope with boxed numeric types, handle lazy evaluation, and
eliminate intermediate data structures''.
\subsection{Vectors versus Tables}
\label{vectors_versus_tables}

A deliberate choice in Q's design was that Vectors, not tables, were the basic
data type. It is a design choice that we were (and continue to be) conflicted with. On the one hand,
it led to simplicity and performance. On the other hand, it put the burden of
table semantics on the programmer. In particular, consider a 
Lua table \(\{f, g\}\) containing 2 Vectors \(f, g\). Maintaining table
semantics means that \(T(i) = (f(i), g(i))\)  i.e., the \(i^{th}\) element of
the table is the \(i^{th}\) element of \(f\) and the \(i^{th}\) element of \(g\). An 
operation like {\tt Q.sort(T.f)} would destroy that contract. Q
provides higher level functions (Figure~\ref{sort_tbl}) which are invoked when the entire table needs to
be sorted e.g. {\tt Q.sort(T, f, "ascending")} 

\begin{figure}[hbtp]
\centering
\fbox{
\begin{minipage}{14 cm}
\centering
\verbatiminput{sort_tbl.lua}
\caption{Re-ordering table based on sort of column f in ascending order}
\label{tbl_sort}
\end{minipage}
}
\end{figure}
%% TODO make sure permute is consistent with Terra code


\subsection{Reducers}
\label{reducers}

A Reducer is a construct that takes one or more Vectors as input and produces
one or more Scalars as output. The simplest Reducers are those for {\tt min},
{\tt max}, {\tt sum} and other associative functions. 
In this case, the Reducer takes 1 Vector as input and 
returns 2 Scalars, one
being the value of the function being computed and the other being the number of
non-null values observed. 

An example of a more advanced Reducer is {\tt minK}. 
Given 2 Vectors, \(d\) and \(g\), minK returns 2
Lua tables of Scalars, \(d'\) and \(g'\) such that \(d'\) contains the \(k\)
smallest values of \(d\) and \(g'\) contains the corresponding values of \(d\).
See Section~\ref{knn} for a use case.

A few notes about Reducers: (a) 
The value of a Reducer can be queried even before it has been evaluated 
completely, a facility that has proven useful while debugging. 
(b) Given that Reducers can return a table of Scalars,
it is possible to 
convert a Lua table of scalars, \(s1, s2, \ldots\) into a vector by
\(Q.\mathrm{pack}(\{s1, s2, \ldots\})\).
Equivalently, \(S = Q.\mathrm{unpack}(x)\) creates a Lua table of Scalars from a
Vector \(x\).



\subsection{Operator polymorphism in Q}
\label{polymorphism}

Vectorized operators in Q are intrinsically overloaded i.e. the same Q operator
(Lua function) can be invoked with vectors of any of the supported Q data-types,
Section~\ref{Vectors}.
Overloading is achieved as follows.
\be
\item static compilation: In this case, the operator writer implements
  templatized C-like code. At build time, the template is fleshed out for all
  relevant combinations of data types to generate the {\tt .c} and {\tt +.h}
  files, which are then compiled into a single {\tt .so}. The {\tt .h} files
  are passed to LuaJIT using the {\tt cdef} function to allow FFI.
\item dynamic compilation: 
  In this case too, we require the existence of templates. 
  The server starts up with a minimal set of core functionality. 
  However, the generation of the source file, compiling and loading
  are done on demand. The results are cached
  so that subsequent calls to the same operator do not incur this overhead.
  This approach is refered to as  ``multi-stage programming''
\item we define a function for a particular combination of data types
  dynamically, in a strongly typed (low-level) language, that can bind seamlessly
  with the host language (Lua, in our case). The low-level language should ideally support templating so we can define templatized functions that can be compiled on-demand within the host-language's runtime.
  We found that the Terra language seemed to satisfy our precise wishlist for
  the low-level language \cite{devito2015}.

Terra is ``a low-level system programming language that is embedded in and
meta-programmed by the Lua programming language'' 
Some of the key features of the Lua/Terra combination that we benefit 
from are:
\be
\item Lua/Terra interoperability does not require glue code since Terra type and function declarations are first-class Lua statements, providing bindings between the two languages automatically. Lua code can call Terra functions directly.

\item Terra functions include type annotations and are statically typed in the sense that types are checked at compile time, but Terra functions are compiled during the execution of the Lua program, which gives us the ability to meta-program their behavior.

\item Compilation of Terra itself occurs dynamically as the Lua program executes. Though Terra programs/functions are embedded inside Lua and share a lexical runtime, the two languages have compartmentalized runtimes. One way to think of this design is that the Terra compiler
is part of the Lua runtime.

\item Terra entities (e.g. types, functions, expressions, symbols) are all first class values in Lua.
They can be held in Lua variables and passed through Lua functions. In particular, there are Lua variables {\tt int, double} etc. representing the primitive Terra types.

\ee

  \ee


% \subsection{Glossary of Q operators}

Table~\ref{tbl_qops} lists Q operators referenced in this paper
\begin{table}
\centering
\begin{tabular}{|l|l|l|l|l|} \hline \hline
  {\bf Operator} & {\bf Explanation} \\ \hline \hline
  const & creates a constant vector \\ \hline
  vvadd & \(v_1 \leftarrow v_2 + v_3\) \\ \hline
  vssub & \(v_1 \leftarrow v_2 - s_1\) \\ \hline
  sqr & \(v_1 \leftarrow v_2^2\) \\ \hline
  mink & Section~\ref{reducers} \\ \hline
  \(n_C\) & size of chunk in which vectors get evaluated \\ \hline
  \hline
  \hline
\end{tabular}
\caption{Q operators (selected list)}
\label{tbl_qops}
\end{table}


