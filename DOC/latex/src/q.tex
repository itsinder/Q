\startreport{Q: A Developer's Guide}
\reportauthor{Ramesh Subramonian}

\section{Introduction}

The aim of this document is to provide a step-by-step guide to being a
library developeri, working on the internals of Q. We will also
discuss how packaging and installation are done and how to operate as
a Q developer. A Q developer need have {\bf no} idea about the
internals of Q --- it is just another Lua package. Of course,
understanding the internals, to some extent, allows for more efficient
usage of the Q primitives.

\section{Getting Started}
\label{getting_started}

\subsection{Environment Variables}
\label{env_var}
So, you want to modify the guts of Q? Here's a step by step guide.

\be
\item Say, you are in \verb+/home/subramon/WORK/+ 
\item \verb+git clone https://github.com/NerdWalletOSS/Q.git+
\item Set the following environment variables
using \verb+source setup.sh -f+. Note that this is just a convenience. If you
want, you can set them yourself but then the onus is on you to get things right.
These are
\be
\item \verb+QC_FLAGS+ --- these will be used as flags to gcc when creating
\verb+.o+ files
\item\verb+Q_LINK_FLAGS+ --- these will be used as flags to gcc when creating
\verb+.so+ files
\item \verb+Q_ROOT+ ---  This is where artifacts created by the build provess
will be stored. As of now, they are
\be
\item \verb+Q_ROOT/lib/+ --- contains \verb+libq_core.so+
\item \verb+Q_ROOT/include/+ --- contains \verb+q_core.h+
\item \verb+Q_ROOT/tmpl/+ --- contains templates, used for dynamic code
generation
\ee
\item \verb+Q_DATA_DIR+ --- This is where data files will be stored. Think of
this as a tablespace and keep a separate one for each project you are working
\item \verb+Q_METADATA_DIR+ --- This is where meta data files will be stored. Think of
this as a tablespace and keep a separate one for each project you are working
on. Default will be \verb+Q_ROOT/meta/+
\item 
\verb+LD_LIBRARY_PATH+ Make sure that this includes \verb+Q_ROOT/lib/+ This is
where \verb+libq_core.so+ will be created
\item \verb+ LUA_PATH+, Section~\ref{masquerade}
\ee
\ee

\subsubsection{Consequences}

There are some important consequences of the above. 
\be
\item 
{\bf Do  not set} these environment variables in any of your scripts.
\item {\bf Do not use} \verb+Q_ROOT+ anywhere except in \verb+Q/UTILS/build+
\ee


\subsection{Building}

C programs are used to augment Lua in two important ways
\be
\item to help with code generation and to perform some functionality that 
could not be done easily (or in a performant manner) in Lua. Examples
of these are text converters like \verb+txt_to_I8+ or \verb+get_cell+
\item the computational workhorse. This is where the heavy lifting happens.
\ee

You will note a bit of a circular dependency. We need C code to create
C code.  This is broken in one of two ways
\be
\item Execute \verb+Q/UTILS/mk_core_so.lua+ This creates the following 
files 
\be
\item \verb+Q_ROOT/include/q_core.h+ --- which is used for \verb+ffi.cdef()+
\item \verb+Q_ROOT/lib/q_core.so+ --- which is used for \verb+ffi.load()+
\ee
You are have the C functionality needed for code generation

\item Within \verb+Q/UTILS/build/+, do {\tt make clean; make} 
\ee

\subsection{Masquerading as a Q developer}
\label{masquerade}

From time to time, you will need to pretend to be a Q developer so that you can
test your code. To enable this to happen without re-installing Q, you set
\verb+LUA_PATH+ as below. Note the double semi-colon at the end. That is needed.
Srinath to fill in the gaps \TBC
\begin{verbatim}
\verb+/home/subramon/WORK/?.lua;;+
\end{verbatim}
