\startreport{Q: A Developer's Guide}
\reportauthor{Ramesh Subramonian}

\section{Introduction}

This document is written with two distinct audiences in mind.
\be
\item Library developer: The folks developing the Q library 
\item Q programmer aka user: The developers who'll use the Q library
\ee
Note these are just roles, the same person can play both.

The aim of this document is to provide a step-by-step guide to being a
library developer, working on the internals of Q. We will also
discuss how packaging and installation are done and how to operate as
a Q programmer. A Q programmer need have {\bf no} idea about the
internals of Q --- it is just another Lua package. Of course,
understanding the internals, to some extent, allows for more efficient
usage of the Q primitives.


\section{Getting Started}
\label{getting_started}

\subsection{Building your machine for the first time}

Starting with a minimal Ubuntu install, you should execute Indrajeet \TBC

Check that we are building LuaJIT with pthread in Make. \TBC



\subsection{Environment Variables}
\label{env_var}
So, you want to modify the guts of Q? Here's a step by step guide.

\be
\item Say, you are in \verb+/home/subramon/WORK/+ 
\item \verb+git clone https://github.com/NerdWalletOSS/Q.git+
\item Set the following environment variables
using \verb+source setup.sh -f+. Note that this is just a convenience. If you
want, you can set them yourself but then the onus is on you to get things right.
These are
\be
\item \verb+QC_FLAGS+ --- these will be used as flags to gcc when creating
\verb+.o+ files
\item\verb+Q_LINK_FLAGS+ --- these will be used as flags to gcc when creating
\verb+.so+ files
\item \verb+Q_ROOT+ ---  This is where artifacts created by the build provess
will be stored. As of now, they are
\be
\item \verb+Q_ROOT/lib/+ --- contains \verb+libq_core.so+
\item \verb+Q_ROOT/include/+ --- contains \verb+q_core.h+
\item \verb+Q_ROOT/tmpl/+ --- contains templates, used for dynamic code
generation
\ee
\item \verb+Q_DATA_DIR+ --- This is where data files will be stored. Think of
this as a tablespace and keep a separate one for each project you are working
\item \verb+Q_METADATA_DIR+ --- This is where meta data files will be stored. Think of
this as a tablespace and keep a separate one for each project you are working
on. Default will be \verb+Q_ROOT/meta/+
\item 
\verb+LD_LIBRARY_PATH+ Make sure that this includes \verb+Q_ROOT/lib/+ This is
where \verb+libq_core.so+ will be created
\item \verb+LUA_PATH+, Section~\ref{masquerade}
\ee
\ee

\subsubsection{Consequences}

There are some important consequences of the above. 
\be
\item {\bf Do  not set} above environment variables in any of your scripts.
\item {\bf Do not use} \verb+Q_ROOT+ anywhere except in \verb+Q/UTILS/build+
\item In your Lua scripts, you must specify the entire path of the file you want
to require e.g.,
\begin{center}
{\tt local foo = require 'Q/UTILS/lua/foo' }
\end{center}
\item There are a few ``requires'' that I would like to highlight.
\begin{verbatim}
local qconsts = require 'Q/UTILS/lua/q_consts'
local ffi     = require 'Q/UTILS/lua/q_ffi'
local qc      = require 'Q/UTILS/lua/q_core'
\end{verbatim}
\be
\item qconsts contains constants e.g., \verb+qconsts.qtypes+ describes the
various qtypes.
\item qc contains all C functions that are accessible to Lua. See  Section~\ref{two_kinds_c_functions}. Example, \verb+qc["txt_to_I8"]+
\item ffi is the same as on \url{http://luajit.org}, except that malloc has been
modified a bit.
\ee
\ee


\subsection{Building}

\subsubsection{Two kinds of C functions}
\label{two_kinds_c_functions}
C programs are used to augment Lua in two important ways
\be
\item to help with code generation and to perform some functionality that 
could not be done easily (or in a performant manner) in Lua. Examples
of these are text converters like \verb+txt_to_I8+ or \verb+get_cell+
\item the computational workhorse. This is where the heavy lifting happens.
\ee

\subsubsection{Breaking circular dependency}
You will note a bit of a circular dependency. We need C code to create
C code.  This is broken in one of two ways
\be
\item Execute \verb+Q/UTILS/mk_core_so.lua+ This creates the following 
files 
\be
\item \verb+Q_ROOT/include/q_core.h+ --- which is used for \verb+ffi.cdef()+
\item \verb+Q_ROOT/lib/q_core.so+ --- which is used for \verb+ffi.load()+
\ee
You are have the C functionality needed for code generation

\item Within \verb+Q/UTILS/build/+, do {\tt make clean; make} 
\ee

\subsection{Registering an operator}

When you create an operator, a functionality that will be used by the Q
programmer e.g, add, sort, \ldots, you need to {\em register} that in the
following manner. Consider \verb+mk_col+ as an example

\begin{verbatim}
local Q       = require 'Q/q_export'
local mk_col = function(input, qtype)
......
return require('Q/q_export').export('mk_col', mk_col)
\end{verbatim}

You need to know a few things to understand the underlying mechanism
\be
\item 
\verb+Q_SRC_ROOT/init.lua+ is called when Q programmer does 
\verb+require ('Q')+
\item \verb+ q_export+ is the table that Q programmer will use
\ee



\subsection{Masquerading as a Q programmer}
\label{masquerade}

From time to time, you will need to pretend to be a Q programmer so that you can
test your code. To enable this to happen without re-installing Q, you set
\verb+LUA_PATH+ as below. Note the double semi-colon at the end. That is needed.
Srinath to fill in the gaps \TBC
\begin{verbatim}
/home/subramon/WORK/?.lua;;+
/home/subramon/WORK/?/init.lua;;
\end{verbatim}
First one above is current, second is Srinath's. Resolve difference \TBC

\subsection{Installation}

At some point in the not too distant future, Q will be installed as 
\begin{verbatim}
sudo luarocks install Q
\end{verbatim}
Until then, it is installed as 
\begin{verbatim}
cd $Q_SRC_ROOT; sudo bash q_install.sh 
\end{verbatim}
%$
\subsection{Usage}

The Q programmer does not need to set any of the environment variables of
Section~\ref{env_var}, except for 
\verb+Q_DATA_DIR, Q_METADATA_DIR+, which {\em must} be set. 
In an earlier version, we had required them
to set \verb+LUA_INIT+. That is no longer needed.
A sample Q script looks as follows
\begin{verbatim}
Q = require 'Q'
x = Q.mk_col({10, 20, 30, 40}, 'I4')
Q.print_csv(x, nil, "")
os.exit()
\end{verbatim}

If you want to change the environment variables, 
\verb+Q_DATA_DIR, Q_METADATA_DIR+, this means that you want to work on a
different project or tablespace. In that case, 
\be
\item \verb+Q.save()+
\item exit the script 
\item reset the environment variables
\item restart the Q server or execute a new script
\ee

\section{Lua Coding Conventions}
This section deals with coding conventions to be followed by a library developer
writing Lua code. 
\be
\item Do {\bf not} pollute the global name space. So, all variables are {\bf
local}.
\item Do not use {\tt dofile}. Use {\tt require instead}
\item LuaJIT scripts that invoke pthreads (indirectly) should end with
\verb+os.exit()+
\ee

\section{LuaJIT}
 I am going to make a difficult decision, favoring immediate 
 expediency for long term portability.  Basically, decision is to tie 
 ourselves to LuaJIT. This means we do {\bf not}  use features of Lua that 
 come after 5.1.3. This is a moving target, depending on what Mike Pall is
 willing to support. 

 Andrew: I think the value terra brings alone is probably worth that. 

\subsection{Performance tips}
Following are all available on \url{luajit.org} and reproduced for convenience. 

\be
\item 
ffi.copy() may be used as a faster (inlinable) replacement
 for the C library functions memcpy(), strcpy() and strncpy().
\ee

\subsection{Converting numbers from Lua to C and vice versa}

A simple way to convert Lua Number to C data struture using FFI API is
\begin{verbatim}
local chunk = ffi.new(ctype .. "[?]", length, input)
\end{verbatim}
where
\be
\item 
\item chunk  ---  C data struture of type ctype
\item ctype  ---  datatype e.g. -  int, char, float etc.
\item input  ---  Lua table of Lua number
\item length ---  number of elements
\ee

Reference link \url{http://luajit.org/ext_ffi.html}

Extensions to the Lua Parser. 
The parser for Lua source code treats numeric literals with the suffixes LL
or ULL as signed or unsigned 64 bit integers. Case doesn't matter, but
uppercase is recommended for readability. It handles both decimal (42LL)
and hexadecimal (0x2aLL) literals.

%----------------------------------

\subsection{LuaJIT and OpenMP}

Ciprian Tomoiaga wrote:
{\it I have a C function containing OpenMP clause which I call with ffi. On my
machine, the program terminates with a segmentation fault. However, it
works without a problem on a Mac and on another machine. }

Mike Pall responded: 
The most likely cause is the dreaded pthread issue: the main
executable must be compiled/linked with -pthread (note: this is
not the same as -lpthread!). Otherwise a shared library which is
loaded later on and uses threads may cause a crash.
The easiest way to test this hypothesis is to rebuild LuaJIT with:
\begin{verbatim}
  make TARGET_FLAGS=-pthread
\end{verbatim}
and then try again.

Ciprian Tomoiaga wrote:
{\it I recompiled with the specified opition, but it still crashes. :(
Do I have any other options? }

Mike Pall responded: 
Well I tried myself and it prints two lines, but then crashes when
the shared library is unloaded (as a consequence of \verb+lua_close()+).
Looks like OpenMP doesn't like to be unloaded --- this never
happens with programs that are statically linked against it.

