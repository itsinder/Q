\usepackage{hyperref}
\startreport{PageFlow Funnel Analysis}
\reportauthor{Ramesh Subramonian}

\section{Introduction}

\TBC

\section{UI}
This is a one-page application, broken down as follows.
\subsection{Exploration}
\subsubsection{Inputs}
\be
\item Date for which data is to be analyzed
\item Action. This can be 
\be
\item Push --- increase sequence length. User must then specify which
page to add to the sequence
\item Pop --- reduce sequence length
\ee
\ee
\subsubsection{Outputs}
The user is provided feedback in terms of Table~\ref{tbl_funnel} and
~\ref{tbl_next_steps}. In this example, it means that the user has
expressed interest in the page sequence \((p_2, p_2, p_3, p_1)\) and
the results are as explained in Table~\ref{tbl_example_1}.

\subsection{Extrapolation}

\subsubsection{Inputs}
\be
\item Date Range over which sequence chosen in Section~\ref{Exploration}
is to be evaluated. 
\ee
\subsubsection{Outputs}
See Table~\ref{tbl_extraploate}


\newpage
\clearpage
\begin{table}[ht]
\centering
\begin{tabular}{|l|l|} \hline \hline 
{\bf Sequence} & {\bf Occurrences} \\ \hline \hline
\((p_2)\) & \(n_1\) \\ \hline
\((p_2, p_2)\) & \(n_2\) \\ \hline
\((p_2, p_2, p_3)\) & \(n_3\) \\ \hline
\((p_2, p_2, p_3, p_1)\) & \(n_4\) \\ \hline \hline
\((p_2, p_2, p_3, p_1, p_1)\) & \(m_1\) \\ \hline 
\((p_2, p_2, p_3, p_1, p_2)\) & \(m_2\) \\ \hline 
\((p_2, p_2, p_3, p_1, p_3)\) & \(m_3\) \\ \hline 
\((p_2, p_2, p_3, p_1, p_4)\) & \(m_4\) \\ \hline 
\hline
\end{tabular}
\caption{Explanation of Example}
\label{tbl_example_1}
\end{table}

\begin{table}[ht]
\centering
\begin{tabular}{|l|l|} \hline \hline 
{\bf Page} & {\bf Cumulative Count} \\ \hline \hline 
\(p_2\) & \(n_1\) \\ \hline 
\(p_2\) & \(n_2\) \\ \hline 
\(p_3\) & \(n_3\) \\ \hline 
\(p_1\) & \(n_4\) \\ \hline 
\ldots & \ldots \\ \hline
\hline
\end{tabular}
\caption{Funnel Analysis}
\label{tbl_funnel}
\end{table}

\begin{table}[ht]
\centering
\begin{tabular}{|l|l|} \hline \hline 
{\bf Page} & {\bf Cumulative Count} \\ \hline \hline 
\(p_1\) & \(m_1\) \\ \hline 
\(p_2\) & \(m_2\) \\ \hline 
\(p_3\) & \(m_3\) \\ \hline 
\(p_4\) & \(m_4\) \\ \hline 
\ldots & \ldots \\ \hline
\hline
\end{tabular}
\caption{Next page (ranked in terms of popularity)}
\label{tbl_next_steps}
\end{table}

\begin{table}[ht]
\centering
\begin{tabular}{|l|l|l|l|} \hline \hline 
{\bf Sequence} & {\bf Count for Day 1} & {\bf Count for Day 2 } &
\ldots  \\ \hline \hline
\(p_1\)                & \(n_{11}\) & \(n_{12}\) & \ldots \\ \hline
\(p_1, p_2\)           & \(n_{21}\) & \(n_{22}\) & \ldots \\ \hline
\(p_1, p_2, p_3\)      & \(n_{31}\) & \(n_{32}\) & \ldots \\ \hline
\(p_1, p_2, p_3, p_4\) & \(n_{41}\) & \(n_{42}\) & \ldots \\ \hline
\ldots & \ldots & \ldots & \\ \hline
\hline
\end{tabular}
\caption{Example of extrapolation}
\label{tbl_extrapolate}
\end{table}

\section{Data Model}

\subsection{Sharding}
We shard data by date, so we have several versions of table T1, named
T1YYYYMMDD, one for each date.

\subsection{T1}
\label{T1}

Fields are 
\be
\item mid, member ID, I4
\item sid, session ID, I8
\item pid, page ID, I4
\item cnt, number of consecutive occurrences of pid for that sid
\item \verb+same_sess_as_prev_1+ I1. True if this row and previous row
have same sid; else, false. Created after load. 
\ee

\section{Queries}
\label{Queries}

\subsection{Query 1}
\label{Query_1}

\bi
\item Input is pid, \(p\)
\item Output is 
\be
\item number of occurrences of that pid. This is computed as follows.
Let \(T' \subseteq T\) be the rows of \(T\) where \(pid=p\). Then, return 
\be
\item \(\sum T'.cnt\)
\item \(|T'|\)
\item \(|T|\)
\ee
\item for each page that occurs after \(p\), provide
Table~\ref{Table_1}. Note that this is sorted descending on Count
\ee
\ei

\begin{table}[ht]
\centering
\begin{tabular}{|l|l|} \hline \hline 
{\bf Page ID} & {\bf Count} \\ \hline \hline 
\(p_2\) & \(n_2\) \\ \hline 
\(p_3\) & \(n_3\) \\ \hline 
\ldots & \ldots \\ \hline
\hline
\end{tabular}
\caption{Sample output for Query 1}
\label{Table_1}
\end{table}


\subsection{Page Sequence}
\label{Page_Sequence}

\bi
\item Input is \verb+p1:n1;p2:n2;p3:n3+, a sequence of pids and numbers.
This can be interepreted as we are looking for cases where
there were \(n_1\) visits to page \(p_1\) followed by 
\(n_2\) visits to page \(p_2\) followed by 
\(n_3\) visits to page \(p_3\).
Constraints  are
\be
\item \(p_i\) must be a valid page 
\item \(n_1\) must be a positive integer. 
\item \(p_i \neq p_{i+1}\)
\ee
\item Output is \(m_1, m_2, m_3\) where \(m_i\) is the number of times
the sub-sequence \((p_1, n_1), \ldots (p_i, n_i)\) occurred.
\ei
