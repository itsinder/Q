\startreport{Q --- An Analytics Framework}
\reportauthor{Ramesh Subramonian}

\newcommand{\YES}{\checkmark}


\section{Introduction}

Q is a vector language, designed for efficient implementation
of counting, sorting and data transformations. It uses a single data
structure --- a table.

\subsection{Motivation}

I will motivate the need for Q by quoting from two of my Gods --- Codd and
Iverson. I could be accused of quoting scripture for my purpose (see
below) and it is true that I am being selective in my extracts.
However, that does not detract from their essential verity.

\begin{verse}
The devil can cite Scripture for his purpose. \\ 
An evil soul producing holy witness \\ 
Is like a villain with a smiling cheek, \\
A goodly apple rotten at the heart:
\end{verse}

\subsubsection{Extracts from Codd}

{\tt 
The most important motivation for the research work that resulted in the
relational model was the objective of providing a sharp and clear
boundary between the logical and physical aspects of database
management. We call this the {\em data independence objective}.

A second objective was to make the model structurally simple, so that
all kinds of users and programmers could have a common understanding of
the data, and could therefore communicate with one another about the
database. We call this the {\em communicability objective}.

A third objective as to introduce high level language concepts *but not
specific syntax) to enable users to express operations upon large chunks
of information at a time. This entailed providing a foundation for
set-oriented processing (i.e., the ability to express in a single
statement the processing of multiple sets of records at a time). We
call this the {\em set-processing objective}.

To satisfy these three objectives, it was necessary to discard all those
data structuring concepts (e.g., repeating groups, linked structures)
that were not familiar to end users and to take a fresh look at
the addressing of data.
}

We have deviated from Codd's preference for the relational model.
Instead, we choose to drop down one level to the table. As Codd writes:

{\tt 
  Tables are at a lower level of abstraction than relations, since they
    give the impression that positional (array-type) addressing is
    applicable (which is not true of \(n\)-ary relations), and they fail
    to show that the information content of a table is independent of
    row order. Nevertheless, even with these minor flaws, tables are the
    most important conceptual representation of relations, because they
    are universally understood.

}

Lastly, in designing Q, we wanted it to be a data model as Codd defines
one

{\tt
A data model is a combination of at least three components:
  \be
  \item A collection of data structure types (the building blocks);
\item A collection of operators or rules of inference, which can be
  applied to any valid instances of the data types listed in (1), to
  retrieve, derive, or modify data from any parts of those structures in
  any combinations desired;
\item A collection of general  integrity rules, which implicitly or
  explicitly define the set of consistent database states or changes of
  states or both
  \ee

}

\subsubsection{Extracts from Iverson}

The importance of language has been stated over the centuries. Iverson
quotes the following from Whitehead:

{\tt By relieving the brain of all unnecessary work, a good notation
  sets it free to concentrate on more advanced problems, and in effect
    increases the mental power of the race}

In the same vein, he quotes Babbage:

{\tt The quantity of meaning compressed into small space by algebraic
  signs, is another circumstance that facilitates the reasonings we are
    accustomed to carry on by their aid}

I would hesitate to claim that Q meets any of the following criteria
that Iverson lays down for good notation. But it is  definitely the
guiding principle and aspirational goal for Q.

{\tt 
\be
\item Ease of expressing constructs arising in problems. If it is to be
effective as a tool of thought, a notation must allow convenient
expression not only of notions arising directly from a problem, but also
of those arising in subsequent analysis, generalization and
specialization.

\item Suggestivity. A notation will be said to be suggestive if the
forms of the expressions arising in one set of problems suggests related
expressions which find application in other problems.

\item Ability to subordinate detail. Brevity is achieved by
subordinating detail, and we will consider three important ways of doing
this
\bi
\item the use of arrays
\item the assignment of names to functions and variables
\item the use of operators
\ei

\item Economy. Economy requires that a large number of ideas be
expressible in terms of a relatively small vocabulary. 

\item Amenability to formal proofs

\ee
}


\subsection{Notations}
\begin{definition}
\(\delta(x, y)  = 1 \) if \(x = y\) and 0 otherwise
\end{definition}

\begin{notation}
For convenience, we shall often blur the distinction between arrays,
sets and tables.  Hence, \(T.f\) refers to field \(f\) of table
\(T\). Similarly, \(T[i].  f\) means the \(i^{th}\) value of field
\(f\) of table \(T\).  
\end{notation}

\begin{notation}
We shall often use \(T_X\) to refer to the table whose name is \(X\).
\end{notation}

\begin{notation}
Also, \(T|f(\ldots)\) refers to the subset of
\(T\) for which the predicate \(f(\ldots)\) is satisfied.
\end{notation}

\begin{notation}
\(Count(T)\) is cardinality of \(T\)
\end{notation}

\begin{notation}
\label{NumNN}
\(NumVal(T.f, v) = \sum_i \delta (T[i].f, v)\)
\end{notation}

\begin{notation}
\(Unique(T.f)\) be the unique values of \(T.f\)
\end{notation}

\begin{notation}
\(CU(T.f)\) is the count of the unique values of \(T.f =
Count(Unique(T.f))\)
\end{notation}

\begin{notation}
\((f_1:v_1 \ldots f_N:v_N) \in T \Rightarrow 
\exists j: T[j].f_1 = v_1 \wedge \ldots T[j].f_N = v_N \)
\end{notation}

\begin{notation}
\((v_1 \ldots v_N) = T[i].(f_1,\ldots f_N) \Rightarrow 
v_1 = T[i].f_1 \ldots v_N = T[j].f_N\)
\end{notation}

\begin{notation}
\(\exists ! \) is read as ``there exists a unique''
\end{notation}


\begin{definition}
Let \(T_1.f_1 = Join (T_2, f_2, l_2, l_1)\) be a field in \(T_1\)
  created by joining field \(f_2\) in Table \(T_2\) using \(l_2\) in
  \(T_2\) and \(l_1\) in \(T_1\) as the link fields.
\end{definition}


\section{System Capabilities}

\subsection{Field Types}
\label{Field_Types}
\bd
\item [B]  bit 
\item [I1] 1-byte signed integer
\item [I2] 2-byte signed integer
\item [I4] 4-byte signed integer
\item [I8] 8-byte signed integer
\item [F4] 4-byte float
\item [F8] 8-byte float
\item [SC]  string. Fixed length.
\item [SV]  string. Variable length. Must have columns len, off in same
table that allow us to identify value in a particular cell.
\ed
Note that if a string can be treated as a label, then we replace it
by an I4 integer which is foreign key to field {\tt idx} in a dictionary table. 

\be
\item If an operation is expected to create a table \(T\) and a table
by that name exists, then the original table is first deleted. Examples
of operations that create tables are
\be
\item add table --- Section~\ref{add_tbl}
\item load from CSV --- Section~\ref{dld}
\ee
\item 
If an operation is expected to create a field with name \(f\) and a
field with such a name exists, then the old field is deleted {\bf after}
the operation is performed and replaced by the newly created field. 
\item A table has a non-null, unique name.
must be unique. 
\item A field has a non-null name that is unique within a table.
\item Most operations cannot be performed on the nn field of a primary
field. When this is possible, we will say so explicitly. 
\ee

\begin{notation}
\(alldef(f)\) means that all values of \(f\) are defined. Hence, \(f\)
  cannot have an {\tt nn} field.
\end{notation}

I have disabled this invariant. Need to be sure that it is okay to do
so.
\begin{invariant}
SC field cannot have a nn field.
\end{invariant}

\subsection{Environment Variables}

\bd
\item [Q\_DOCROOT] Name of directory where meta data is stored. 
Needs to be unique for each table space. 
\item [Q\_DATA\_DIR] Name of directory where data is stored
Needs to be unique for each table space. 
\item [Q\_RUN\_TIME\_CHECKS] If it is set, then we execute run time.
\ed

%-----------------------------------------------------

\section{Functions}

\subsection{Accessing meta-data}

\subsubsection{file\_to\_fld}
\label{file_to_fld}

Input is \(n\), positive integer. Output is \verb+<tbl>:<fld>+ if there
is a field with this as its file number.

\subsubsection{list\_files}
\label{list_files}

Arguments are
\be
\item \(D\) --- data directory (if not specified, then environment
    variable \verb+Q_DATA_DIR+ is used)
\ee

Returns list of files in the docroot that are in this directory. Output
format is:
\be
\item is\_external --- 1 means that it is owned by somebody else; else, 0
\item file\_name  --- unique, non-null string
\ee

%--------------------------------------------------------------
\subsubsection{dump}
\label{dump}

Arguments are
\be
\item name of file into which tables are dumped
\item name of file into which fields are dumped
\ee

Creates CSV files of meta-data.

%--------------------------------------------------------------
\subsubsection{list\_tbls}
\label{list_tbls}

Returns list of tables in the docroot as a 3-column CSV text file
\be
\item  tbl ID
\item name
\item display name
\ee

%--------------------------------------------------------------
\subsubsection{orphan\_files}
\label{orphan_files}

Arguments are
\be
\item \(D\) --- data directory (if not specified, then environment
    variable \verb+Q_DATA_DIR+ is used)
\ee

Returns list of files in the data directory that are not in meta data. 

%----------------------------------------------------------
\subsubsection{is\_tbl}
\label{is_tbl}
Arguments 
\be
\item \(T\) --- table
\ee
If table \(T\) exists, output is \(1, tbl\_id)\). Else, 
output is \((0, -1)\)

%--------------------------------------------------------------
\subsubsection{is\_fld}
\label{is_fld}
Arguments 
\be
\item \(T\) --- table
\item \(f\) --- field
\ee

If table \(T\) exists and field \(f\) exists in table \(T\), 
output is \(1, fld\_id)\). Else, output is \((0, -1)\)

%--------------------------------------------------------------
\subsubsection{is\_aux\_fld}
\label{is_aux_fld}
Arguments 
\be
\item \(T\) --- table
\item \(f\) --- field
\item \(auxtype\) --- can be {\tt nn, len, off}
\ee

Returns \(1, fld\_id)\) if 
\be
\item \(f\) exists in \(T\) 
\item \(f\) is a primary field 
\item \(f\) has an auxiliary field of type auxtype
\ee
Else, returns \(0, -1)\)

\subsubsection{describe}
\label{describe}

Arguments
\be
\item \(T\) --- table
\item \(f\) --- field 
\item \(attribute\) --- 
\ee
There are 3 invocation types
\be
\item Only \(T\) specified.  For each field, we print
\be
\item ID
\item name
\item field type --- Section~\ref{Field_Types}
\item is external?
\item is fk? If so, print tbl ID; else, print -1
\ee
\item \(T, f\) specified. For that field, we print
\be
\item name
\item fname
\item fldtype
\item parent ID
\item nn fld ID 
\item dict tbl ID 
\item srttype
\item is external?
\item count --- only if field is I1 or B; else, -1 
\ee

\item All three specified. Prints that attribute of that field of that
table
\ee

\subsection{get\_nR}
\label{get_nR}
Arguments
\be
\item \(T\) --- table
\ee
Returns number of rows in table \(T\)


\subsection{Modifying meta-data}
%---------------------------------------------------------------
\subsubsection{rename}
\label{rename}
Two styles of invocation. 
\be
\item to rename a table 
\be
\item \(T_1\) --- old name of table
\item \(T_2\) --- new name. If such a table exists, it is deleted
\ee
\item to rename a field 
\be
\item \(T_1\) --- name of table
\item \(f_1\) --- old name of field 
\item \(f_2\) --- new name of field. If such a field exists, it is
deleted
\ee
\ee

\subsubsection{drop\_nn\_fld}
\label{drop_nn_fld}
\be
\item \(T\) 
\item \(f\)
\ee

If \(T.f\) has an {\tt nn} field, it is deleted i.e., all values of
\(f\) are now defined. It is important to be sure that you know the
value of the field in its undefined state. Usually, this will be 0, but
you had better be damn sure!


\subsubsection{del\_tbl}
\label{del_tbl}

Arguments
\be
\item \(T\) --- table
\ee

Restrictions
\be
\item No other field should depend on this table
\ee

Deletes table \(T\) and all fields in \(T\)

Can also be invoked as \verb+ q del_tbl tbl1:tbl2+

Can also be invoked as \verb+ q delete tbl1:tbl2+

\subsubsection{del\_fld}
\label{del_fld}

Arguments 
\be
\item \(T\) --- table
\item \(f\) --- field
\ee

Deletes field \(f\) in table \(T\). If \(f\) is a primary field, then
all auxiliary fields (if any) are deleted as well. If \(is\_external = 0\), then the corresponding file is {\bf not} deleted; else, it is deleted.

Can also be invoked as \verb+ q del_tbl tbl fld1:fld2+

Can also be invoked as \verb+ q delete tbl fld1:fld2+
%------------------------------------------------------------

%% \subsubsection{set\_nR}
%% \label{set_nR}
%% Arguments
%% \be
%% \item \(T\) --- table
%% \item \(n_R\) --- positive integer
%% \ee
%% 
%% Sets number of rows in table \(T\) to \(n_R\). Can be done only when
%% there are no fields in table. 

\subsubsection{set\_meta}
\label{set_meta}
Arguments
\be
\item \(T\) 
\item \(f\) 
\item \(attr\) 
\item \(value\) 
\ee

Sets attribute \(a\) of field \(f\) in table \(T\) to value \(v\). If
\(f\) is null, then it sets attribute \(a\) of table \(T\).
Options for \(a\) are
\bd
\item [cnt] --- sets count. \(f.fldtype \in\{B, I1\}\)
\item [srttype] --- can be set to {\tt unknown, ascending, descending,
  unsorted}
\item [fldtype] --- can be set to {\tt I1, I2, I4, I8, F4, F8}
\item [is\_dict\_tbl] --- can be set to true or false (table attribute)
\ed

\subsection{import}
\label{import}
Arguments are
\be
\item \(D_{from}\) --- from docroot 
\item \(T_{from}\)
\item \(T_{to}\)
\ee
Copies the table \(T_{from}\) from the docroot to the current docroot,
       \(D_{to}\)
Notes
\be
\item \(D_{from} \neq D_{to}\)
\item If \(T_{to}\) exists in \(D_{to}\), then it is deleted
\item If we delete a field in \(D_{to}\) or the entire table itself,
there is no impact on \(T_{from}\) in \(D_{from}\). (As an
  implementation note, this is because we mark the imported fields with
  an \verb+is_external+ flag and we do not allow modifications of these
  files. This is specially important for operations which modify the
  underyling storage like Section~\ref{fop}.)
\ee

%--------------------------------------------------------------
%--------------------------------------------------------------
\subsection{Accessing Data}
\subsubsection{pr\_fld}

\label{pr_fld}
Arguments 
\be
\item \(T\) --- table
\item \(f\) --- field
\item selection is \(f_c\) or \(lb:ub\) --- condition field or range. 
If null, all rows printed
\item output file name. If null, writes to stdout
\ee

Prints all rows of field \(f\) of table \(T\). Behavior of print can be
modified by specifying third parameter as either
\be
\item \(f_c\) --- printing of \(T[i].f\) is suppressed if \(T[i].f_c = false\).
\item \(lb:ub\) --- we print rows with lb as lower bound inclusive and
ub as upper bound exclusive
\ee

If range is specified, following must be satisified
\be
\item \(lb > 0 \)
\item \(ub \leq n_R\)
\item \(lb < ub\)
\ee

\subsubsection{bindmp}
\label{bindmp}
Arguments are
\be
\item \(T\) --- input table
\item \(f_1:f_2:\ldots f_N\) --- list of fields separated by colon. Must
have at least one field
\item \(f_c\) --- boolean field in \(T_1\) optional
\item \(F\) --- output file to be created
\item \(D\) --- output directory in which file is to be created. If
null, current working directory is used. 
\ee

Creates file \(F\) by dumping the desired fields, a row at a time. The
\(i^{th}\) row is output if either (i) \(f_c = \bot\) or (ii) 
  \(f_c \neq \bot \wedge f_c[i] = 1\) 

Restrictions are
\be
\item \(fldtype(f_i) \in \{ I1, I2, I4, I8, F4, F8\}\)
\item \(alldef(f_i)\)
\item \(f_c \neq \bot \Rightarrow fldtype(f_c) = I1\)
\ee

%--------------------------------------------------------------
\subsection{Creating Data}

\subsubsection{add\_tbl}
\label{add_tbl}
Arguments 
\be
\item \(T\) --- table
\item \(n_R\) --- number of rows. Must be specified.
\ee
Creates a table with name \(T\). If a table with such a name exists, it
will be deleted.

%----------------------------------------------------

\subsubsection{add\_fld}
\label{add_fld}

Creates field \(f\) in table \(T\), whose values are stored in specified
data file.  Arguments are 
\be
\item \(T\) --- table
\item \(f\) --- field
\item attributes --- string. This is a colon separated concatenation of
\verb+name=value+ pairs where the names are
\be
\item \verb+file+ --- name of file which contains data for field 
\item \verb+dir+ --- directory in which file exists
\item \verb+fldtype+ --- see Section~\ref{Field_Types} ( however, string
    is not allowed)
\ee
\ee

Error conditions
\be
\item \(T.n_R = 0 \)
\item If \(T.f\) exists, it will be over-written.
\item name of file must not be similar to internal files i.e., first
character is underscore and others are digits
\ee

%----------------------------------------------------

\subsubsection{dup\_fld}
\label{dup_fld}
Arguments are 
\be
\item \(T\) --- table
\item \(f_1\) --- field in \(T\)
\item \(f_2\) --- newly created field in \(T\)
\ee

Creates \(f_2\) in\(T\) which is a clone of \(f_1\). Requires
\be
\item \(f_1 \neq f_2\)
\item \(f_1.fldtype \not \in \{B, S\}\)
\ee

%----------------------------------------------------

\subsubsection{mk\_idx}
\label{mk_idx} 
Arguments are 
\be
\item \(T\) --- table
\item \(f\) --- field
\item \(fldtype \in \{I1, I2, I4, I8\}\)
\ee
Creates a field \(f\) of specified type which starts at 0 and
increments by 1.

%----------------------------------------------------
\subsubsection{get\_val}
\label{get_val}
Arguments are 
\be
\item \(T\) --- table
\item \(f\) --- field
\item \(i\) --- index 
\ee

Returns \(T[i].f\). 

Restrictions are
\be
\item \(0 \leq i < |T|\)
\item \(fldtype(f) \in \{I1, I2, I4, I8, F4, F8\}\)
\ee
%----------------------------------------------------

\subsubsection{set\_val}
\label{set_val}
Arguments are 
\be
\item \(T\) --- table
\item \(f\) --- field
\item \(l:u\) --- range, such that \(0 \leq l < u < nR\) 
\item \(v\) --- value 
\ee

\(\forall i: l \leq i < u, T[i].f = v\)

Restrictions are
\be
\item \(fldtype(f) \in \{I1, I2, I4, I8\}\)
\ee

\subsubsection{copy\_tbl}
\label{copy_tbl}

Creates \(T_2\) with as many rows as in \(T_1\) and copies all fields
from \(T_1\) to \(T_2\) using Section~\ref{copy_fld}. Note that string
fields do not get copied. Also, you cannot specify a conditional field.

LIMITATIONS: SC and SV fields do not get copied over

%----------------------------------------------------

\subsubsection{copy\_fld}
\label{copy_fld}

Creates field \(f_2\) table \(T_2\) by copying
\(T[i].f\) to \(T_2\) if \(T_1[i].f_c = true\). The order is preserved
in the copy operation.
Arguments are
\be
\item \(T_1\) --- input table 
\item \(f_1\) --- input field 
\item \(f_c\) or \(l:u\) --- optional condition field or range 
\item \(T_2\) --- output table 
\item \(f_2\) --- newly created output field 
\ee
Restrictions are
\be
\item If \(T_2\) does not exist, it will be created
\item If \(T_2\) exists
\be
\item If no selection specified, \(|T_1| = |T_2|\). 
\item If selection specified as \(f_c\), then \(NumVal(T_1.f_c, true), = |T_2|\)
\item If selection specified as \(l:u\), then \(|T_2| = u - l\)
\ee
\item \(Type(f_1)\) must be one of {\tt I1, I2, I4, I8, F4, F8}
\ee

\subsubsection{copy\_fld\_ranges}
\label{copy_fld_ranges}

Simiar to Section~\ref{copy_fld} but uses a set of ranges instead of a 
condition field.  Arguments are
\be
\item \(T_1\) --- input table 
\item \(f_1\) --- input field in \(T_1\)
\item \(T_r\) --- range table 
\item \(f_l\) --- lower bound field in \(T_r\)
\item \(f_u\) --- upper bound field in \(T_r\)
\item \(T_2\) --- output table 
\item \(f_2\) --- output field in \(T_2\)
\ee
Restrictions are
\be
\item If \(T_2\) does not exist, it will be created
\item \(0 \leq T_r[i].lb < T_r[i].ub \leq |T_1| \)
\item If \(T_2\) exists, \(\sum (T_r[i].ub - T_r[i].lb) = |T_2|\)
\item Field type of \(f_1\) must be one of {\tt I1, I2, I4, I8, F4, F8}
\item \(fldtype(f_l) = fldtype(f_u) = I8\)
\item \(alldef(f_l), alldef(f_u)\)
\ee

\subsubsection{mv\_fld}
\label{mv_fld}

Arguments are
\be
\item \(T_1\) --- input table 
\item \(f_1\) --- input field 
\item \(T_2\) --- output table 
\ee
Moves field \(f\) from table \(T_2\) to \(T_1\)

Restrictions
\be
\item \(|T_1| = |T_2|\)
\item \(T_1 \neq = T_2\)
\item \(fldtype(f_1) \in \{I1, I2, I4, I8, F4, F8\}\)
\ee

\subsubsection{mk\_nn\_fld}
\label{mk_nn_fld}

Arguments are
\be
\item \(T_S\) --- input table 
\item \(f_S\) --- input field 
\item \(T_D\) --- output table 
\item \(f_D\) --- output field 
\ee

Makes field \(f_S\) the {\tt nn} field of field \(f_D\). Note that
\(f_S\) will cease to exist.

Restrictions
\be
\item \(|T_S| = |T_D|\)
\item It is okay for \(T_S = T_D\)
\item \(fldtype(f_S) = \{B, I1\} \) 
\item \(alldef(f_S)\)
\item \(f_S\) must not be in use by any other field 
\ee

\subsubsection{break\_nn\_fld}
\label{break_nn_fld}

Arguments are
\be
\item \(T_S\) --- input table 
\item \(f_S\) --- input field 
\item \(T_D\) --- output table 
\item \(f_D\) --- output field 
\ee

Makes field \(f_D\) from the {\tt nn} field of field \(f_S\). At
end,\(alldef(f_S) = true\)

Restrictions
\be
\item \(alldef(f_S) = false\)
\item It is okay for \(T_S = T_D\)
\item \(fldtype(f_D) = \{B, I1\} \) 
\item \(f_S \neq f_D\)
\ee

%----------------------------------------------------

\subsection{Operations}
\subsubsection{regex\_match}
\label{regex_match}

Arguments are
\be
\item \(T\)
\item \(f\)
\item \(r\) --- regular expression
\item \(k\) --- kind of match 
\item \(f_C\) --- newly created condition field in \(T\)
\ee 

\be
\item \(fldtpye(f_C) = I1\)
\item \(fldtpye(f) \in \{SC, SV\}\)
\item \(k \in \{exact\}\) (other kinds of matches to be supported)
\ee

\subsubsection{stride}
\label{stride}
Arguments are 
\be
\item \(T_S\) --- source table
\item \(f_S\) --- input field 
\item \(i_0\) --  starting position
\item \(m\)   --  stride
\item \(T_D\) --- destination table
\item \(f_D\) --- count field 
\item \(n_D\) --- number of rows in destination table.
\ee

Selects every \(m^{th}\) value of \(T_S.f_S\) starting from index \(i_0\).
Creates \(T_D\) with a single column \(f_D\) and \(n_D\) rows as
\(T_D[i].f_D = T_S[i*m + n].f_S\).
\bi
\item \(0 \leq n < n_S\)
\item \(0 \leq m < n_S\)
\item 
\(n_D\) will be reduced from the desired value so that
\((n_D-1)*m + n < |T_S|\)
\ei

\be
\item \(fldtype(f_S) \in \{I4, I8\}\)
\item \(alldef(f_S)\)
\ee

%---------------------------------------------------------------

\subsubsection{count}
\label{count}
Arguments are 
\be
\item \(T_S\) --- source table
\item \(f_S\) --- input field 
\item \(f_c\) --  optional condition field in \(T_S\).
\item \(f_D\) --- destination table
\item \(f_D\) --- count field 
\ee

\(T_D[j].f_D = |T_S[i].f_S = j|\)

\(f_c \neq \bot \Rightarrow T_D[j].f_D = |T_S[i](f_S = j, f_C = 1)\)

Restrictions are 
\be
\item \(f_S\) has no {\tt nn} field 
\item fldtype of \(f_D\) is {\tt I1, I2, I4, I8}
\item \(|T_D| \leq I4_{max}\). For good peformance, it should be quite small.
\item \(|T_S| \leq I4_{max}\). 
\item \(fldtype(f_D) = I4\)
\item \(0 \leq f_S < |T_D| \) --- \(f_S\) must be thought of as an
index into \(T_D\)
\ee

%---------------------------------------------------------------

\subsubsection{countf}
\label{countf}
Arguments are 
\be
\item \(T_S\) --- source table
\item \(f_I\) --- field that serves as index into \(T_D\)
\item \(f_V\) --- field whose values are to be accumulated
\item \(f_c\) --  optional condition field in \(T_S\).
\item \(f_D\) --- destination table
\item \(f_D\) --- count field 
\item whether to run safe or not, optional field, default true.
Specified as \verb+safe_mode=[true]+
\ee

Section~\ref{count} is a special case of this operator where \(f_V =
  1\).
\be
\item  \(f_C = \bot \Rightarrow 
T_D[j].f_D = \sum_i \delta(j, T_S[i].f_I) \times T_S[i].f_V\)
\item  \(f_C \neq \bot \Rightarrow 
T_D[j].f_D = \sum_i f_C[i] \delta(j, T_S[i].f_I) \times T_S[i].f_V\)
\ee

Restrictions are 
\be
\item \(f_S\) has no {\tt nn} field 
\item fldtype of \(f_D\) is {\tt I1, I2, I4, I8}
\item \(|T_D| \leq I4_{max}\). For good peformance, it should be quite small.
\item \(|T_S| \leq I4_{max}\). 
\item \(fldtype(f_D) = I4\)
\item \(0 \leq f_S < |T_D| \) --- \(f_S\) must be thought of as an
index into \(T_D\)
\item If safe mode is false, then we do not check each value of \(f_I\)
  to see whether it is in the range \([0, |T_D|-1]\). Can lead to
  segfaults if you do not know what you are doing!
\ee

%--------------------------------------------------------------

\subsection{count\_ht}
\label{count_ht}

The function {\tt count} in Section~\ref{count} is useful when 
the range of values is \(0, 1, \ldots, |T_2|-1\). When one
does not know the range of values, \verb+count_ht+ is useful.

Arguments are 
\be
\item \(T_S\) --- source table
\item \(f_S\) --- input field 
\item \(f_C\) --- condition field  (optional)
\item \(f_D\) --- destination table
\ee

Creates table \(T_D\) with 2 fields, {\tt value} and {\tt count}.

\be
\item \(|T_D| < 2^{21}\). Important limitation of current implementation.
\item \(T_S \neq T_D\)
\item \(f_S \neq f_C\)
\item \(fldtype(f_S) \in \{I1 I2, I4, I8\}\)
\item \(T_D.value \subseteq T_S.f_S\)
\item \(T_D[j].count = |T_S[i].f_S = T_D[j].value\)
\item \(fldtype(value) = fldtype(f_S)\)
\item \(fldtype(count) = \) smaller of {\tt I4, I8}. Note that I8 used
only if some value occurs more than \(I4_{max}\) times.
\ee


Restrictions are 
\be
\item Number of distinct values of \(f_S\) must be less than \(2^{20}\).
This limitation should be lifted.
\item For any particular value of \(f_S\), the number of rows with that
value must be \(leq I4_{max}\)
\item Note that \(T_D\) might be empty. Caller needs to check.
\ee

%--------------------------------------------------------------

\subsection{s\_to\_f}
\label{s_to_f}
Produces a field from a scalar. Arguments are 
\be
\item \(T\) 
\item \(f\) 
\item how to create the field 
\ee

The following options are supported
\bd
\item [CONST] In this case a value must be a provided.
Restrictions
\be
\item fldtype is {\tt I1, I2, I4, F4, I8}
\ee
\item [SEQ]   In this case, a starting value and an increment must be
provided. 
Restrictions
\be
\item fldtype is {\tt I2, I4, I8}
\item No overflow/underflow of values e.g., if fldtype is I1,
  then values produced must be in range \([-128, 127]\)
\ee
\ed

Operations supported for Section~\ref{s_to_f} are in
Table~\ref{tbl_s_to_f}
\begin{table}[hb]
\centering
\begin{tabular}{|l||l|l|l|l|l|l|l|l|}  \hline \hline
{\bf Operation} & {\bf B} & {\bf I1} & {\bf I2} & {\bf I4} & {\bf I8}
& {\bf F4 } & {\bf F8} \\ \hline \hline
const       &      & \YES & \YES & \YES & \YES & \YES &    \\ \hline
seq         &      &      & \YES & \YES & \YES &      &    \\ \hline
set\_labels &      &      &      & \YES &      &      &    \\ \hline
\hline
\end{tabular}
\caption{Supported Operations for s\_to\_f}
\label{tbl_s_to_f}
\end{table}

%--------------------------------------------------------------
\subsection{pack}
\label{pack}

Arguments are
\be
\item \(T\)
\item \(f_1:f_2:f_3 \ldots f_N\) --- input fields
\item \(n_1:n_2:n_3 \ldots f_N\) --- non-negative integers indicating shift
amount
\item \(fldtype\) --- type of output field \(f_o\)
\item \(f_o\)  --- output field created
\ee

Creates \(f_o\) in \(T\) by performing an ``or'' of 
(i) \(f_1\) shifted left by \(n_1\), 
(ii) \(f_2\) shifted left by \(n_2\), \ldots

Restrictions
\be
\item \(2 \leq N \leq 8 \)
\item \(fldtype(f_o) \in \{I4, I8\}\)
\item all input fields must be alldef
\item fldtypes of all input fields must be in \(\{I1, I2, I4, I8\}\)
\item \(f_i \geq 0 \) --- no negative values in input 
\item Output field not one of input fields
\item \(fldtype(f_o) = I4 \Rightarrow fldtype(f_i) \neq I8\)
\ee

%--------------------------------------------------------------
\subsection{unpack}
\label{unpack}

Arguments are
\be
\item \(T\)
\item \(f_I\)  --- input field 
\item \(n_1:n_2:n_3 \ldots n_N\) --- non-negative integers indicating shift
amount
\item \(m1:m2:m3\) --- number of bits for each input field 
\item \(t1:t2:t3\) --- types of output fields
\item \(f_1:f_2:f_3 \ldots f_N\) --- output fields 
\ee

Creates field \(f_i\) in \(T\) of type \(t_i\) by (unsigned) shifting
\(f_I\) \(n_i\) bits to the right and then ``and-ing'' it with a mask
which is all zeroes with the \(m_i\) least signicant bits being 1. 


Restrictions
\be
\item \(2 \leq N \leq 8 \)
\item \(fldtype(f_I) \in \{I4, I8\}\)
\item input field must be alldef
\item fldtypes of all output fields must be in \(\{I1, I2, I4, I8\}\)
\item Output field not one of input fields
\item \(fldtype(f_i) = I1 \Rightarrow m_i \leq 7\)
\item \(fldtype(f_i) = I2 \Rightarrow m_i \leq 15\)
\item \(fldtype(f_i) = I4 \Rightarrow m_i \leq 31\)
\item \(fldtype(f_i) = I8 \Rightarrow m_i \leq 63\)
\item \(fldtype(f_I) = I4 \Rightarrow fldtype(f_i) \neq I8\)
\ee


%--------------------------------------------------------------
\subsection{subsample}
\label{subsample}

\be
\item \(T_1\) --- input table (has \(n_1\) rows)
\item \(f_1\) --- input field 
\item \(n_2\) --- number of rows in output table
\item \(T_2\) --- output table
\item \(f_2\) --- output field 
\ee

Creates table \(T_2\) with single field \(f_2\) which contains \(0 < n_2 <
n_1\) rows of \(T_1.f_1\) selected at random. 
\be
\item \(fldtype(f_1) \in \{ I4, I8\}\)
\item \(fldtype(f_2) = fldtype(f_2)\)
\item \(alldef(f_1)\)
\item This is not completely uniformly at random because of need to
cilk-ify the loop. So, what we do is to break the input into blocks and
then select uniformly at random from each block.
\ee

%--------------------------------------------------------------
%
\subsection{wisifxthenyelsez}
\label{wisifxthenyelsez}

\be
\item \(T\) --- table
\item \(w\) --- output field 
\item \(x\) --- boolean field 
\item \(y\) --- input field (or scalar)
\item \(z\) --- input field (or scalar)
\ee

Creates field \(w\), whose type is same as that of field \(y\) or \(z\),
depending on which is specified. 
Algorithm in Figure~\ref{pseudo_code_wisifxthenyelsez}. Restrictions
\be
\item \(fldtype(y) = fldtype(z)\)
\item \(fldtype(y) \in \{I1, I2, I4, I8, F4\}\)
\item It is possible for one of \(y\) or \(z\) (but not both) to be a scalar.
In that case, the type of \(w\) is the type of the field that is not a
scalar. 
\ee

\begin{figure}
\centering
\fbox{
\begin{minipage}{15cm}
\begin{tabbing} \hspace*{0.25in} \=  \hspace*{0.25in} \= \kill
{\bf if} \(T_x[i] = \bot\) then \+ \\
  \(T_w[i] \la \bot \) \- \\
{\bf else} \+  \\
  {\bf if} \(T_x[i] = true\) then \+  \\
    \(T_w[i] \la T_y[i] \) \- \\
  {\bf else} \+  \\
    \(T_w[i] \la T_z[i] \) \- \\
  {\bf endif} \-  \\
{\bf endif} 
\end{tabbing}
\end{minipage}
}
\label{pseudo_code_wisifxthenyelsez}
\caption{Pseudo-code for wisifxthenyelsez}
\end{figure}


%--------------------------------------------------------------

\subsection{fop}
\label{fop}
Arguments are 
\be
\item \(T\) --- table
\item \(f\) --- field in \(T\). Modified in situ
\item \(\circ\)  --- operation on \(f\) 
\ee

Restrictions are
\be
\item \(alldef(f)\)
\item \(internal(f)\)
\ee

Operations supported for Section~\ref{fop} are in
Table~\ref{tbl_fop}
\begin{table}[hb]
\centering
\begin{tabular}{|l||l|l|l|l|l|l|l|l|}  \hline \hline
{\bf Operation} & {\bf B} & {\bf I1} & {\bf I2} & {\bf I4} & {\bf I8}
& {\bf F4 } & {\bf F8} \\ \hline \hline
permute        &      &      &      & \YES & \YES &      &    \\ \hline
sort           &      &      &      & \YES & \YES &      &    \\ \hline
saturate       &      & \YES & \YES & \YES & \YES &      &    \\ \hline
zero\_after\_n & \YES &      &      &      &      &      &    \\ \hline
\hline
\end{tabular}
\caption{Supported Operations for fop}
\label{tbl_fop}
\end{table}


\be
%----------------------------------------------
\item permute {\tt op=[permute]}
Field types supported are {\tt I4}. Does random permutation.
%----------------------------------------------
\item sort ascending, {\tt op=[sort]:order=[asc]}. 
Field types supported are {\tt I4, I8}
%----------------------------------------------
\item sort descending {\tt op=[sort]:order=[dsc]}
Field types supported are {\tt I4, I8}
%----------------------------------------------
\item saturate {\tt op=[saturate]:maxval=[user specified]}
Field types supported are {\tt I1, I2, I4, I8}. 

\(T[i].f \leftarrow max(T[i].f, maxval)\)
%----------------------------------------------
\item zero\_after\_n {\tt op=[zero\_after\_n]:n=[M]}
\bi
\item Field types supported are {\tt B}. 
\item \(0 < M < |T|\)
\item Starting with \(i = |T|-1\) and decrementing \(i\) by 1, set 
\(T[i].f \leftarrow 0\) as long as \(\sum_{j=1}^{j=i-1}T[i].f > n\).
Note that we might include \(m \geq n\) 1's but \(m < n+64\). 
\ei
\ee

%--------------------------------------------------------------

\subsection{num\_in\_range}
\label{num_in_range}

Arguments are 
\be
\item \(T_1\)
\item \(f_1\) --- field in \(T_1\)
\item \(T_2\)
\item \(f_{lb}\) --- field in \(T_2\)
\item \(f_{ub}\) --- field in \(T_2\)
\item \(f_C\) --- newly created field in \(T_2\)
\ee

Consider the \(i^{th}\) row of \(T_2\). 
Let \(l = T_2[i].lb, u = T_2[i].ub\). Then, \(T_2[i].cnt\) counts the
number of rows of \(T_1\) where \(l \leq f_1 \leq u\)

Restrictions
\be
\item More efficient when \(srttype(f_1) = asc\)
\item \(f_{lb} \neq f_{ub}\)
\item ranges specified by \(f_{lb},  f_{ub}\) are non-overlapping
\item \(fldtype(f_1) = fldtype(f_{lb}) = fldtype(f_{ub}) = I4\)
\item \(alldef(f_1), alldef(f_{lb}), alldef(f_{ub})\)
\item \(T_2[j].f_{lb} \leq T_2[j].f_{ub}\)
\item \(|T_1| < I4_{max}\)
\item \(fldtype(f_C) = I4\)
\item Recommend usage of this only when \(|T_2|\) is small. In
particular, \(|T_2| \leq 4096\)
\item \(srttype(f_{lb}) = srttype(f_{ub}) = \) ascending
\ee

\subsection{f1s1opf2}
\label{f1s1opf2}

Arguments are 
\be
\item \(T_1\)
\item \(f_1\)
\item scalar (or colon separated list of scalars)
\item \(f_2\)
\ee

Operations supported for Section~\ref{f1s1opf2} are in
Table~\ref{tbl_f1s1opf2}
\begin{table}[hb]
\centering
\begin{tabular}{|l||l|l|l|l|l|l|l|}  \hline \hline
{\bf Operation} & {\bf I1} & {\bf I2} & {\bf I4} & {\bf I8}
& {\bf F4 } & {\bf F8} \\ \hline \hline
\verb=+=      & \YES & \YES & \YES & \YES & \YES &    \\ \hline
\verb+-+      &      &      & \YES & \YES & \YES &    \\ \hline
\verb+*+      &      &      & \YES & \YES & \YES &    \\ \hline
\verb+/+      &      &      & \YES & \YES & \YES &    \\ \hline
\verb+%+      &      &      & \YES & \YES &      &    \\ \hline
\verb+&+      & \YES &      & \YES & \YES &      &    \\ \hline
\verb+|+      & \YES &      & \YES & \YES &      &    \\ \hline
\verb+^+      & \YES &      & \YES & \YES &      &    \\ \hline
\verb+>+      & \YES &      & \YES & \YES & \YES &    \\ \hline
\verb+<+      & \YES &      & \YES & \YES & \YES &    \\ \hline
\verb+>=+     & \YES &      & \YES & \YES & \YES &    \\ \hline
\verb+<=+     & \YES &      & \YES & \YES & \YES &    \\ \hline
\verb+!=+     & \YES &      & \YES & \YES & \YES &    \\ \hline
\verb+==+     & \YES & \YES & \YES & \YES & \YES &    \\ \hline
\verb+<<+     &      &      & \YES & \YES &      &    \\ \hline
\verb+>>+     &      &      & \YES & \YES &      &    \\ \hline
\verb+<=||>=+ &      &      & \YES &      &      &    \\ \hline
\verb+>&&<+   &      &      & \YES &      &      &    \\ \hline
\verb+>=&&<=+ &      &      & \YES &      &      &    \\ \hline
\hline
\end{tabular}
\caption{Supported Operations for f1s1opf2}
\label{tbl_f1s1opf2}
\end{table}



%--------------------------------------------------------------

\subsection{f1f2\_to\_s}
\label{f1f2_to_s}
Produces a scalar from 2 fields. Arguments are 
\be
\item \(T\) --- table
\item \(f_1\) --- field in \(T\). 
\item \(f_2\) --- field in \(T\). 
\item \(\circ\)  --- operation on \(f\) 
\ee
\bd
\item [\(circ = sum\)]. Add up the values of \(f_2\) where \(f_1 = 1\)
\be
\item \(fldtype(f_1) = I1\)
\item \(fldtype(f_2) = \{I1, I2, I4, I8\}\)
\item \(alldef(f_1), alldef(f_2)\)
\ee
\ed

%--------------------------------------------------------------

\subsection{f\_to\_s}
\label{f_to_s}
Produces a scalar from a field. Arguments are 
\be
\item \(T\) --- table
\item \(f\) --- field in \(T\). 
\item \(\circ\)  --- operation on \(f\) 
\ee

\begin{table}[hb]
\centering
\begin{tabular}{|l||l|l|l|l|l|l|l|}  \hline \hline
{\bf Operation} & {\bf B} & {\bf I1} & {\bf I2} & {\bf I4} & {\bf I8}
\\ \hline \hline
{\bf min}        &      & \YES & \YES & \YES & \YES \\ \hline
{\bf max}        &      & \YES & \YES & \YES & \YES \\ \hline
{\bf sum}        & \YES & \YES & \YES & \YES & \YES \\ \hline
{\bf is\_sorted} &      & \YES & \YES & \YES & \YES \\ \hline
{\bf get\_idx}   &      & \YES & \YES & \YES & \YES \\ \hline
{\bf approx\_uq} &      &      &      & \YES &      \\ \hline
\hline
\end{tabular}
\caption{Supported Operations for f\_to\_s}
\label{tbl_f_to_s}
\end{table}

Operations supported are in Table~\ref{tbl_f_to_s}. The acronym mms
stands for min, max, sum.
Depending on the operation, output is as follows
\be
\item is sorted --- ascending, descending, unsorted, unknown
\item get\_idx --- call looks like \verb+op=[get_idx]:val=[v]+ 

Returns \(i\) such that \(T[i].f = v\). If no \(i\) satisifes the
inequality, -1 is returned. If more than 1 satisifies it, an aribtrary
one is returned.

Current restrictions are
\be
\item \(Type(f)\) = {\tt I1, I2, I4, I8}
\item \(f\) has no {\bf nn} field 
\ee

\item min --- \(n_1:n_2\) where \(n_1\) is min value and \(n_2\) is
number of non-null values
\item max --- \(n_1:n_2\) where \(n_1\) is max value and \(n_2\) is
number of non-null values
\item sum --- \(n_1:n_2\) where \(n_1\) is sum       and \(n_2\) is
number of non-null values
\item mms --- \(n_1:n_2:n_3:n_4\) where \(n_1\) is min, \(n_2\) is max,
  \(n_3\) is sum and \(n_4\) is number of non-null values
\ee

%-------------------------------------------------------------

\subsection{f1opf2}
\label{f1opf2}

Arguments are 
\be
\item \(T_1\) --- table
\item \(f_1\) --- field in \(T_1\)
\item \(\circ\)  --- operation to b performed on \(f_1\)
\item \(f_2\) --- newly created field in \(T_1\)
\ee

Creates field \(f_1\) in \(T_1\) by performing operation \(\circ\) on
\(f_1\) in \(T_1\). 
Operations supported are in Table~\ref{tbl_f1opf2}. Detailed
explanations in Section~\ref{details_f1opf2}.
\begin{table}[hb]
\centering
\begin{tabular}{|l||l|l|l|l|l|l|l|}  \hline \hline
{\bf Operation} & {\bf B} & {\bf I1} & {\bf I2} & {\bf I4} & {\bf I8} & {\bf F4 } & {\bf F8} \\ \hline \hline
{\bf conv    }  & \YES & \YES &      & \YES & \YES & \YES & \YES \\ \hline
{\bf bitcount}  &      &      &      & \YES & \YES &      &   \\ \hline
{\bf sqrt    }  &      &      &      &      & \YES &      & \YES  \\ \hline
{\bf abs     }  &      &      &      &      &      &      & \YES  \\ \hline
{\bf reciprocal}&      &      &      &      & \YES &      & \YES \\ \hline
\verb+!+        &      & \YES &      & \YES &      &      &   \\ \hline
\verb=++=       &      &      &      & \YES &      &      &   \\ \hline
\verb+--+       &      &      &      & \YES &      &      &   \\ \hline
\verb+~+        &      & \YES &      & \YES &      &      &   \\ \hline
{\bf hash}      &      &      &      & \YES & \YES &      &   \\ \hline
{\bf shift   }  &      & \YES & \YES & \YES & \YES &      &   \\ \hline
{\bf cum     }  &      & \YES & \YES & \YES & \YES &      &   \\ \hline
{\bf smear   }  &      & \YES &      &      &      &      &   \\ \hline
\hline
\end{tabular}
\caption{Supported Operations for f1opf2}
\label{tbl_f1opf2}
\end{table}

\begin{table}[hb]
\centering
\begin{tabular}{|l||l|l|l|l|l|l|l|}  \hline \hline
{\bf From} & {\bf To} \\ \hline \hline
B  & I1 \\ \hline 
\hline
I1 & B \\ \hline 
I1 & I4 \\ \hline 
I1 & I8 \\ \hline 
\hline
I2 & I4 \\ \hline 
I2 & I8 \\ \hline 
%
\hline
I4 & I1 \\ \hline 
I4 & I2 \\ \hline 
I4 & I8 \\ \hline 
I4 & F4 \\ \hline 
%
\hline
I8 & I1 \\ \hline 
I8 & I4 \\ \hline 
I8 & F4 \\ \hline 
%
\hline
F4 & I4 \\ \hline 
F4 & I8 \\ \hline 
F4 & F8 \\ \hline 
\hline
\end{tabular}
\caption{Supported conversions for f1opf2}
\label{tbl_conversions}
\end{table}


\subsubsection{Details for f1opf2}
\label{details_f1opf2}
We explain each of the operations below
\bd
\item [CONV] Converts from one type to another. Supported conversions
are in Table~\ref{tbl_conversions}
\item [BITCOUNT] 
Counts the number of bits.
Example \verb+op=bitcount+
\(Type(f_2) = I4\)
\item [SQRT] 
Operation is \(\sqrt{x}\). 
Example \verb+op=sqrt+
\(Type(f_2) = Type(f_1)\)
\item [NEGATION] Toggles bits. 
Example is \verb+op=!+
\(Type(f_2) = Type(f_1)\)
\item [ONE's COMPLEMENT] 
Takes one's complement. 
Example is \verb+op=~+
\(Type(f_2) = Type(f_1)\)
\item [INCREMENT]
Increments by 1
Example is \verb~op=++~
\(Type(f_2) = Type(f_1)\)
\item [DECREMENT]
Decrements by 1
Example is \verb~op=--~
\(Type(f_2) = Type(f_1)\)
\item [CUM] 
Accumulates the values of \(f_1\).  Notes
\be
\item 
\(i = 0 \Rightarrow f_2[i] = f_1[i]\); else, \(f_2[i] = f_1[i] + f_2[i-1]\)
\item \(alldef(f_1)\)
\item user has to specify argument {\tt newtype} which specifies
\(fldtype(f_2)\) and which can be {\tt I4, I8}
\item \(fldtype(f_1) = \{I1, I2, I4, I8\}\)
\ee

\item [SHIFT] Shifts the values up or down up to a maximum of 16
Example is \verb+op=shift:val=N+. In this case,
\(f_2[i] = f_1[i+N]\). If \(N > 0 \), then it like a shift down
of the column i.e., the first \(N\) values of \(f_2\) are set to {\tt
  newval}. 
  If
\(N < 0\), then it is like a shift up and the last \(N\) values will be
set to {\tt newval}. 
Note that \(N \neq 0\). Currently, \(|N| \leq 16\).

%% \item [HASH] hashes input value. Notes
%% \be
%% \item \(fldtype(f_1) = \{I4, I8\}\)
%% \item user has to specify argument {\tt seed} which is an unsigned long long
%% \item \(fldtype(f_2) = I8\)
%% \ee
\item [SMEAR] ``smears'' the selection by \(n_R\) to the right and
\(n_L\) to the left i.e., 
\bi
\item 
\(f_1[i] = 1 \Rightarrow \forall j: 1 \leq j \leq n_R, f_2[i+j] \leftarrow 1\)
  \item 
\(f_1[i] = 1 \Rightarrow \forall j: 1 \leq j \leq n_L, f_2[i-j] \leftarrow 1\) 
\ei
\be
\item user has to specify argument {\tt plus} which is the number of
positions to the right that the selection should be smeared
\item user has to specify argument {\tt minus} which is the number of
positions to the right that the selection should be smeared
\item \(n_R \geq 0\), \(n_L \geq 0\), \(n_R\) and \(n_L\) cannot both be
0
\item There is a maximum value for \(n_R, n_L\)
\item \(fldtype(f_1) = fldtype(f_2) = I1\)
\item \(alldef(f_1)\)
\ee
\ed

There are 2 arcane statistical functions, for which \(fldtype(f_1) =
fldtype(f_2) = F8\). They are
\be
\item normal cdf inverse
\item calculation of p-value from z-value
\ee

%-------------------------------------------------------------
\subsection{xfer}
\label{xfer}

Arguments are 
\be
\item \(T_1\)
\item \(f_S\) --- must exist in \(T_1\)
\item \(T_2\)
\item \(f_{idx}\) --- must exist in \(T_2\)
\item \(f_D\) --- newly created in \(T_2\)
\ee

Creates \(f_D\) in \(T_2\) with same type as \(f_S\) as follows.
\(T_2[i].f_D \leftarrow T_1[T_2[i].f_{idx}].f_S\)

Restrictions
\be
\item \(0 \leq T_2[i].f_{idx} < |T_2|\)
\item \(fldtype(f_S) \in \{ I1, I2, I4, I8, F4, F8, SC\}\)
\item \(fldtype(f_{idx}) = I4\)
\ee

%-------------------------------------------------------------

\subsection{is\_a\_in\_b}
\label{is_a_in_b}

\be
\item \(T_1\) --- table
\item \(f_1\) --- field in \(T_1\). 
\item \(T_2\) --- table
\item \(f_2\) --- field in \(T_2\). 
\item \(f_c\) --- newly created condition field in \(T_1\). 
\item \(f_S\) --- source field in \(T_2\)
\item \(f_D\) --- destination field in \(T_1\)
\ee

One of 
\be
\item 
\(\forall i: T[i].f_1 \in T_2.f_2 \Rightarrow T[i].f_c \leftarrow true\)
\item \(\exists j: T_2[j].f_2 = T_1[i].f_1 \Rightarrow T_1[i].f_D
\leftarrow T_2[j].f_S\)
\ee

Restrictions
\be
\item \(Type(f_c) = I1\)
\item Either specify \(f_c\) or specify \(f_S, f_D\) but not both
\be
\item \(f_c = \bot \Rightarrow f_S \neq \bot \wedge f_D \neq \bot\)
\item \(f_c \neq \bot \Rightarrow f_S = \bot \wedge f_D = \bot\)
\ee
\item \(fldtype(f_1) = fldtype(f_2)\)
\item \(fldtype(f_1) = \) {\tt I1, I2, I4, I8}
\item \(f_2\) has no null values
\item \(f_2\) is sorted in ascending order
\ee

\subsection{f1opf2f3}
elabel{f1opf2f3}

Arguments are 
\be
\item \(T_1\) --- table
\item \(f_1\) --- field in \(T_1\)
\item \(\circ\)  --- operation to be performed on \(f_1\)
\item \(f_2\) --- field in \(T_1\)
\item \(f_3\) --- newly created field in \(T_1\)
\ee

Creates field \(f_2, f_3\) in \(T_1\) by performing operation \(\circ\) on
\(f_1\) in \(T_1\). 

Restrictions
\be
\item \(fldtype(f_1) = I8\)
\item \(fldtype(f_2) = fldtype(f_3)= I4\)
\item \(op = unconcat\). Breaks up the input 8-byte integer into 2
4-byte integers
\ee

\subsection{f1f2opf3}
\label{f1f2opf3}

Arguments are 
\be
\item \(T_1\) --- table
\item \(f_1\) --- field in \(T_1\)
\item \(f_2\) --- field in \(T_1\)
\item \(\circ\)  --- operation to be performed on \(f_1\)
\item \(f_3\) --- newly created field in \(T_1\)
\ee

Creates field \(f_3\) in \(T_1\) by performing operation \(\circ\) on
\(f_1\, f_2\) in \(T_1\). 
Operations supported are in Table~\ref{tbl_f1f2opf3}. 
\begin{table}[hb]
\centering
\begin{tabular}{|l||l|l|l|l|l|l|l||l|}  \hline \hline
{\bf Operation} & {\bf B} & {\bf I1} & {\bf I2} & {\bf I4} & {\bf I8} & {\bf F4 } & {\bf F8} & {\bf Type\((f_3)\)}\\ \hline \hline
 concat    &      &      &      & \YES &      &      &   & I8 \\ \hline
\verb=+=   &      &      & \YES & \YES & \YES & \YES &   & \\ \hline
\verb+--+  &      &      & \YES & \YES & \YES & \YES &   & \\ \hline
\verb+*+   &      &      &      & \YES & \YES & \YES &   & \\ \hline
\verb+/+   &      &      &      & \YES & \YES & \YES &   & \\ \hline
\verb+%+   &      &      &      & \YES & \YES &      &   & \\ \hline
\verb+&&+  & \YES & \YES &      &      &      &      &   & \\ \hline
\verb+||+  & \YES & \YES &      &      &      &      &   & \\ \hline
\verb+>+   &      &      &      & \YES & \YES & \YES &   & I1 \\ \hline
\verb+<+   &      &      &      & \YES & \YES & \YES &   & I1 \\ \hline
\verb+>=+  &      &      &      & \YES & \YES & \YES &   & I1 \\ \hline
\verb+<=+  &      &      &      & \YES & \YES & \YES &   & I1 \\ \hline
\verb+!=+  &      &      &      & \YES & \YES & \YES &   & I1 \\ \hline
\verb+==+  &      &      &      & \YES & \YES & \YES &   & I1 \\ \hline
\verb+&&!+ & \YES &      &      &      & \YES &      &   & \\ \hline
\verb+&+   &      &      &      & \YES & \YES &      &   & \\ \hline
\verb+|+   &      &      &      & \YES & \YES &      &   & \\ \hline
\verb+^+   &      &      &      & \YES & \YES &      &   & \\ \hline
\verb+<<+  &      &      &      & \YES & \YES &      &   & \\ \hline
\verb+>>+  &      &      &      & \YES & \YES &      &   & \\ \hline
\hline
\end{tabular}
\caption{Supported Operations for f1f2opf3}
\label{tbl_f1f2opf3}
\end{table}

Most operations in Section~\ref{f1f2opf3} are self-explanatory. Here are
the rest
\be
\item concat \(f_3 \leftarrow (f_1 << 32) | f_2 \)
\item \verb+&&!+ \(f_3 \leftarrow (f_1 \wedge (\neg f_2) \)
  \ee

Restrictions
\be
\item \(Type(f_1) = Type(f_2)\)
\item \(Type(f_3) = Type(f_1)\), except when otherwise noted in last
column of Table~\ref{tbl_f1f2opf3}
\ee

\subsection{crossprod}
\label{crossprod}

Arguments are 
\be
\item \(T_1\)
\item \(f_{11}\) --- field in \(T_1\) 
\item \(f_{12}\) --- field in \(T_1\) 
\item \(T_2\)
\item \(f_{21}\) --- newly created field in \(T_2\) 
\item \(f_{22}\) --- newly created field in \(T_2\) 
\item \(T_{aux}\) --- optional 
\item \(f_{cnt}\) --- optional 
\item \(mode\)
\ee

Creates a new table \(T_2\) with fields \(f_{21}\) and \(f_{22}\).  
If such a table exists, it is deleted. 

Mode can be one of
\be
\item complete
\item upper triangular 
\item upper triangular minus diagonal
\ee

To understand this operation, start by assuming that \(T_{aux} = \bot\).
Let \(n_2 = |T_2|, n_1 = |T_1|\). 
Assume \(f_{11} = (1, 2, 3)\)
and    \(f_{21} = (4, 5, 6)\)
\be
\item mode = {\tt COMPLETE}. 
\(n_2 = n_1 \times n_1\). 
\((f_{21}, f_{22}) = \{
(1,4), (1, 5), (1, 6), 
(2,4), (2, 5), (2, 6), 
(3,4), (3, 5), (3, 6)\}
\)
\item mode {\tt UPPER TRIANGULAR }. 
\(n_2 = \frac{n_1 \times (n_1+1)}{2}\).
\((f_{21}, f_{22}) = \{
(1,4), (1, 5), (1, 6), 
       (2, 5), (2, 6), 
               (3, 6)\}
\)
\item mode = {\tt UPPER TRIANGULAR MINUS DIAGONAL}. 
\(n_2 = \frac{n_1 \times (n_1-1)}{2}\).
\((f_{21}, f_{22}) = \{
       (1, 5), (1, 6), 
               (2, 6)
                     \}
\)
  \ee

Restrictions
\be
\item \(n_1 \geq 2\) 
\item \(T_{aux}  = \bot \Leftrightarrow f_{cnt} = \bot\)
\item \(alldef(f_{11}), alldef(f_{12}) \)
\item \(fldtype(f_{11}) = fldtype(f_{12}) = I4\)
\item \(fldtype(f_{21}) \leftarrow I4\)
\item \(fldtype(f_{22}) \leftarrow I4\)
\ee


\subsection{is\_A\_in\_b}
\label{is_A_in_b}

\be
\item \(T_A\) --- table
\item \(l_A\) --- link field in \(T_A\)
\item \(T_B\) --- table
\item \(l_A\) --- link field in \(T_B\)
\item \(f_c\) --- newly created field in \(T_A\)
\ee

Restrictions
\be
\item \(srttype(l_A) = srttype(l_B = \) ascending
\item \(alldef(l_A) = alldef(l_B = \) true
\item \(fldtype(l_A) = fldtype(l_B \)
\item \(fldtype(l_A) \in \{I4, I8\}\)
\item \(fldtype(l_B) \in \{I4, I8\}\)
\ee
Creates field \(f_c \in T_A\), such that 
\be
\item \(alldef(f_c) = \) true
\item \(fldtype(f_c) = I1\)
\item \(T_A[i].f_c = true \Rightarrow \exists j: T_A[i].l_A =
T_B[j].l_B\)
\ee

Considerations
\be
\item Current implementation best used when \(T_A\) is large and \(T_B\)
is small. This could be fixed when we bring meta data to bear.
\ee

\subsection{srt\_join}
\label{srt_join}

\be
\item \(T_s\) --- table
\item \(l_s\) --- link field in \(T_1\)
\item \(f_s\) --- source field in \(T_1\)
\item \(T_d\) --- table
\item \(l_d\) --- link field in \(T_2\)
\item \(f_d\) --- newly created field in \(T_2\)
\item \(m\) --- type of join. 
\ee

\be
\item 
\(m = \) {\tt reg, min, max, sum, cnt, and, or}
\ee


Let \(T_d[i].l_d = v_d\). \(T'_s = T_s[l_s = v_d]\). 
Then, \(T_s' = \phi \Rightarrow T_d[i] = \bot\).  Depending on the mode, 
\bd
\item [reg] \(T_d[i].f_d \in T_s'.f_s\). If \(|T'_s] > 0 \), then an
arbitrary value in it is used.
\item [min] \(T_d[i].f_d = min(T_s'.f_s)\)
\item [max] \(T_d[i].f_d = max(T_s'.f_s)\)
\item [sum] \(T_d[i].f_d = \Sigma(T_s'.f_s)\)
\item [and] \(T_d[i].f_d = and(T_s'.f_s)\)
\item [or ] \(T_d[i].f_d = or(T_s'.f_s)\)
\item [cnt] \(T_d[i].f_d = |T_s'|)\)
\ed

\subsection{rng\_join}
\label{rng_join}

\be
\item \(T_s\) --- source table
\item \(l_s\) --- source field in \(T_1\)
\item \(f_l\) --- lower bound field in \(T_1\)
\item \(f_u\) --- upper bound field in \(T_1\)
\item \(T_d\) --- destination table
\item \(l_d\) --- link field in \(T_2\)
\item \(f_d\) --- newly created field in \(T_2\)
\ee

%
\be
\item \(|T_s| \leq I4_{max}\)
\item \(f_l \neq f_u\)
\item \(l_s \neq f_u\)
\item \(l_s \neq f_l\)
\item \(fldtype(f_l) = fldtype(l_u) = fldtype(l_d) = I4\)
\item \(srttype(f_l) = srttype(l_u) = srttype(l_d) = asc\)
\item \(alldef(l_s),   alldef(l_u),   alldef(l_d)\)
\item \(fldtype(f_d) = I4\)
\item \(l_s \neq \bot \Rightarrow alldef(l_s), fldtype(l_s) = I4\)
\item \(T_S[j].f_l \leq T_S[j].f_u\)
\ee

Let \(v = T_D[i].l_d\). If \(\not \exists j: T_S[j].f_l \leq v \leq
T_S[j].f_u \Rightarrow T_D[i].f_d = \bot\). Else, 
\bi
\item \(f_l = \bot \Rightarrow T_D[i].f_d = j\)
\item \(f_l \neq \bot \Rightarrow T_D[i].f_d = T_S[j].f_l\)
\ei


%-----------------------------------------------------------

\subsection{count\_vals}
\label{count_vals}

Arguments are 
\be
\item \(T_1\) --- input table 
\item \(f_1\) --- input field  in \(T_1\)
\item \(f^c_1\) --- optional counter field in \(T_1\)
\item \(T_2\) --- output table 
\item \(f_2\) --- output field in \(T_1\)
\item \(f^c_2\) --- counter field in \(T_2\)
\ee

Restrictions
\be
\item \(Type(f_1) = Type(f_2\)
\item \(Type(f_1) = \) {\tt I4, I8}
\item \(Type(f^c_2) = \) {\tt I4}. This means that 
\be
\item if \(f^c_1\) is not provided, then the maximum number of 
occurrences of any value in \(f_1\) is \(I4_{max}\)
\item if \(f^c_1\) is provided, similar restriction (see below)
\ee
\ee

We create a table \(T_2\) where \(f_2\) is the unique, non-null values
of \(f_1\) in \(T_1\). If \(f^c_1\) is not specified, \(f^c_2[i] =
|T_1[f_1 = T_2[f^c-2[i]]|\), the number of times the corresponding
value of \(f_1\) occurred in \(T_1\). If it is specified, then the
output is computed in much the same way but weighted by \(f^c_1\). In
other words, let \(T_2[j].f_2 = V\). Then, 
\begin{displaymath}
T_2[j].f^c_2 = \sum_i T_1[i].f^c_1[i] \times \delta(T_1[i].f_1, V) 
\end{displaymath}

\subsubsection{Enhancements}
Do not create an output count field if not desired. % TODO

%-----------------------------------------------------------

\subsection{mk\_bins}
\label{mk_bins}

Arguments are
\be
\item \(T_1\)
\item \(f_1\)
\item \(T_2\) --- newly created table 
\item \(n_2\) --- desired number of bins
\ee

Creates \(n_2'\) rows in \(T_2\) (where \(1 \leq n_2' \leq n_2\)) with columns
\be
\item \verb+lb+  Type is I4
\item \verb+ub+  Type is I4
\item \verb+cnt+ Type is I8
\ee
such that 
\be
\item \(cnt > 0 \)
\item \(lb < ub \)
\item \(cnt\) is as evenly distributed as possible
\item \(n_2'\) is as close to \(n_2\) as possible
\item \(\sum T_2[i].cnt = |T_1|\)
\ee

Restrictions
\be
\item \(alldef(f_1)\) 
\item \(srttype(f_1) = asc\) 
\item \(fldtype(f_1) = I4\)
\item Currently, we use a greedy algorithm that can result in the bin
sizes being not as evenly distributed as one would like them to be.
\ee

%-----------------------------------------------------------

\subsection{t1f1t2f2opt3f3}
\label{t1f1t2f2opt3f3}

Arguments are
\be
\item \(T_1\)
\item \(f_1\)
\item \(T_2\) 
\item \(f_2\)
\item \(\circ\) operation to be performed
\item \(T_3\) 
\item \(f_3\)
\ee

Creates field \(T_3\) with single column \(f_3\) of same type as input
fields. 

Restrictions
\be
\item Supported operations for \(\circ\) are 
\be
\item \(A \cup B\) --- union
\item \(A \cap B\) --- intersection
\item \(A - B\) --- \verb+a_minus_b+
\item pvalcalc --- Calculation of \(p\)-value
\ee
\item \(srttype(f_1) = srttype(f_2) = asc\) 
\item \(fldtype(f_1) = fldtype(f_2)  \in \{I4, I8\}\)
\item \(alldef(f_1),  alldef(f_2)\)
\item Because of input restrictions, 
  \(fldtype(f_3) \leftarrow fldtype(f_1)\) and \(srttype(f_3) = asc\)
\ee

%-------------------------------------------------------------

\subsection{binld}
\label{binld}
Similar to bindmp in Section~\ref{bindmp}.
Arguments are
\be
\item \(T\) --- table to be created
\item \(f_1:f_2:\ldots f_N\) --- list of fields separated by colon. Must
have at least one field
\item \(t_1:t_2:\ldots t_N\) --- where \(t_i\) is specifier of field
type for field \(f_i\). 
\item \(F\) --- input file to be read
\item \(D\) --- input directory from which file is to be read. If
null, current working directory is used. 
\ee

Creates table \(T\) from file \(F\). Restrictions are 
\be
\item \(fldtype(f_i) \in \{I1, I2, I4, I8, F4, F8\}\)
\item \(alldef(f_i)\)
\ee

\subsection{percentiles}
\label{percentiles}
Arguments are
\be
\item \(T_I\) --- input table
\item \(f\)   --- field in input table
\item \(T_O\) --- output table 
\item \(n\) --- desired number of rows in output table
\ee
\bi
\item Breaks input into \(n\) bins of equal size, last bin being potentially
larger than others. 
\item Creates \(T_O\), where every row corresponds to a bin, 
with 4 fields --- min, max, avg, cnt.
\be
\item \(min[i]\) is smallest value in bin \(i\)
\item \(max[i]\) is largest value in bin \(i\)
\item \(cnt[i]\) is number of values in bin \(i\)
\item \(avg[i]\) is average value in bin \(i\)
\ee
\ei


Restrictions
\be
\item \(fldtype(f) \in \{I4, I8, F4, F8\}\)
\item \(alldef(f)\)
\item \(srttype(f) = ascending\)
\item \(2 \leq n < min(|T|, 1000)\)
\ee

\subsection{rng\_sort}
\label{rng_sort}

Sorts \(f_1\) in \(T_1\) in batches specified by \(T_2\)
\be
\item \(f_{lb}\) --- lower bound field in \(T_2\) 
\item \(f_{ub}\) --- upper bound field in \(T_2\) 
\item \(mode\) --- {\tt A} for ascending, {\tt D} for descending
\ee

The rows of \(T_2\) denote non-overlapping ranges in \(T_1\). These
ranges are sorted independently.

Restrictions
\be
\item \(fldtype(f_1) \in \{I4, I8\}\)
\item \(fldtype(f_{lb})  = fldtype(f_{ub}) = I8\)
\item \(alldef(f_1), alldef(f_{lb}), alldef(f_{ub})\) 
\item \(external(f_1) = false\)
\item \(0 < T_2[i].f_{lb} < T_2[i].f_{ub} \leq |T_1|\)
\ee

\subsection{app\_tbl}
\label{app_tbl}

Appends table \(T_2\) to \(T_1\). \(T_2\) is untouched. 
Arguments are
\be
\item \(T_1\) --- destination table 
\item \(T_2\) --- source table
\ee
Restrictions are
\be
\item Every field in \(T_1\) must be present in \(T_2\) and have same
type in both \(T_1\) and \(T_2\)
\item Allowable field types are {\tt I1, I2, I4, I8, F4, F8}
\item All fields in \(T_1\) are internal (not external)
\item Note that it is possible for \(T_1 = T_2\)
\ee

\subsection{sortf1f2}
\label{sortf1f2}

Arguments are
\be
\item \(T_1\) --- input table 
\item \(f_1\) --- primary input field 
\item \(f_2\) --- secondary input field 
\item \(srttype\) --- see below
\ee
Restrictions are
\be
\item \(f_1 \neq f_2\)
\item \(alldef(f_1), internal(f_1)\)
\item \(alldef(f_2), internal(f_2)\)
\item \(fldtype(f_1) = \{I4, I8\}\)
\item \(fldtype(f_2) = \{I4, I8\}\)
\item srttype must be one of 
\be
\item \verb+A_+ --- \(f_1\) primary ascending and \(f_2\) drag along
\item \verb+AA+ --- \(f_1\) primary ascending and \(f_2\) secondary ascending
\item \verb+AD+ --- \(f_1\) primary ascending and \(f_2\) secondary descending
\item \verb+D_+ --- \(f_1\) primary descending and \(f_2\) drag along
\item \verb+DA+ --- \(f_1\) primary descending and \(f_2\) secondary ascending
\item \verb+DD+ --- \(f_1\) primary descending and \(f_2\) secondary descending
\ee
\item Currently only \verb+A_+ and \verb+D)+ implemented
\ee

\subsection{no\_op}
\label{no_op}

As expected, does nothing. Although it writes a line to the log file. 

\subsection{dld}
\label{dld}

Meta data file has 3 columns
\be
\item field name. Must be alphanumeric.
\item field type. See Section~\ref{Field_Types}, except for
\be
\item {\tt B}
\item {\tt SC}
\ee
\item auxiliary information about the field. This is optional. It can
contain the following
\be
\item \verb+is_load=[true|false]+ Whether to load field or not. Default
is true.
\item \verb+is_all_def=[true|false]+ Whether all values are defined or
not. Default is false. If we discover that all values are defined,
then we can drop the {\tt nn} field that is created to mark undefined
values.
\item \verb+is_null_if_missing=[true|false]+ This is used for fields
that are lookups into an existing dictionary. Default is false.  
\bi
\item When
set to true, if we find a value that is not in the dictionary, we mark
it as null.  
\item When set to false, if we find a value that is not in the
dictionary, we quit the dld operation.
\item \verb+is_dict_old=[true|false]+ Whether to use an existing
directory or create a new one.
\ei
\ee
\ee
When {\tt fldtype = SV}, it is converted into I4 and we use
\verb+dict_tbl_id+ to denote the lookup table used to map ints to
strings.

\begin{invariant}
\(is\_load = false \Rightarrow dict = \bot\)
\end{invariant}

\begin{invariant}
\(is\_null\_if\_missing = true \Rightarrow is\_dict\_old = true \)
\end{invariant}

\begin{invariant}
Consider 2 fields being loaded in a given table. If both are going to
create a new table, the tables must be different. 
\(is\_dict\_old_i = false \wedge is\_dict\_old_j = false \Rightarrow dict_i
\neq dict_j\)
\end{invariant}

\subsubsection{Input Restrictions}
\label{Input_Restrictions}

\be
\item 
When reading a CSV file, the length of a text field is limited to 32767
characters.
\ee

%% \subsection{get\_t2f2\_val}
%% \label{get_t2f2_val}
%% Arguments are 
%% \be
%% \item \(T_D\) --- destination table 
%% \item \(l_D\) --- link field in destination table 
%% \item \(T_S\) --- source table 
%% \item \(l_S\) --- link field in source table 
%% \item \(f_S\) --- field in source table
%% \item \(f_D\) --- newly created field in destination table
%% \ee
%% 
%% 
%% Let \(T_D[i].l_D = l_d\). 
%% If \(\exists j : T_S[j].l_S = l_d\), then let \(T_D[i].f_D
%% \leftarrow T_D[j].f_S\); else, \(T_D[i] = \bot\)
%% If there are multiple values of \(j\), then an arbitrary one is
%% selected. 
%% Restrictions
%% \be
%% \item \(fldtype(l_D) = fldtype(l_S)\)
%% \item \TBC
%% \ee
% 
\subsection{lkp\_sort}
\label{lkp_sort}

Arguments are 
\be
\item \(T_1\)
\item \(f_1\) --- input field to be sorted
\item \(T_2\) --- table with information about number of each value of
\(f_1\)
\item \(f_{cnt}\) --- field in \(T_2\)
\item \(f_{idx}\) --- newly created field in \(T_1\) {\bf OPTIONAL}.
\(fldtype(T_1.f_{idx}) = I8, alldef(T_1.f_{idx})\)
\item \(f_{srt}\) --- newly created field in \(T_1\)
\(fldtype(T_1.f_{srt}) = I4, alldef(T_1.f_{srt})\)
\ee

This is an ascending sort which does a one-pass sort provided one knows
the distribution of the data. For example, let us say that we have a
field \(f_{cnt}\) in table \(T_2\) such that the \(i^{th}\) row 
tells us the number of values of \(f_1\) that are \(= i\).
Then \(f_{srt}\) is a permutation of
\(f_1\) in sorted order ascending and \(f_{idx}\) is created as if we did a
drag-along of the index field \(0, 1, 2, \ldots\)

Restrictions
\be
\item \(fldtype(f_{cnt}) = I8\)
\item \(fldtype(f1) = I8\)
\item \(alldef(f_{cnt}) alldef(f_1)\)
\item \(0 \leq T_1.f_1 < |T_2| \)
\ee

%------------------------------------------------------
\section{GPU Operations}

As a convention, all GPU operations will start with \verb+g_+
We will use {\tt D} to indicate device and {\tt H} to indicate host.

I have not fully understood how to share the GPU across multiple
concurrent Q invocations. Need to think through this further. 

\subsection{g\_load}
\label{g_load}

Arguments are 
\be
\item \(T\)   --- string, table
\item \(f_H\) --- string, field
\item \(f_D\) --- string, register name 
\ee

Internally, this is translated to the following load operation
\be
\item file name ---  string, name of binary file which contains data 
to be loaded
\item field type --- string, Can be any of \verb+B, I1, I2, I4, I8, F4, F8+
\item \(n_R\). Number of rows. Uses for book-keeping.
\item \(r\) --- register number. 
\item over write or not --- Could be 
\be
\item \verb+N+ means operation fails if register is in use or 
\item \verb+Y+ means free register if in use and then perform operation
\ee
\ee

Above operation fails if any of the validations fail e.g.,
\be
\item file not found
\item \(r\) not a valid registe number. Valid numbers are currently
\([0, 31]\)
\item Insufficient space on device
\item \ldots
\ee

\subsection{g\_info}
\label{g_info}

Prints following meta data to stdout for each GPU register that is in use.

\be
\item index
\item Size in bytes
\item File name from which it was loaded 
\item Table name from which it was loaded 
\item Field name from which it was loaded 
\item Logical name provided to it by Q 
\ee

\section{Compound Expressions}
\label{Compound}

The motivation behind compound statements is allowing users to take
get the benefits of strip-mining. The user needs to issue the command
{\tt q start\_compound} to denote that subsequent commands will be part of
a compound expression. None of the subsequent commands are executed
(they are simply recorded) until the user issues a {\tt q
  stop\_compound} statement.

At this time, they are first verified and then executed.

\subsection{Verification}
\be
\item Only one table should be referenced. We will relax this gradually
to allow operations like count which read from one table and write to
another. 
\item Given above limitation, the table referenced should be the same in
each statement 
\item operations should come from a limited set. They are either 
\be
\item regular e.g., \verb+q <op> <tbl> .......+
\be
\item \verb+s_to_f+
\item \verb+f1opf2+
\item \verb+f1s1opf2+
\item \verb+f1f2opf3+
\ee
\item reduction e.g., \verb+A=`q <op> <tbl> .....`+. In this case,
{\tt A} is the destination of the reduction 
\be
\item \verb+f_to_s+
\item \verb+f1f2_to_s+
\ee
\ee
\item ephemeral fields are suffixed with the prime character i.e.,
  \verb+[']+. ephemeral fields must be created before they are referenced 
\item all non-ephemeral fields that are referenced must either exist
prior to the start of the compound expression or must be created in a
statement preceding the one in which it is referenced
\item reduction operations, if any, should come at end
\item the destinations of reduction operations should be unique
\ee



\subsection{For some later point in time}

A compound statement is of the form
\begin{verbatim}
x' = a + b; y' = (int)x'; z' = x' > y'; w = z' ? e : f
\end{verbatim}
which is the same as Figure~\ref{fig_compound}. Note that any variable
with a prime is a temporary variable. Hence, in this example, a column
called \(w\) is created but columns \((x', y', z')\) are not created.
Some limitations of a compound statement
\be
\item All the assignments, except the last one, must be to temporary
variables
\item The last assignment must be a to a permanent variable
\item Each assignment is separated by a semi-colon
\item The maximum number of assignments is 16. This is an arbitrary
number, which can be increased when we are more comfortable.
\ee



\begin{figure}[hb]
\centering
\fbox{
\begin{minipage}{15cm}
\begin{tabbing} \hspace*{0.25in} \=  \hspace*{0.25in} \= \kill
{\bf for} \(i \leftarrow 0\) {\bf to} \(|T| -1 \) {\bf do} \+ \\
   \(x' \leftarrow a[i] + b[i] \) \\
   \(y' \leftarrow (int)x' \) \\ 
  {\bf if } \(x' > y' \) {\bf then} \+ \\
    \(z' \leftarrow true \) \- \\
  {\bf else} \+ \\ 
    \(z' \leftarrow false \) \- \\
  {\bf endif} \\
  {\bf if } \(z' == true\) {\bf then} \+ \\
    \(w[i] \leftarrow e[i] \) \- \\ 
  {\bf else} \+ \\ 
    \(w[i] \leftarrow f[i] \) \- \\ 
  {\bf endif} \- \\
{\bf endfor} \\ 
\end{tabbing}
\end{minipage}
}
\caption{Explanation of compound statement}
\label{fig_compound}
\end{figure}


