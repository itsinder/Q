\documentclass[letterpaper]{article}
\usepackage{times}
\usepackage{helvet}
\usepackage{courier}
\usepackage{fancyheadings}
\usepackage{hyperref}
\pagestyle{fancy}
\usepackage{pmc}
\usepackage{graphicx}
\setlength\textwidth{6.5in}
\setlength\textheight{8.5in}

%%%%%%%%%%%%%%%%%%%%%%%%%%%%%%%%%%%%%%%%%%%%%%%%%%%%%%%%%%%%%%%%%%%%%%%%%%%%%%%%%%
% Special for e-unibus doc commands

\newcommand{\ForLater}{
\begin{center}
{\bf NOT FOR CURRENT VERSION}
\end{center}
}
\newcommand{\TBC}{\framebox{\textbf{TO BE COMPLETED}}}
\newcommand{\DISCUSS}{\Ovalbox {\bf \textcolor{red}{FOR DISCUSSION}}}
\newcommand{\Input}{\framebox{\textsf{in}}}
\newcommand{\Output}{\framebox{\textsf{out}}}
\newcommand{\debug}[1]{\textbf{debug start} #1 \textbf{debug finish}}
\newcommand{\inx}[1]{\emph{#1}}
\newtheorem{notation}{Notation}
\newtheorem{definition}{Definition}
\newtheorem{problem_statement}{Problem Statement}
\newtheorem{invariant}{Invariant}
\newtheorem{assumption}{Assumption}
\newtheorem{resource_string}{Resource String}
\newtheorem{testcase}{Test Case}
\newtheorem{note}{Note}
\newtheorem{specification}{Specification}
\newtheorem{caution}{Caution}
\newtheorem{prereq}{Pre-requisite}
\newtheorem{action}{Action}
\newtheorem{query}{Query}
\newcommand{\be}{\begin{enumerate}}
\newcommand{\ee}{\end{enumerate}}
\newcommand{\bi}{\begin{itemize}}
\newcommand{\ei}{\end{itemize}}
\newcommand{\bv}{\begin{verbatim}}
\newcommand{\ev}{\end{verbatim}}
\newcommand{\bd}{\begin{description}}
\newcommand{\ed}{\end{description}}
\newcommand{\bpre}{\begin{prereq}}
\newcommand{\epre}{\end{prereq}}
\newcommand{\bact}{\begin{action}}
\newcommand{\eact}{\end{action}}
\newcommand{\bs}{\begin{specification}}
\newcommand{\es}{\end{specification}}
\newcommand{\btc}{\begin{testcase}}
\newcommand{\etc}{\end{testcase}}
\newcommand{\bc}{\begin{caution}}
\newcommand{\ec}{\end{caution}}
\newcommand{\la}{\leftarrow}
\newcommand{\IpArgs}{\subsection{Input Arguments}}
\newcommand{\PreReqs}{\subsection{Pre-requisities}}
\newcommand{\Actions}{\subsection{Actions}}
\newcommand{\Coverage}{{\bf To test coverage.}}

%%%%%%%%%%%%%%%%%%%%%%%%%%%%%%%%%%%%%%%%%%%%%%%%%%%%%%%%%%%%%%%%%%%%%%%%%%%


\newtheorem{theorem}{Theorem}[section]
\newtheorem{lemma}[theorem]{Lemma}
\newtheorem{proposition}[theorem]{Proposition}
\newtheorem{corollary}[theorem]{Corollary}

\newenvironment{proof}[1][Proof]{\begin{trivlist}
\item[\hskip \labelsep {\bfseries #1}]}{\end{trivlist}}
\newenvironment{intuition}[1][Intuition]{\begin{trivlist}
\item[\hskip \labelsep {\bfseries #1}]}{\end{trivlist}}
%% \newenvironment{definition}[1][Definition]{\begin{trivlist}
%% \item[\hskip \labelsep {\bfseries #1}]}{\end{trivlist}}
\newenvironment{example}[1][Example]{\begin{trivlist}
\item[\hskip \labelsep {\bfseries #1}]}{\end{trivlist}}
\newenvironment{remark}[1][Remark]{\begin{trivlist}
\item[\hskip \labelsep {\bfseries #1}]}{\end{trivlist}}

\newcommand{\qed}{\nobreak \ifvmode \relax \else
      \ifdim\lastskip<1.5em \hskip-\lastskip
      \hskip1.5em plus0em minus0.5em \fi \nobreak
      \vrule height0.75em width0.5em depth0.25em\fi}

%%%%%%%%%%%%%%%%%%%%%%%%%%%%%%%%%%%%%%%%%%%%%%%%%%%%%%%%%%%%%%%%%
% \newcommand{\Alogon}{\mbox{\fontfamily{ptm}\selectfont {\large \selectfont A} \hspace{-1.2ex} {\large \selectfont L} \hspace{-2.3ex} \raisebox{0.45ex}{ {\footnotesize \selectfont O} } \hspace{-1.80ex} {\large \selectfont G} \hspace{-1.80ex} \raisebox{-0.33ex}{ {\large \selectfont O} } \hspace{-1.8ex} {\large \selectfont N}}}


\begin{document}
\title{Open-sourcing ``qtils''}
\author{ Ramesh Subramonian }
\maketitle
\thispagestyle{fancy}
\lhead{}
\chead{}
\rhead{}
\lfoot{{\small Decision Sciences Team}}
\cfoot{}
\rfoot{{\small \thepage}}

\section{Introduction}

{\tt qtils} is a collection of data-wrangling utilities. While Hadoop has
emerged as the de-facto standard for large data analysis tasks, there are
often a large number of simple data manipulation operations that need to
be performed in interactive mode as a post-processing step. The data for
post-processing is often small enough to fit on a user's desktop. {\tt
  qtils} is collection of efficient utilities for a data scientist.

In our experience, the life of a data scientist is often less glamorous
than the public perception. Accelerating the speed of inquiry --- both
in terms of run time and scripting time --- significantly decreases some
of the pains that come with the territory, opening up time to do
higher-level thinking.

Our hope is that open-sourcing {\tt qtils} will make it become the
``go-to'' toolbox for data scientists, in much the same way that the
number of shell utilities have become invaluable for a sysadmin.

At LinkedIn, {\tt qtils} has proved useful for a variety of tasks e.g.,
\begin{enumerate}
\item path analysis --- to validate hypotheses of how users navigate
pages.  Sponsor is Jim Baer. 
\item nCRM --- creating lists for marketing.  Sponsor is Simon Zhang.
\item data driven testing. Sponsor was Bob Schulman.
\end{enumerate}

\subsection{External Interest}

While {\tt qtils} was designed primarily as a useful toolkit for data
scientists, we believe it can be used in other contexts as well. A case
in point is our paper (collaboration with the Data Infra team) entitled
``In Data Veritas --- Data Driven Testing for Distributed
Systems'', which was presented at a SIGMOD workshop this year. After the
presentation, several members of the audience expressed an interest in
experimenting with the tools and approach.

We believe by open sourcing, we can both contribute to the community
while benefiting from their feedback and contributions. 

\section{Examples of usage}

For interactive data exploration, approximate results are often good
enough. Examples are 
\be
\item approximate quantiles --- gives the user a high-level distribution
of the data e.g., how are ``number of connections'' distributed?
\item top \(n\) --- often we are interested in the items that occur most
frequently e.g., which are the top 10 industries of members who have
visited a company page in the last month
\item approximate count distinct --- given every ad click, select ads
from a given advertiser and determine number of members who saw an ad
from that advertiser
\ee

\section{Operational aspects of open-sourcing}

\begin{enumerate}
\item Every source file has the following header with the copyright
symbol \(\copyright\) at the start
\begin{verbatim}
<copyright symbol> [2013] LinkedIn Corp. All rights reserved.
Licensed under the Apache License, Version 2.0 (the "License");
you may
not use this file except in compliance with the License.
You may obtain
a copy of the License at  http://www.apache.org/licenses/LICENSE-2.0
 
Unless required by applicable law or agreed to in writing,
software
distributed under the License is distributed on an "AS IS"
BASIS,
WITHOUT WARRANTIES OR CONDITIONS OF ANY KIND, either express or
implied.
\end{verbatim}
\item
Code has been checked in to gitli
\url{https://gitli.corp.linkedin.com/qtils/qtils}
\item Currently, we do not plan to make it an Apache project, although
this is a possibility in the future.
\item Documentation has been written as a PDF document and is available
along with the source code.
\item A few ``hello world'' style examples have been provided to jump
start developers.
\item Currently, {\tt qtils} is invoked from the shell. Our early
experimentation convinces us that creating bindings to other languages
--- Ruby, PHP, Python \ldots --- is not particuarly difficult. We hope
to find champions in the open source community who would take this on.
\end{enumerate}

\section{Installation}
\be
\item Copy the file {\tt qtils.tgz} into a directory and {\tt cd} to that directory
\item {\tt tar -zcvf qtils.tgz}
\item {\tt cd src; make}
\item You should get a single file called {\tt q}
\ee

You need to set 2 environment variables.
\be
\item 
\verb+Q_DOCROOT+  This is the directory where the meta data will be stored
\item 
\verb+Q_DATA_DIR+ This is the directory where the      data will be stored
\ee

\subsection{Hello World}
\begin{verbatim}
mkdir            /tmp/qtils_test
export Q_DOCROOT=/tmp/qtils_test
export Q_DATA_DIR=/tmp/qtils_test
q init # This initializes the meta data
q add_tbl t1 10 # Creates a table with 10 rows
q s_to_f t1 f1 'op=[seq]:start=[1]:incr=[1]:fldtype=[I4]'
q list_tbls # list tables
q describe t1    # describes table t1 
q describe t1 f1 # describes fld f1 in table t1 
# Creates a 4 byte integer field f1 in t1 whose values are 1, 2, ...
q pr_fld t1 f1 # Prints the values of f1 in t1 
q fop    t1 f1 sortD # Sorts f1 in t1 in descending order
q pr_fld t1 f1 # Notice that values have switched order
q delete t1    # Clean up
\end{verbatim}
\end{document}
