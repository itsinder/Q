\startreport{The Janus Project}
\reportauthor{Ramesh Subramonian}

\section{Introduction}

This document is meant to capture my understanding of what Janus is
intended to do and a proposal of how it could do it. It does seem
presumptuous to write this with barely a week of study. Therefore,
this document should be read as a strawman to be debated and not a
product of deep thought. Several assumptions will be made as the paper proceeds,
some of which are limitations of the current approach.
Relevant prior work can be found in \cite{sarawagi99}
and \cite{Fagin05}

Of the three proposals made in \cite{Fagin05} --- DIVIDE,
DIFFERENTIATE, DISCOVER --- only DIFFERENTIATE seems worth pursuing. 
I am less convinced about the value of the other two, although they do 
make for an interesting academic exercise. 
Therefore, this paper will focus on DIFFERENTIATE.

The implementation is discussed in the context of Q. We take some liberties when
presenting pseudo-code in Q --- for ease of exposition, we suppress some
details. However, doing so does not detract from their correctness.

\section{Building Blocks}

\subsection{Input Data}
Following \cite{Fagin05}, we assume that the input is a 
single relational
table where columns are either dimensions or measures. 

Each row refers to an item or a transaction or a document. 
Therefore, note that each item is associated with a single value of a
categorical dimension. For example, an item can be associated with Gender = Male
or Gender = Female but not both.
Using an example from \cite{Fagin0}'s work, if a dimension were Language which had 
values English, Chinese, Russian, then a document could be not be clsassified as
both English and Russian.


\subsubsection{Dimensions}
Dimensions are
attributes along which aggregations will be performed e.g., time, age,
gender, product category. These may be numeric, like time or
categorical, like gender.
Conversions of ``raw'' values like a datetime value to a day of the week or a
month are discussed in Section~\ref{Discretization}

\subsubsection{Measures}
Measures are numerical and are the target of aggregations. Examples are sale
price, number of items, \ldots We might want to know how average sale price
differs based on Gender. Here, Price and Number are the measures and Gender is
the dimension.

DIFFERENTIATE can be broken up into the following building blocks
\be
\item Discretization --- Section~\ref{Discretization}
\item Select A, B  --- Section~\ref{SelectAB}
\item Select Dimensions and Measures --- Section~\ref{DimsAndMeasures}
\item Core Differentiation  --- Section~\ref{Differentiation}
\item Refinement --- Section~\ref{Refinement}
\item Classification --- Section~\ref{Classification}
\ee

\subsection{Discretization}
\label{Discretization}

Given a ``raw'' attribute, we can create derived attributes. This is most often
performed for numeric dimensions, like time. This is done by providing a mapping
function, the input of which is value of the raw attribute and the output is a
value of the derived attribute e.g., input = datetime, output = day of week.

Several different discretizations of the same raw attribute can be performed.  So, timestamp could be used to create day of week, month, holiday or not, \ldots
Categorical attributes, whether raw or created by discretization, can be further
summarized. Hence, states CA, OR, WA could be mapped to the value ``West Coast''  of the derived dimension, Region.

We assume that 
\be
\item Differentiation is done using only derived categorical attributes, except
as discussed in Section~\ref{Classification}
\item the number of distinct values of a categorical 
attribute used for differentiation is small, in particular less than 255., which
allows us to represent the value in one byte. We will relax this assumption in
Section~\ref{Refinement}

\item 
We have not yet discovered any efficiencies in the case where one derived
attribute dovetails into another, such as what happens when Month 
dovetails into Quarter.
\ee

\subsection{Select A, B}
\label{SelectAB}

The aim of differentiation is to provide the user with a synopsis of
``interesting'' ways in which two subsets of the input table differ. 
More on what constitutes ``interesting'' in Section~\ref{interesting}

Note that A and B do not need to be mutually exclusive and exhaustive. 
\bi
\item not mutually exclusive --- 
A could be rows categorized as Janaury and B could be those
categorized as Male
\item not exhaustive  ---
A could be rows categorized as California and B could be those
categorized as Texas. 
\ei

We leave it to the user to define meaningful A, B sets.
Clearly, poor choices of A and B are likely to result in meaningless
answers, the infamous GIGO\footnote{Garbage In, Garbage Out} phenomenon.

The Q operators needed for selection are 
\be
\item vseq --- Section~\ref{vseq}
\item ainb --- Section~\ref{ainb}
\item vvand --- Section~\ref{vvand}
\item sum --- Section~\ref{sum}
\ee

For example, if we wanted to provide the user with the count of elements of A,
where A is defined as rows where state = California or Texas and Gender = Male,
then we would write
\begin{verbatim}
Q.sum(Q.vvand(Q.vseq(Gender, Male), Q.ainb(State, {California, Texas})))
\end{verbatim}
On a 6-core VM, the above operation takes XX seconds with 4B rows.

\subsection{Select Dimensions and Measures}
\label{DimsAndMeasures}

We assume that the user will select one or more dimensions. For example, they
may be interested in geographical dimensions but not care about how transactions
differ when aggregated by Gender. 

Similarly, we assume that the user will select one or more measures. For
example, they may care about the average
price of a transaction but not the number of transactions.

\subsection{Core Differentation}
\label{Differentation}

\subsection{Classification}
\label{Classification}




\section{Q Operators}

In this section, we list the various Q operators needed

\subsection{sum}
\label{sum}

\begin{itemize}
\item \verb+s = Q.sum(x)+ 
\item \(x\) is a numeric vector of length \(n\)
\item \(s\) is a scalar of the same type as \(x\) and \(s = \sum_i x_i\)
\end{itemize}

\subsection{vseq}
\label{vseq}

\begin{itemize}
\item \verb+y = Q.vseq(x, s)+ 
\item \(x\) is a numeric vector of length \(n\)
\item \(s\) is a scalar of the same type as \(x\)
\item \(y\) is a boolean vector of length \(n\), where \(y_i = \mathrm{true}\)
if \(x_i = s\) and false otherwise.
\end{itemize}

\subsection{where}
\label{where}

\begin{itemize}
\item \verb+z = Q.where(x, y)+ 
\item \(x\) is a numeric vector of length \(n\)
\item \(y\) is a boolean vector of length \(n\)
\item \(z\) is a numeric vector of the same type as \(x\). Consists of each
element of \(x\) for which the corresponding element of \(y\) is true. The
boundary cases are \(z = x\) when \(y\) is all true and \(z =\) {\tt nil} when
\(y\) is all false
\end{itemize}

\subsection{vvand}
\label{vvand}

\begin{itemize}
\item \verb+z = Q.vvand(x, y)+ 
\item \(x\) is a boolean vector of length \(n\)
\item \(y\) is a boolean vector of length \(n\)
\item \(z\) is a boolean vector of length \(n\) such that \(z_i = x_i \wedge z_i\)
\end{itemize}

\subsection{sumby}
\label{sumby}
\begin{itemize}
\item \verb+w = Q.sumby(x, y, n_y, z)+ 
\item \(x\) is a numeric vector of length \(n\)
\item \(y\) is a numeric vector of length \(n\)
\item \(n_y\) is a number such that 
\(0 \leq \mathrm{min}(y) \leq \mathrm{max}(y) \leq n_y-1\)
\item \(z\) is a boolean vector of length \(n\)
\item \(w\) is a numeric vector of length \(n+1\) such that 
\(w_i = \sum_{j=1}^{j=m} \delta(y_j, i) x_i\), where \(\delta(x, y) = 1 \) if
\(x = y\) and 0 otherwise.
\ei

\section{Classification}
\label{Classification}
\section{What's Interesting}
\label{interesting}

\subsection{Defining difference}

\subsubsection{Kullback-Leibler Divergence}
\label{KLDistance}

\begin{displaymath}
D_{KL}(P | Q ) = - \sum_{x \in X} P(x) \log ( \frac{Q(x)}{P(x)})
\end{displaymath}

See \url{https://en.wikipedia.org/wiki/Kullback%E2%80%93Leibler_divergence}

\subsection{What's New}
\label{WhatsNew}

\begin{quote}
Those who don't remember the past are condemned to repeat it\footnote{George
Santayana}
\end{quote}

\bibliographystyle{alpha}
\bibliography{../../DOC/Q_PAPER/ref} 

