% \usepackage{hyperref}
\startreport{Network Endorsements}
\reportauthor{Ramesh Subramonian}

\section{Introduction}

The aim of this anlaysis is to find, for a given member, 
the most common endorsements 
\be
\item being made {\bf by} the member's {\bf network}
\item being made {\bf to} the member's {\bf network}
\ee

\section{Raw Input}

\subsection{TC}
\label{TC}

\(T_C\) contains connections. It has fields
\be
\item from --- from member ID 
\item   to --- from member ID 
\ee

\subsection{TE}
\label{T_E}

\(T_E\) contains endorsements. It has fields
\be
\item from --- from member ID 
\item   to --- from member ID 
\item endorsement text
\item status --- integer, \TBC
\item t\_create --- time made (Unix epoch time in seconds)
\item t\_mod --- time accepted (Unix epoch time in seconds)
\ee


\section{Data Structures}

\subsection{TC}
\label{TC_plp}

\(T_C\) contains connections. It has fields
\be
\item from --- from member ID 
\item   to --- from member ID 
\item TC\_id --- primary key with values 0, 1, \ldots
\ee
constructed so that 
\be
\item ascending on field {\tt from} (primary)
\item ascending on field {\tt to} (secondary)
\item \(T_C[i].from \leq T_C[i+1].from\)
\item \(T_C[i].from = T_C[i+1].from \Rightarrow
T_C[i].to < T_C[i+1].to\)
\ee

\subsection{TM}
\label{T_M}

\(T_M\) contains members. It has fields
\be
\item mid --- member ID
\item \(T_{C_{lb}}\) --- foreign key into \(T_C\)
\item \(T_{C_{ub}}\) --- foreign key into \(T_C\)
\item \(T_{f_{lb}}\) --- foreign key into \(T_f\)
\item \(T_{f_{ub}}\) --- foreign key into \(T_f\)
\item \(T_{t_{lb}}\) --- foreign key into \(T_t\)
\item \(T_{t_{ub}}\) --- foreign key into \(T_t\)
\item fk\_lkp\_endo --- foreign key into lkp\_endo
\ee
constructed so that 
\be
\item \(0 \leq T_M[i].T_{C_{lb}} < T_M[i].T_{C_{ub}} \leq |T_C|\)
\item \(\forall j: T_M[i].T_{C_{lb}} \leq j < T_M[i].T_{C_{ub}}
\Rightarrow T_C[j].from = T_M[i].mid\)
\ee

\subsection{lkp\_endo}
\label{lkp_endo}

Contains endorsements 
\be
\item idx --- primary key 
\item text --- description
\ee

\section{T\_to\_uq\_endo}
\label{T_to_uq_endo}

\(T^U_t\) contains number of endorsements to a member. It has fields
\be
\item mid --- member ID
\item fk\_lkp\_endo --- endorsement (foreign key to lkp\_endo)
\item cnt --- number of times this member was endorsed for this
endorsement
\ee
constructed so that it is sorted ascending on mid (primary) and
ascending on fk\_lkp\_endo (secondary)

\section{T\_from\_uq\_endo}
\label{T_from_uq_endo}

\(T^U_f\) contains number of endorsements from a member. Similar to
Section~\ref{T_to_uq_endo}

\section{T\_to\_endo}
\label{T_to_endo}

\(T_t\) contains endorsements to a member. It has fields
\be
\item mid --- member ID
\item fk\_lkp\_endo --- endorsement (foreign key to lkp\_endo)
\item idx --- primary key
\ee
constructed so that it is sorted ascending on mid (primary) and
ascending on fk\_lkp\_endo (secondary)

\section{T\_from\_endo}
\label{T_from_endo}

\(T_f\) contains endorsements from a member. It is similar to 
Section~\ref{T_to_endo}.

\newpage

\section{UI Schematics}

In both UIs (Sections~\ref{ui_network}, ~\ref{ui_scysm}), 
nothing really gets going until we select a member (and, for
Section~\ref{ui_network}, a direction). These selections are called
the {\bf context}. We will use
\be
\item MMMM to represent the member
\item DDDD to represent the direction, can be either {\bf from} or {\bf
  to}
\item EEEE to represent the endorsement
\ee

The descriptions of the UI represent a
logical layout --- it should, in no way, be intepreted as the actual
look-and-feel.

\section{UI --- Network Activity}
\label{ui_network}


\subsection{Select Context: Member and Direction}
\label{Select_Member}

\bi
\item Member can be selected with auto-complete or with box where you can enter
\verb+member_sk+. 
\item Direction can be either received or given, defaults to received
\ei
A selection here leads to Section~\ref{Popular_Endorsements}.

\subsection{Popular Endorsements}
\label{Popular_Endorsements}

This looks like Table~\ref{tbl_popular_endorsements}. The inputs to
creating this table are (i) member (ii) direction
The first column is
clickable. Doing so takes us to Section~\ref{Popular_Endorsees}. 
API is shown below.
\begin{verbatim}
curl --url http:172.18.42.75:8080/dp/endo1.php \
       	--data-urlencode mid='MMMM' --data-urlencode dirn='DDDD'
\end{verbatim}

\begin{table}[hb]
\centering
\begin{tabular}{|l|l|l|} \hline \hline
{\bf Endorsement} & {\bf Number Received or Given} & {\bf Num Members}
\\ \hline \hline
\url{endorsement 1} & \(n^R_1\) & \(m^R_1\) \\ \hline 
\url{endorsement 2} & \(n^R_2\) & \(m^R_2\) \\ \hline
--- & --- & --- \\ \hline
\hline
\end{tabular}
\caption{Endorsements received and given (sorted descending on number
    receieved)}
\label{tbl_popular_endorsements}
\end{table}

\subsection{Popular Endorsees}
\label{Popular_Endorsees}

This looks like Table~\ref{tbl_popular_endorsees}. The inputs to
creating this table are (i) member (the context) (ii) direction and (iii) an
endorsement. 
It contains members in the network of the context member who 
received (or were given) a particular endorsement. It is sorted descending 
on number receieved (or given).
The first column is
clickable. Doing so takes us to Section~\ref{Select_Member}
API is shown below.
\begin{verbatim}
curl --url http:172.18.42.75:8080/dp/endo1.php \
       	--data-urlencode mid='MMMM' --data-urlencode dirn='DDDD' \
       	--data-urlencode endo='EEEE' 
\end{verbatim}


\begin{table}[hb]
\centering
\begin{tabular}{|l|l|} \hline \hline
{\bf Member} & {\bf Number Received or Given} \\ \hline \hline
\url{member 1} & \(n_1\) \\ \hline 
\url{member 2} & \(n_2\) \\ \hline 
--- & --- \\ \hline
\hline
\end{tabular}
\caption{Members who received (or were given) a particular endorsement (sorted descending on number)}
\label{tbl_popular_endorsees}
\end{table}


\section{UI --- SCYSM }
\label{ui_scysm}


Starts with Section~\ref{Select_Member}. Selecting a member takes us to
Section~\ref{Suggested_Introductions}

\subsection{Suggested Introductions}
\label{Suggested_Introductions}

This looks like Table~\ref{tbl_suggested_introductions}. 
Clicking on either column 1 or 2 takes us to Section~\ref{Select_Member}.
Clicking on column 3 takes us to Section~\ref{Common_Endorsements}.

\begin{table}[hb]
\centering
\begin{tabular}{|l|l|l|} \hline \hline
{\bf Member 1} & {\bf Member 2} & {\bf Why? } \\ \hline
\hline
\url{member 11} & \url{member 21} & \url{reason 1} \\ \hline
\url{member 12} & \url{member 22} & \url{reason 2} \\ \hline
--- & --- & --- \\ \hline
\hline
\end{tabular}
\caption{Members who should be introduced}
\label{tbl_suggested_introductions}
\end{table}


\subsection{Common Endorsements}
\label{Common_Endorsements}

Clicking on either column 1 to Section~\ref{Popular_Endorsees}.

\begin{table}[hb]
\centering
\begin{tabular}{|l|l|l|} \hline \hline
{\bf Endorsement} & {\bf Count for \url{Member 1}} & {\bf Count for
  \url{Member 2}} \\ \hline \hline 
\url{Endorsement 1} & \(n_{1,1}\) & \(n_{1,2}\) \\ \hline 
\url{Endorsement 2} & \(n_{2,1}\) & \(n_{2,2}\) \\ \hline 
--- & --- & --- \\ \hline
\hline
\end{tabular}
\caption{Endorsements common to member 1 and member 2}
\label{tbl_common_endorsements}
\end{table}

