\startreport{Rough k-means in Q}
\reportauthor{Ramesh Subramonian}

\section{Introduction}

We describe the rough k-means algorithm \cite{Peters2012}. 
We assume that the reader is familiar with the \(k\)-means algorithm
\cite{Hastie2009}. We make a minor change to their terminology, 
refering to the 
\bi
\item lower approximation as the ``inner'' approximation
\item upper approximation as the ``outer'' approximation
\ei

\subsection{Notation}

\be
\item Let \(n_J\) be the number of features. 
We shall use \(j\) as the feature index. 
\item Let \(n_I\) be the number of objects/instances/observations
We shall use \(i\) as the instance index. 
\item Let \(n_K\) be the number of means/centroids.
We shall use \(k\) as the centroid index. 
\item Let \(X\) be a set of observations in \(n_J\) dimensional space, 
such that \(X_{j, i}\) is the value of the \(j^{th}\) feature of 
the \(i^{th}\) instance. 
\item Let \(\mu_{k, j}\) be the value of the \(j^{th}\) feature of the \(k^{th}\)
centroid
\item Let \(I_{k, i}\) be true if instance \(i\) is part of the 
{\bf I}nner approximation of centroid \(k\)
\item Let \(O_{k, i}\) be true if instance \(i\) is part of the 
{\bf O}uter approximation of centroid \(k\)
\item We shall treat the boolean value ``true'' and the integer
1 interchangeably. 
\item We shall treat the boolean value ``false'' and the integer
0 interchangeably. 
\item Let \(w_I\) and \(w_O\) be the weights assigned to the Inner and Outer
approximations
\item Let \(\alpha\) be the threshold for determining whether an instance
belongs to the outer approximation. This is explained in 
Section~\ref{update}
\item Identifying \(i, j, k\) as feature indexes simplifies the notation. For
example, \(\sum_k\) is actually \(\sum_{k=1}^{k=n_K}\). Similarly, \(\forall k\)
means for all centroids, numbered \(1, \ldots, n_K\)
\ee

\subsection{Invariants}

\begin{invariant}
An instance can belong to the inner approximation of at most one centroid.

\(\forall i: \sum_k I_{k, i} = 1\)
\end{invariant}

\begin{invariant}
If an instance is not part of any inner approximation, it must belong to 
two or more outer approximations. This implies that an instance cannot belong to only a single boundary region.

\(\sum_k I_{k, i} = 0 \Rightarrow \sum_k O_{k, i} \geq 2\)

\end{invariant}


\subsection{Mathematics to Code}
\label{math_to_code}
The mathematical terms we use are terse, the variable names in the code somewhat
more verbose. A mapping is provided in Table~\ref{tbl_mapping}.
\begin{table}[htbp]
\centering
\begin{tabular}{|l|l|l|} \hline
{\bf Math}     & {\bf Code}      \\ \hline
\(d\) & dist \\ \hline
\(\bar{d}\) & best\_dist \\ \hline
\(\bar{k}\) & best\_clss \\ \hline
\(O\) & is\_outer \\ \hline
\(n^O\) & num\_in\_outer \\ \hline
\(I\) & inner \\ \hline
\hline
\end{tabular}
\caption{Math symbols to names in code}
\label{tbl_mapping}
\end{table}


\section{Computation}

\subsection{The Update Step}
\label{update}

\be
\item \(\forall k: d_k \) is a Vector such that \(d_{k, i}\)
is distance of instance \(i\) from centroid \(k\)
\item \(\bar{d}\) is a Vector such that \(\bar{d}_i\) is
smallest distance of instance \(i\) from any centroid i.e.,
\(\bar{d}_i = \min_k d_{k, i}\)
\item \(\bar{k}\) is a Vector such that \(\bar{k}_i\) identifies
the centroid that is closest to instance \(i\). Note that 
\(\bar{k}_i \in [1, \ldots n_K]\)
\item \(\forall k: O_k\) is a boolean Vector of length \(n_I\) such that 
\(d_{k, i} \leq \bar{d}_i \times \alpha\)
\item \(N^O\) is a Vector of length \(n_I\), 
where \(N^O_{k, i}  = \sum_k O_k[i]\)
\item Let \(\hat{I}\) be a Vector of length \(n_I\) such that 
\(O_i = \mathrm{true} \Rightarrow \hat{I}_i = 0\); else, \(\hat{I}_i = \bar{k}_i\). What
we are doing here is stating that if nobody else has a claim on instance \(i\), then
it belongs to the inner approximation of \(\bar{k}_i\). In other words,
\be
\item \(\hat{I}_i = 0 \Rightarrow \) instance \(i\) not in inner approximation
of any centroid
\item \(\hat{I}_i = k' \Rightarrow \) instance \(i\) in inner approximation
of centroid \(k'\)
\ee
\ee

\subsection{The Assignment Step}
\label{assignment}


\bibliographystyle{alpha}
\bibliography{../../DT/doc/ref}
