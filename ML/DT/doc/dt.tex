\startreport{Gambling on Decision Trees}
\reportauthor{Ramesh Subramonian}

\section{Introduction}

Decision trees are a well understood statistical technique \cite{Hastie2009}.
This paper makes the following contributions:
\be
\item proposes a gambling-inspired metric that allows us to answer
the two fundamental questions associated with building a decision tree:
\be
\item what is the decision to be made at a given node e.g., \(x \leq 0.5?\)
\item when should a node not be split further?
\ee
This is in contrast to the well known metrics (i) gini impurity (ii) information
gain and (iii) variance reduction
\item a more intuitive way to deal with unknown values, both during 
training and testing. This approach is particularly useful when
the fact that the value is null does not have some systemic bias.
Section~\ref{unknown}
\item what ``decision'' does the decision tree really make? For example, if the
evaluation of the tree leaves us at a node which had 
\(n_H\) instances of class ``Heads'' and 
\(n_T\) instances of class ``Tails'' at training time, what statement are we
making?
\item when using several decision trees to create a random forest, using
principal component analysis to prune correlated trees
\item weighting the results of the different decision trees based 
on their accuracy, instead of simply averaging the results.
\ee

We believe that these changes to the way decision trees are created and their
results combined is particularly relevant to the case where relative
confidences are involved. This is often the case when multiple candidates are
competing for a limited resource, like screen real estate or user attention. In
such cases, we need each classifier to predict (with as much accuracy as
possible) the likelihood that a particular event will occur.

\subsection{Conventions}
Throughout the paper, we use the data sets listed in Table~\ref{tbl_data_sets}
for our experiments. 
\begin{table}
\centering
\begin{tabular}{|l|l|l|l|} \hline \hline
{\bf Data Set Name } & {\bf Rows} & {\bf Columns} & {\bf Location} \\ \hline 
Breast Cancer & X & X & \\ \hline
Titanit & X & X & \\ \hline
\ldots & X & X & \\ \hline
\end{tabular}
\label{tbl_data_sets}
\caption{Data Sets used for experimentation}
\end{table}

\subsection{Notations}

\bi
\item Let \(F = \{f_i\}\) be a table of vectors, representing the features.
\item Let \(g\) be a vector, representing the outcome which we wish to
predict. We will refer to the value 1 as heads and 0 as tails.
\item Restrictions on vectors (for a Q implementation)
\be
\item lengths of all vectors must be the same. 
\item all feature vectors should be either F4, I4 or SV
\item goal vector should be I4
\ee
\ei

A decision tree is a Lua table where each element identifies
\be
\item a feature
\item the comparison to be made. 
\be 
\item If the feature is ordered (I4 or F4), then the
comparision is always \(\leq\) and a threshold needs to be provided e.g., age
\(\leq 45\) 
\item If the feature is unordered (SV), then the values on the left branch need
to be provided e.g., State is one of California, Washington, Oregon; the values
on the right branch being all other values that the feature can assume.

\ee
\item a left decision tree
\item a right decision tree
\ee

\begin{figure}
\centering
\fbox{
\begin{minipage}{35cm}
\begin{tabbing} \hspace*{0.25in} \=  \hspace*{0.25in} \=
                \hspace*{0.25in} \=  \hspace*{0.25in} \= \kill
Let \(\alpha\) be minimum benefit required to continue branching \\
Let values taken on by goal be 0 (tails), 1 (heads)  \\
Let \(n_G = 2\) be the number of values of the goal attribute \\
Let \(n_T =\) number of instances classified as tails \\
Let \(n_H =\) number of instances classified as heads \\
Let \(T\) be table of vectors, each vector being an attribute \\
Initialize, decision tree \(D = \{\}\) \\
\(F, g\) as described above \\
{\bf function } MakeDecisionTree(D, T, g) \+ \ \\
  \(C = Q.\mathrm{numby}(g, n_G);~n_T = C[0];~n_H = C[1]\) \\
  {\bf forall} \(f \in T:~ b(f), s(f) = \mathrm{Benefit}(f, g, n_T, n_H)\) \\
  Let \(f'\) be feature with maximum benefit. \\
  {\bf if} \(b(f') < \alpha\) {\bf then } \+  \\
    {\bf return} \- \\
  {\bf endif} \\
    \(x = \mathrm{Q.vsgt}(f', s(f'))\) \\
    \(F_L = F_R = \{\}\) \\
    {\bf forall} \(f \in F\) {\bf do} \+ \\
      \(F_L = F_L \cup \mathrm{Q.where}(f, x)\) \\ 
      \(F_R = F_L \cup \mathrm{Q.where}(f, \mathrm{Q.not}(x))\) \\ 
    {\bf endfor} \\
    T.feature = \(f'\) \\
    T.threshold = \(b(f')\) \\
    T.left = \(\{\}\) \\
    T.right = \(\{\}\) \\
    \(DT(F_L, g_L, T_L)\) \\
    \(DT(F_R, g_R, T_R)\) \- \\
{\bf end} 
\end{tabbing}
\end{minipage}
}
\label{dt_pseudo_code}
\caption{Decision Tree algorithm}
\end{figure}

\begin{figure}
\centering
\fbox{
\begin{minipage}{15cm}
\begin{tabbing} \hspace*{0.25in} \=  \hspace*{0.25in} \= 
                \hspace*{0.25in} \=  \hspace*{0.25in} \= \kill
{\bf function } \(\mathrm{Benefit}(f, g, n_T, n_H)\) \+  \\
  \(f', h_0, h_1 = CountIntervals(f, g)\) \\
  \(b = \mathrm{Q.wtbnft}(h_0, h_1, n_T, n_H)\) (Figure~\ref{algo_weighted_benefit}) \\
  \(b', \_, i = \mathrm{Q.max}(b)\) \\
  {\bf return} \(b', f'[i]\)  \- \\
{\bf end} 
\end{tabbing}
\end{minipage}
}
\label{compute_benefit_numeric}
\caption{Benefit Computation (numeric attributes)}
\end{figure}
%%-------------------------------------------
\begin{figure}
\centering
\fbox{
\begin{minipage}{15cm}
\begin{tabbing} \hspace*{0.25in} \=  \hspace*{0.25in} \= \kill
{\bf function } \(\mathrm{WeightedBenefit}(n_T^L, n_H^L, n_T^R, n_H^R)\) \+  \\
  \(n_L = n_T^L + n_H^L \) --- number on left \\
  \(n_R = n_T^R + n_H^R \) --- number on right \\

  \(n = n_T + n_H \) --- total number \\

  \(w_L = \frac{n^L}{n}\)  --- weight on left \\
  \(w_R = \frac{n^R}{n}\)  --- weight on left \\

  \(o_H^L = \frac{n_T^L}{n_H^L}\) --- odds for heads on left \\
  \(o_T^L = \frac{n_H^L}{n_T^L}\) --- odds for tails on left \\
  \(o_H^R = \frac{n_T^R}{n_H^R}\) --- odds for heads on right \\
  \(o_T^R = \frac{n_H^R}{n_T^R}\) --- odds for tails on tails \\

  \(b_H^L = o_H^L \times \frac{n_H^L}{n_L} +
            (-1)  \times \frac{n_T^L}{n_L}\) 
            --- benefit of betting heads on left \\
  \(b_T^L = o_T^L \times \frac{n_T^L}{n_L} +
            (-1)  \times \frac{n_H^L}{n_L}\) 
            --- benefit of betting tails on left \\

  \(b_H^R = o_H^R \times \frac{n_H^R}{n_L} +
            (-1)  \times \frac{n_T^R}{n_L}\)
            --- benefit of betting heads on right \\
  \(b_T^R = o_T^R \times \frac{n_T^R}{n_L} +
            (-1)  \times \frac{n_H^R}{n_L}\)
            --- benefit of betting tails on right \\

  \(b_L = \mathrm{max}(b_H^L, b_T^L)\) --- benefit on left \\
  \(b_R = \mathrm{max}(b_H^R, b_T^R)\) --- benefit on right \\

  \(b = b_L \times w_L + b_R \times w_R\) --- weighted benefit \\
  {\bf return} \(b\) \- \\
{\bf end} 
\end{tabbing}
\end{minipage}
}
\label{algo_weighted_benefit}
\caption{Weighted Benefit Computation}
\end{figure}
%%-------------------------------------------
\begin{figure}
\centering
\fbox{
\begin{minipage}{15cm}
\begin{tabbing} \hspace*{0.25in} \=  \hspace*{0.25in} \= \kill
{\bf function } \(\mathrm{CountIntervals}(f, g)\) \+  \\
{\bf Inputs} \+ \\  
Vector \(f\) of length \(n\) type integer or floating point \\
Vector \(g\) of length \(n\) type integer with values 0, 1 \- \\
{\bf Outputs} \+ \\
Vector \(f'\) of length \(n'\) type integer \\
Vector \(h_0\) of length \(n'\) type integer  \\
Vector \(h_1\) of length \(n'\) type integer  \- \\
\(f'\) is the unique values of \(f\), sorted ascending \\
Let \(\delta(x)\) = 1 if \(x\) is true and 0 otherwise \\
\({h_k}_j = \sum_{i=1}^{i=n} \delta(g_i = k \wedge f_i \leq f_j)\) \- \\
{\bf end} 
\end{tabbing}
\end{minipage}
}
\label{count_intervals}
\caption{Count Intervals}
\end{figure}
%%-------------------------------------------



\begin{figure}
\centering
\fbox{
\begin{minipage}{15cm}
\begin{tabbing} \hspace*{0.25in} \=  \hspace*{0.25in} \= \kill
{\bf function } \(\mathrm{Benefit}(f, g, n_T, n_H)\) \+ 

{\bf end} 
\end{tabbing}
\end{minipage}
}
\label{compute_benefit_categorical}
\caption{Benefit Computation (categorical attributes)}
\end{figure}

\section{Algorithm}

Since decision trees have been well documented, we focus our attention on the
salient characteristics of our approach which are
\be
\item how does one decide on the comparison to be performed at a node?
\item how does one decide when to not split a node further, assuming that asplit
is possible at all.
\ee

Given the simple recursive nature of decision trees, we focus our 
decision on what is done at a single node. Without loss of generality, 
let this be the root node. At this point, we know the number of elements
classified as Heads and Tails, \(n_H, n_T\) respectively. This allows us to set
the odds of betting on heads or tails. At each node, we are asked to a single
question. 
\bi
\item For a numeric attribute, it is of the form \(x \leq y\)
\item For a boolean attribute, it is of the form \(x = y\), where y is true or
false
\item For an unordered categorical attribute that can take on values in \(X\),
it is of the form \(x \in y \subset X\)
\ei
We iterate over all possible values of \(y\) for each attribute and find the
attribute and decision that produces the maximum benefit. If this benefit
exceeds a pre-defined threshold, we split the data into two, a left branch and a
right branch and proceed recursively. Else, the expansion of the decision tree
stops at this point and control returns to the parent node. 
Figure~\ref{XXX} describes the algorithm at this level. For clarity of 
exposition, we only describe the treatment of numeric attributes.


\subsection{Useful subroutines}
We list a number of subroutines used to describe the algorthms
\be
\item Q.not(x). \(x\) is a boolean Vector. output is y, a boolean vector, the
negation of \(x\)
\item Q.vsgt(x, s). \(x\) is a Vector, \(s\) is a scalar. Output is \(y\) a
boolean vector such that \(x_i \geq s \Rightarrow y_i = \mathrm{true}\) and
\(y_i\) is false otherwise
\item Q.sum(x). \(x\) is a Vector. output is a scalar
\item Q.numby(x, n). \(x\) is a Vector, \(n\) is a positive integer 
such that \(x_i \in [0, n-1]\). output is a vector \(y\) such that \(y_i\) is
the number of elements of \(x\) that have value \(i\)
\ee

\subsection{Benefit Computation}
It is useful to model this problem as a gambling problem where we are 
sampling with replacement. 
\bi
\item Let there be 100 balls, with 60  being marked Heads and 40
being marked Tails. 
\item Assume that we have to gamble on whether a ball picked at random is Heads
ot Tails. If the ``house'' sets odds to ensure a fair game, then it would offer
a payout of 4/6 if you bet heads and 6/4 if you bet tails. 
\item If you bet heads, then the expected benefit is 
\(\frac{60}{100} \times \frac{4}{6} + 
\frac{40}{100} \times (-1)  = 0\)
\item If you bet tails, then the expected benefit is 
\(\frac{60}{100} \times (-1) + 
\frac{40}{100} \times \frac{6}{4}  = 0\)
\ei

Now, assume that before we place out bet, we are allowed to ask one questionm
which has an answer of left or right. If left, we are told that the ball was
selected from a population of 40 heads and 15 tails. If right, we are told that
the ball was selected from a population of 20 heads and 25 tails. Clearly, we
would bet heads if the answer is left and tails if the answer is right. The
question is: What is the expected benefit for 100 trials?

\(100 ( \times ( \frac{40+15}{100} \times ( 
  \frac{40}{40+15} \times \frac{4}{6}  + 
  \frac{15}{40+15} \times -1 ) ) + 
  ( \frac{20+25}{100} \times ( 
  \frac{20}{20+25} \times -1 + 
  \frac{25}{20+25} \times \frac{6}{4}  ) ) )
  \)
which is \(
 = (40 \times  \frac{4}{6}) -15 - 20 + (25 \times \frac{6}{4} )
 = 29 \frac{1}{6}\)

\subsubsection{Categorical Attributes}
We deal with categorical attributes as follows. Let \(X\) be a categorical
attribute that takes on values \(\{x_1, x_2, \ldots x_M\}\), where \(M\) is
typically small. We can assume less than 100 values.

Let \(Y = \{(x_i, g_i)\}\) be the data set, where \(i = 1, \ldots N\). 
By this we ean
that the \(i^{th}\) point has value \(x_i \in X\) and its goal attribute has
value \(g_i \in \{0, 1\}\). 
\(Y\) may have other features but for now we are concentrating on \(X\).

Create a table, \(Z\), of vectors \(N, P, \rho\) of length \(M\) as follows.  
\begin{enumerate}
\item 
\(n_j = \sum_{i=1}^{i=N} z(i, j)\) where 
\(z_j = 1 \) if \(x_i = x_j \) and \(g_i = 0\)
\item 
\(p_j = \sum_{i=1}^{i=N} z(i, j)\) where 
\(z_j = 1 \) if \(x_i = x_j \) and \(g_i = 1\)
\item 
\(rho_j = \frac{p_j}{p_j+n_j}\)
\end{enumerate}
Sort \(Z\) on \(rho\). Create vectors, \(N', P'\) of length \(M\) as follows
\be
\item 
\ee

\newpage
\section{Evaluating a decision tree}
In addition to the standard metrics, such as classification accuracy, we
evaluate decision trees as follows. Every element of the data set is 
fed into the decision tree, thereby
assigning each one to a leaf. Now, for each leaf, we have four numbers.
\be
\item \(n^H_1, n^T_1\) which is the number of testing samples assigned 
to the leaf that were labeled heads and tails respectively. The ``weight'' of a
leaf is \(n^H_1 + n^T_1\)
\item \(n^H_0, n^T_0\) same as above but for the training samples
\ee

These numbers are used follows. Continuing our gambling metaphor,
the house is given only \(n^H_0, n^T_0\) to set the odds, whereas the gambler
has access to all four.
The ``true'' probability of heads, \(\rho\), is assumed to be 
\(\frac{n^H_1}{n^H_1+ n^T_1}\). 

Consider three examples, in all of which \(n^H_0, n^T_0 = 60, 40\). 
The benefit, \(b\) that the gambler can expect to make per trial varies
depending on \(rho\) as follows.
\be
\item \(\rho = 0.8\). The adversary would bet heads.
\(b = 0.8 \times \frac{40}{60} + 0.2 \times -1 = \frac{1}{3}\)
\item \(\rho = 0.6\). If the adversary bet heads, 
\(b = 0.6 \times \frac{40}{60} + 0.4 \times -1 = 0\)
If the adversary bet tails, 
\(b = 0.6 \times -1 + 0.4 \times \frac{60}{40} = 0\).
\item \(\rho = 0.4\). The adversary would bet tails.
\(b = 0.4 \times -1 + 0.6 \times \frac{60}{40} = \frac{1}{2}\) 
\ee

The weighted benefit of a leaf is simply \(w \times b\).
The ``cost'' of the decision tree is sum over the 
weighted benefit, normalized by the weights of all leaves 
\(= \frac{\sum_l w_l \times b_l}{\sum_l w_l}\)

\subsection{When not to split?}

In Figure~\ref{XXX}, we stated that we stop splitting when the weighted benefit
is less than a threshold of \(\alpha\). We are now in a position to describe how
\(\alpha\) is determined. We search the space of possible values of \(\alpha =
(0, 1)\), evaluating the ``cost'' for each \(\alpha\). A sample on the breast
cancer data set (reference TODO) is in
Figure~\ref{fig_breast_cancer_cost_versus_alpha}.

We pick the \(alpha\) which produces the minimum cost. For a variety of data
sets, the shape of the curve is similar to 
Figure~\ref{fig_breast_cancer_cost_versus_alpha} but we have not been able to
establish an analytical proof that the function is U-shaped.

\section{Unknown Values}
\label{unknowns}

\begin{quote}
When you come to a fork in the road, take it \footnote{attributed to Yogi Berra}
\end{quote}
Assume that we have \(X\) instances of which the fraction which have class
``Heads'' is \(p_H\) and Tails is \(p_T\).
Assume that we are considering splitting using
a feature \(f\) such that \(X' \subseteq X\) of the instances have unknown
values for this feature. 
the form \(x \leq y\). 

In terms of the training process, unknown values affect the process 
as follows.  With each instance \(x_i \in X\) is associated a weight \(w_i\),
initialized to 1. Let \(X'\) be the instances where the value of the splitting feature is unknown.

\be
\item Benefit benefit calculation of Section~\ref{XXX}. 
For all \(x_i \in X'\), we treat it as if it were two instances, one with a class of Heads and a weight of \(w_i \times p_H\) and the other with a class of Tails and a weight of \(w_i \times p_T\)
\item Partitioning the subsets of the data passed to the left child and right child.  All elements of \(X'\) are passed to the left child with a weight of \(w_i \times p_H\) and to the right child with a weight of \(w_i \times p_T\).  See Figure~\ref{fig_unknown_values}.
\ee

In terms of the testing process, the change in evaluation is as follows.
Hitherto, a test instance given to a decision tree ends up at precisely one leaf
node. The proportion of heads and tails at that leaf is used to make a
prediction, the simplest being that \(p_H = \frac{n_H}{n_H+n_T}\). Now, if
decision requires knowing the value of a feature that is unknown, we send the
instance down to both left and right childs and combine the resulting decision,
weighted by the proportions of heads and tails at that node. In
Figure~\ref{fig_unknown_accuracy}, we show how the accuracy of classification
falls off as the proportion of unknown values in the data set increases. We
change the proportion of unknown values by randomly selecting instances and
randomly selecting features to artificially turn to null values. The results are
reported for the data sets of Table~\ref{tbl_data_sets}.


\begin{figure}
\centering
\fbox{
\begin{minipage}{35cm}
TODO 
\end{minipage}
}
\label{fig_unknown_values}
\caption{Change in weight of datum when split involves unknown values}
\end{figure}



\bibliographystyle{alpha}
\bibliography{../../../DOC/ref.bib}
