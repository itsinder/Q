\section{Unknown Values}
\label{unknowns}

\begin{quote}
When you come to a fork in the road, take it \footnote{attributed to Yogi Berra}
\end{quote}
Assume that we have \(X\) instances of which the fraction which have class
``Heads'' is \(p_H\) and Tails is \(p_T\).
Assume that we are considering splitting using
a feature \(f\) such that \(X' \subseteq X\) of the instances have unknown
values for this feature. 
the form \(x \leq y\). 

In terms of the training process, unknown values affect the process 
as follows.  With each instance \(x_i \in X\) is associated a weight \(w_i\),
initialized to 1. Let \(X'\) be the instances where the value of the splitting feature is unknown.

\be
\item Benefit benefit calculation of Section~\ref{XXX}. 
For all \(x_i \in X'\), we treat it as if it were two instances, one with a class of Heads and a weight of \(w_i \times p_H\) and the other with a class of Tails and a weight of \(w_i \times p_T\)
\item Partitioning the subsets of the data passed to the left child and right child.  All elements of \(X'\) are passed to the left child with a weight of \(w_i \times p_H\) and to the right child with a weight of \(w_i \times p_T\).  See Figure~\ref{fig_unknown_values}.
\ee

In terms of the testing process, the change in evaluation is as follows.
Hitherto, a test instance given to a decision tree ends up at precisely one leaf
node. The proportion of heads and tails at that leaf is used to make a
prediction, the simplest being that \(p_H = \frac{n_H}{n_H+n_T}\). Now, if
decision requires knowing the value of a feature that is unknown, we send the
instance down to both left and right childs and combine the resulting decision,
weighted by the proportions of heads and tails at that node. In
Figure~\ref{fig_unknown_accuracy}, we show how the accuracy of classification
falls off as the proportion of unknown values in the data set increases. We
change the proportion of unknown values by randomly selecting instances and
randomly selecting features to artificially turn to null values. The results are
reported for the data sets of Table~\ref{tbl_data_sets}.


\begin{figure}
\centering
\fbox{
\begin{minipage}{35cm}
TODO 
\end{minipage}
}
\label{fig_unknown_values}
\caption{Change in weight of datum when split involves unknown values}
\end{figure}



