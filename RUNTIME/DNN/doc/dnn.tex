\startreport{Dense Neural Networks}
\reportauthor{Ramesh Subramonian}

\section{Computational Considerations}
We start by describing the dense neural network (DNN) computation. In
Section~\ref{the_math}, we will discuss the underyling mathematics. 
\subsection{Data Structure of DNN}
\label{data_struct}
We start by presenting the data structure of the DNN
\begin{figure}
\centering
% TODO: Following should be auto-generated not hard-coded
\verbatiminput{dnn_types.h}
\end{figure}

We now explain each element of the 
\be
\item nl --- number of layers. The simplest DNN has 3 layers --- 
an input layer, a hidden layer and an output layer.
\item npl[i] --- number of neurons per layer. The network of
Figure~\ref{sample_network} would have nl = 3, npl = \({3,4,1}\)
\item W --- the weights on the edges. 
\(W[i]\) contains the edges from layer \(i-1\) to layer \(i\). 
Hence, 
\be
\item \(W[0] = \bot\)
\item \(W[i] = \) edges from layer \(i-1\) to layer \(i\)
\item \(W_i[j] = \) edges from neuron \(j\) of layer \(i-1\) 
\item \(W_{i_j}[k] = \) edges from neuron \(j\) to neuron \(k\) of layer \(i-1\) 
\ee
\item b --- the bias of the neurons
\(b[i]\) contains the bias of neurons in layer \(i\). 
Hence, 
\be
\item \(b[0] = \bot\)
\item \(b[i] = \) biases of neurons in layer \(i\)
\item \(b_i[j] = \) bias of neuron \(j\) of layer \(i-1\) 
\ee
\item z --- the intermediate output of a neuron
\(z[i]\) contains the intermediate output of neurons in layer \(i\). 
Hence, 
\be
\item \(z[0] = \bot\)
\item \(z[i] = \) intermediate output of neurons in layer \(i\)
\item \(z_i[j] = \) intermediate output of neuron \(j\) of layer \(i-1\) 
\ee
\item a --- the output output of a neuron --- after intermediate output has been
passed through activation function.
\(a[i]\) contains the output of neurons in layer \(i\). 
Hence, 
\be
\item \(a[0] = \bot\)
\item \(a[i] = \) output of neurons in layer \(i\)
\item \(a_i[j] = \) output of neuron \(j\) of layer \(i-1\) 
\ee
\ee

\section{The Math}

\subsection{Dropouts}
Dropouts are explained in \cite{Srivastava14}.

\TBC
\newpage
\bibliographystyle{alpha}
\bibliography{../../../DOC/Q_PAPER/ref}
