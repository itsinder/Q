\startreport{Dense Neural Networks}
\reportauthor{Ramesh Subramonian}

\section{Computational Considerations}
We start by describing the dense neural network (DNN) computation. In
Section~\ref{the_math}, we will discuss the underyling mathematics. 
\subsection{Data Structure of DNN}
\label{data_struct}
We start by presenting the data structure of the DNN
\begin{figure}
\centering
% TODO: Following should be auto-generated not hard-coded
\verbatiminput{dnn_types.h}
\end{figure}

We now explain each element of the 
\be
\item \(n^l\) --- {\tt nl}, number of layers. The simplest DNN has 3 layers --- 
an input layer, a hidden layer and an output layer.
\item \(p\) -- {\tt npl}, number of neurons per layer. \(p_l\) is number of
neurons in layer \(l\), where \(0 \leq l < n^l\). The network of
Figure~\ref{sample_network} would have \(n^l = 3, p = \{3,4,1\}\)
\item \(W\) --- the weights on the edges. 
\(W_i\) contains the edges from layer \(i-1\) to layer \(i\). 
Hence, 
\be
\item \(W_0 = \bot\)
\item \(W_i = \) edges from layer \(i-1\) to layer \(i\). Note that 
\(0 < i < n^l\)
\item \(W_{i,j} = \) edges from neuron \(j\) of layer \(i-1\) to layer \(i\).
Note that \( 0 \leq < p_{i-1}\).
\item \(W_{i,j,k} = \) edge from neuron \(j\) of layer \(i-1\) to 
neuron \(k\) of layer \(i\).
Note that \(0 \leq k < p_i\).
\ee
\item \(b\) --- the bias of the neurons.
\(b_i\) contains the bias of neurons in layer \(i\). 
Note that \(0 < i < n^l\).
Hence, 
\be
\item \(b_0 = \bot\)
\item \(b_i = \) biases of neurons in layer \(i\)
\item \(b_{i,j} = \) bias of neuron \(j\) of layer \(i\), 
where \(0 \leq j < p_i\)
\ee
\item \(z\) --- the intermediate output of a neuron
\(z_i\) contains the intermediate output of neurons in layer \(i\). 
Hence, 
\be
\item \(z_0 = \bot\)
\item \(z_i = \) intermediate output of neurons in layer \(i\)
Note that \(0 < i < n^l\).
\item \(z_{i,j} = \) intermediate output of neuron \(j\) of layer \(i-1\) 
\ee
\item \(a\) --- the output output of a neuron --- after intermediate output has been
passed through activation function.
\(a_i\) contains the output of neurons in layer \(i\). 
Hence, 
\be
\item \(a_0 = \bot\)
\item \(a_i = \) output of neurons in layer \(i\)
Note that \(0 < i < n^l\).
\item \(a_{i,j} = \) output of neuron \(j\) of layer \(i-1\) 
\ee
The observant reader would have noticed that the above description indicates
that \(z, a\) are 2-dimensional arrays whereas they have been defined as {\tt
float ***}. This is because the network is evaluated in batches, explained in
Section~\ref{batching}
\ee

\subsection{Batching}
What is batching? Assume that we have \(n\) instances and a batch size of \(m <
n\). Then, it means that an epoch consists of performing the forward pass 
for \(m\) instances, then the back propagation, then doing the same for the next
\(m\) instances and so on.

This makes \(z, a\) 3-dimensional arrays such that 
\(a_{i,j, k} = \) output of neuron \(j\) of layer \(i-1\) for \(k^{th}\)
instance.

\section{The Math}

\subsection{Dropouts}
Dropouts are explained in \cite{Srivastava14}.

\TBC
\newpage
\bibliographystyle{alpha}
\bibliography{../../../DOC/Q_PAPER/ref}
