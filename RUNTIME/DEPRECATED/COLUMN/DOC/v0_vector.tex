\startreport{Q's Vector Data Type}
\reportauthor{Ramesh Subramonian}
\newcommand{\Assign}{\verb+:=+ }

\section{Introduction}

For now, I am being blase in mixing Lua and C but this is pseudo-code 
so I will take the liberty of doing so.

\TBC


\section{Data storage and access}

Q supports only one data structure, a vector. All elements of the
vector must be of the same type,  Section~\ref, {q_types}. Note that all
elements of a field have the same length. 

A vector is best thought of as an object. Lua provides basic but
entirely servicable object oriented
support \url{https://www.lua.org/pil/16.html}. For something to be a
vector, it must satisfy the following contractual obligations,
Section~\ref{vector_contract}.

\subsection{Q Types}
\label{q_types}
Q currently supports the types listed in Table~\ref{tbl_q_types}

\begin{table}[hb]
\centering
\begin{tabular}{|l|l|l|l|l|} \hline \hline
  {\bf Name} & {\bf Input Type} & {\bf Output Type} & {\bf Return Value} \\ \hline \hline
  logit & Vector \(x\) & Vector \(y\) & \(y = \frac{e^x}{1 + 1^x}\) \\ \hline
  constant & Scalar \(x\) & Vector \(y\) & \(\forall i:~y_i = x \) \\ \hline
  vvadd & Vector \(x\), Vector \(y\) & Vector \(z\) & \(z = x + y \)  \\ \hline
  vvsub & Vector \(x\), Vector \(y\) & Vector \(z\) & \(z = x - y \)  \\ \hline
  vvmul & Vector \(x\), Vector \(y\) & Vector \(z\) & \(z = x \times y \)  \\ \hline
  vvdiv & Vector \(x\), Vector \(y\) & Vector \(z\) & \(z = x / y \)  \\ \hline
  vsmul & Vector \(x\), Scalar \(y\) & Vector \(z\) & \(\forall i:~z_i = x_ \times y \)  \\ \hline
  sumprod & Vector \(x\), Vector \(y\) & Scalar \(z\) & \(z = \sum_i (x_i \times y_i)\) \\ \hline
  sumprod2 & Vector \(x\), Vector \(y\), Vector \(z\) & Scalar \(w\) &  \(w = \sum_i (x_i \times y_i \times z_i)\) \\ \hline
\hline
\end{tabular}
\caption{Necessary Operators}
\label{tbl_custom_ops}
\end{table}


\section{Details}


\section{Definitions}
In this section, we make some important definitions which will be useful for the
subsequent discussion. 

\subsection{Chunks}
While a vector may have many elements, for efficiency reasons, we will
often access that data in chunks. The size of a chunk is fixed at the
time the server boots up and stays the same as long as the server is
running. In particular, this means that the same chunk size is used
regardless of which vector is accessed. So, if chunk \(n\) refers to
elements \(m_1\) through \( m_2\) of vector \(x\), then it refers to
the same elements of any other vector \(y\).

In future implementations, we may relax this assumption. Why would we
want to do so? The reason is that we want to adjust the length of the
chunk to be small enough that intermediate results do not have to get
flushed to lower levels of the memory hierarchy yet large enough that
we get the benefits of vectorization.  Given that the memory size is
fixed, expressions which have more partial results would prefer to
have smaller chunk sizes.

\subsection{Materialization}
A vector is said to have been completely materialized if it can return {\bf any}
chunk of data without computation.


\section{Vector Contract}
\label{vector_contract}

\be
\item len() --- returns number of elements. 

Before the vector is completely materialized, the return value may or may not be
known, in which case {\tt nil} is returned.
Once the vector is completely materialized, it must return a non-negative
integer .

Can a vector have zero length? Any special treatment? \TBC
%-------
\item num\_null() --- returns number of elements that are null.

Once again, there are cases where we won't know for sure whether there are null
values or not until the vector is completely materialized. In such cases, we
will return {\tt nil}. 
After the vector is completely materialized, we must a non-negative integer.
%-------
\item fldtype() --- returns kind of element in the vector. 
We support 6 types.
\begin{center}
{\tt int8\_t, int16\_t, int32\_t, int64\_t, float, double}
\end{center}
In the near  future, we will extend these to include 
\begin{center}
{\tt uint8\_t, uint16\_t, uint32\_t, uint64\_t, constant length strings}
\end{center}
%----------
\item materialized() --- returns true or false depending on whether the vector
has been completely materialized.
\item sz() ---  This is sizeof(fldtype) and
hence this is redundant except when the fldtype is ``constant length
string''.
\item memo(bool) --- this tells the vector whether or not it should remember
anything other than the most recent chunk that it generated. It must be called
before any calls to chunk() are made. It returns true or false depending
on whether it was called at the right time. 

\item ismemo() --- returns true or false depending on how memo() was called
earlier. If memo() was never called, it will return true, the default behavior.

\item lastchunk() --- returns nil before any calls to chunk() are made. After
that, returns the highest valid number with which chunk() was called.
\item chunk(int n) --- The expected response depends as follows
\be
\item If the vector has been materialized
\be
\item \(n \geq 0\) 
\be
\item If there are more than (n \(\times\) chunksize) elements, a pointer t
contiguous area of memory containing that data is returned
\item Else, nil
\ee
\item \(n < 0\)

A pointer to a contiguous area of memory containing the data of the
entire vector is returned
\ee
\item If the vector has {\bf not} been materialized
\be
\item If n == lastchunk(), the data for the last chunk generated is returned
without any computation
\item If n == lastchunk()+1, the next chunk is generated and the data is
returned.
\item If n \(>\) lastchunk()+1, error 
\item If n \(<\) lastchunk(), 

\be
\item If memo set to true, then the data for that chunk is available to us and
is returned
\item If memo set to false, error
\ee
\ee
\ee
\ee

{\bf Important}
The novice Q programmer should be unaware of any of the details discussed in
this section.
The advanced Q programmer can squeeze additional performance by judiciously 
\be
\item using memo(false)
\item using fold (Section~\ref{fold}) where several functions are applied inlock
step to a vector
\ee

\section{Implementation Details}
\label{vector_implementation}

These represent suggestions on how to efficiently meet the contract described in
Section~\ref{vector_contract}. While it behooves library writers to follow the
same patterns when creating functions that return vectors, they are not bound to
do so.

A vector consists of the following
\be
\item {\bf M1}, a Lua table which contains mandatory meta-data such as 
\be
\item {\bf cdata}, \verb+char *+, a pointer to a region of {\bf contiguous} memory
containing data 
\item {\bf filename}, \verb+char []+, The name of the file where cdata has been
persisted, if at all. 

At least one of cdata and filename must be non-null.
If filename is defined and cdata is null, then mmap can be used to set cdata. If
cdata is defined and filename is null, then this memory was obtained from
malloc()
and has not yet been persisted to disk.
%----------------------
\item {\bf nncdata}, \verb+char *+, a pointer to a region of {\bf contiguous} memory
containing boolean information about whether corresponding data is null or not.
As an implementation detail, this is stored as bytes, although it could be
stored as bits. In particular, nncdata[i] can be either 0 or 1 where 
1 means that the \(i^{th}\) value is not null and 
0 means that the \(i^{th}\) value is null. 

\item {\bf nnfilename}, \verb+char []+, The name of the file where nncdata has been
persisted, if at all. 

It is possible for nncdata and nnfilename to both be null.
%----------------------
If there are null values, then either nncdata or nnfilename should be non-null.  Else, both nncdata and nnfilename should be null
%-------
\item {\bf lastchunkbuf}, this is a buffer which contains the last chunk
generated. It is allocated before the first call to chunk(). 

When is it freed? Do we care? 
\ee
\item {\bf M2}, a Lua table which contains optional meta-data such as 
\be
\item min, max, sum, \ldots
\item number of null values. If this is 0, then nncdata and nnfilename must be
null. It is possible for this to be nil and one of nncdata and nnfilename to be
defined. This means we {\em may} have null values but haven't calculated that as
yet. 
\item number of distinct values
\item md5hash of data, used as a check that data has not gotten corrupted
\item filesz, size of file referenced in filename. Used for checking.

Note that the size of the file, if one exists, may be set to a
multiple of the file block size and hence may be greater than
required.
\item nnfilesz, size of file referenced in nnfilename. Used for checking.
\item \ldots
\ee

\ee

\subsection{Dealing with memo-ization}

When a vector has been materialized, its data is available either in (cdata,
nncdata) or in (filename, nnfilename). In this case, returning any valid chunk
is simply a matter of pointer arithmetic and an optional mmap step.

Let's consider what happens when chunk() is called on the vector for the firs
time. The buffer, lastchunkbuf, is allocated and 

