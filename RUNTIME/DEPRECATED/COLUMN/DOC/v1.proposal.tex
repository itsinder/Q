\startreport{Lazy Data generation in Q}
\reportauthor{Ramesh Subramonian}
\newcommand{\Assign}{\verb+:=+ }

\section{Introduction}

For now, I am being blase in mixing Lua and C but this is pseudo-code 
so I will take the liberty of doing so.

\TBC


\section{Data storage and access}

Q supports only one data structure, a vector. All elements of the vector must be
of the same type and, for starters, we will support only primitive types ---
integers, floating point numbers and possibly constant length strings.

A vector is best thought of as an object. Lua provides a basic but entirely
servicable object oriented support \url{https://www.lua.org/pil/16.html}.
We now describe what methods such an object has

\be
\item {\bf len}, returns number of elements. There is a special case where this
is allowed to return {\tt nil} indicating unknown but for starters, assume that it
must return a positive integer. 
%-------

\item {\bf isnn}, returns true or false, depending on whether there are null values. 
%-------
\item {\bf fldtype}, The kind of element in the vector. 
We support 6 types.
{\tt int8\_t, int16\_t, int32\_t, int64\_t, float, double}
In the near  future, we will extend these to include constant length 
strings and then unsigned integers. 
%----------

\item {\bf sz}, \verb+ssize_t+ size of each element. This is sizeof(fldtype) and
hence this is redundant except when the fldtype is ``constant length
string''.
\ee

There are two kinds of vectors, depending on whether the underlying data has
been materialized i.e., the entire vector exists in some contiguous block of
memory, optionally persisted to a file on disk.

A description of the data structures necessary to support a materialized vector
is in Section~\ref{data_vector}. A description of how a vector is materialized
is in Section~\ref{vec_from_coro}.

Whether or not a vector has been materialized, it should be possible for the
vector to be consumed in chunks. The construct that supports this is called a
data generator and is implemented as a co-routine. In
Section~\ref{coro_from_vec}, we describe how a materialized vector is served up
in chunks to a potential consumer.

For the most part, we want the Q programmer to be unaware of whether or not 
a vector has been materialized. The Q library writer will need to be keenly
aware of this difference. As of now, this goal is aspirational not real as we
will see in Section~\ref{pitfalls}.


\subsection{Data as a vector}
\label{data_vector}

A vector consists of the following
\be
\item {\bf M1}, a Lua table which contains mandatory meta-data such as 
\be
\item {\bf cdata}, \verb+char *+, a pointer to a region of {\bf contiguous} memory
containing data 
\item {\bf filename}, \verb+char []+, The name of the file where cdata has been
persisted, if at all. 

At least one of cdata and filename must be non-null.
If filename is defined and cdata is null, then mmap can be used to set cdata. If
cdata is defined and filename is null, then this memory was obtained from
malloc()
and has not yet been persisted to disk.
%----------------------
\item {\bf nncdata}, \verb+char *+, a pointer to a region of {\bf contiguous} memory
containing boolean information about whether corresponding data is null or not.
As an implementation detail, this is stored as bytes, although it could be
stored as bits. In particular, nncdata[i] can be either 0 or 1 where 
1 means that the \(i^{th}\) value is not null and 
0 means that the \(i^{th}\) value is null. 

\item {\bf nnfilename}, \verb+char []+, The name of the file where nncdata has been
persisted, if at all. 

It is possible for nncdata and nnfilename to both be null.
%----------------------
\item {\bf len}, \verb+ssize_t+. number of elements
\item {\bf isnn}, \verb+bool+, whether there are null values. 

If isnn = true, then either nncdata or nnfilename should be non-null
If isnn = false, then both nncdata and nnfilename should be null
%-------
\item {\bf fldtype}, how to interpret the memory. We intend to support 6 types.
In the near  future, will extend to constant length strings and then unsigned integers. 
\begin{verbatim}
int8_t, int16_t, int32_t, int64_t, float, double
\end{verbatim}

\item {\bf sz}, \verb+ssize_t+ size of each element. This is sizeof(fldtype) and
hence this is redundant except when the fldtype is ``constant length
string''.
\ee
\item {\bf M2}, a Lua table which contains optional meta-data such as 
\be
\item min, max, sum, \ldots
\item number of null values. If this is 0, then nncdata and nnfilename must be
null. It is possible for this to be nil and one of nncdata and nnfilename to be
defined. This means we {\em may} have null values but haven't calculated that as
yet. 
\item number of distinct values
\item md5hash of data, used as a check that data has not gotten corrupted
\item filesz, size of file referenced in filename. Used for checking.

Note that the size of the file, if one exists, may be set to a
multiple of the file block size and hence may be greater than
required.
\item nnfilesz, size of file referenced in nnfilename. Used for checking.
\item \ldots
\ee


\ee

\subsection{Creating a vector from a co-routine}
\label{vec_from_coro}

Figure~\ref{eval} describes eval(x), the function used to create a
vector from a coroutine.

\begin{figure}
\centering
\fbox{
\begin{minipage}{20cm}
\begin{tabbing} \hspace*{0.25in} \= \hspace*{0.25in} \= \hspace*{0.25in} \= \kill
FILE *fp = NULL, *nnfp = NULL; \\
char filename[] = "/tmp/myfileXXXX", nnfilename[] = "/tmp/myfileXXXX"; \\
int clock = 0; \\
ssize\_t len = 0; \\
prev = nil; \\ 
M1, M2 = coroutine.resume(x, -1);  /* Get meta data */ \\
/* If resume of x returns a scalar, then M2 will be null; else, a table,
even if an empty one */ \\
{\bf if not} M2 {\bf then} isvec = true {\bf else} isvec = false {\bf endif} \\
{\bf if } isvec {\bf then} \+ \\ 
mktemp(filename); fp = fopen(filename, "wb"); \\
{\bf if} M1.isnn {\bf then} mktemp(nnfilename); nnfp = fopen(nnfilename, "wb"); {\bf endif} \- \\
{\bf endif} \\
%---------------------------------------------
{\bf for} ( ; ; ) {\bf do}  /* while chunks to consume */ \+ \\
  rslt = coroutine.resume(x, clock); \\
  {\bf if} {\bf not} rslt  {\bf then} /* prepare for return */ \+ \\
    {\bf if not} prev  {\bf then} \+ \\
      error("first invocation cannot return nil")' \- \\
    {\bf endif} \\
    {\bf if} isvec {\bf then} \+ \\ 
      if ( fp != NULL ) { fclose(fp); }; if ( nnfp != NULL ) { fclose(nnfp); } \\
      local x = \{\} \\
      x.isnn = M1.isnn; x.filename = filename; x.fldtype = M1.fldtype \\
      x.sz = M1.sz; x.len = len; \\
      {\bf if } M1.isnn {\bf then} x.nnfilename = nnfilename {\bf endif} \\
      return x; \- \\
    {\bf else}  \+ \\ 
      return prev; \- \\
    {\bf endif} \- \\
  {\bf endif} \\
  {\bf if} isvec {\bf then} \+ \\
    len += rslt.len; /* Cumulative length of vector */ \\
    fwrite(rslt.cdata, rslt.len, sizeof(rslt.fldtype), fp); \\
{\bf if} M1.isnn {\bf then} fwrite(rslt.nncdata, rslt.len, sizeof(char), nnfp); {\bf endif} \- \\
  {\bf endif} \\ 
  clock++; \\
  prev = rslt; \- \\
{\bf endfor}
\end{tabbing}
\end{minipage}
}
\label{eval}
\caption{Function eval(x)}
\end{figure}

\subsection{Creating a co-routine from a vector}
\label{coro_from_vec}
Figure~\ref{wrap} describes wrap(), the function used to create a
coroutine from a vector.  The details of the function used to create
the co-routine are in Figure~\ref{wrap}.
\begin{verbatim}
q['wrap'] = function (invec)
  return coroutine.create(
    function(clock)
      ...
    end
  )
\end{verbatim}
\begin{figure}
\centering
\fbox{
\begin{minipage}{20cm}
\begin{tabbing} 
\hspace*{0.25in} \= \hspace*{0.25in} \= \hspace*{0.25in} \= 
\hspace*{0.25in} \= \hspace*{0.25in} \= \hspace*{0.25in} \= 
\kill
local o = \{\} ; \\
local lclock = 0; \\
{\bf while} clock \(\neq\) nil and clock \(<\) 0 {\bf then} \+ \\
  coroutine.yield(o, invec.M1, invec.M2) \- \\
{\bf endif} \\
local X = invec.cdata; \\
{\bf if} ( not X ) {\bf then} \+ \\ 
  local struct stat filestat; \\
  int fd = open(invec.filename, O\_RDONLY); \\
  fstat(fd, \&filestat); filesz = filestat.st\_size; \\
  assert(invec.len \(\times\) sizeof(invec.fldtype) \(\geq\) filesz) ; \\
  X = mmap(NULL, filesz, PROT\_READ, fd, 0); \- \\
{\bf endif} \\ 
{\bf for} ( ever ) {\bf do} \+ \\ 
  lclock++; lclock = clock or lclock; \\
  local ssize\_t lb = (clock-1) \(\times\) chunksize \\
  {\bf if } lb \(>\) len {\bf then return} {\bf endif} \\
  local ssize\_t ub = min(len, lb + chunksize) \\
  o.cdata = X + ( lb \(\times\) sizeof(invec.fldtype) ) \\
  o.len   = (ub - lb); \\ 
  o.fldtype = invec.fldtype; \\
  o.sz = invec.sz; \\
  coroutine.yield(o, x.M1, x.M2) \\
{\bf while} clock \(\neq\) nil and clock \(<\) 0 {\bf then} \+ \\
  coroutine.yield(o, invec.M1, invec.M2) \- \\
{\bf endif} \- \\
{\bf endfor} 
\end{tabbing}
\end{minipage}
}
\label{wrap}
\caption{Function wrap(x)}
\end{figure}

\subsection{An example: add} 
\label{example_add}

let us consider the function add which returns a co-routine that will yeild up
chunks of the results of adding two ``things'', whether they be vectors or
co-routines. 
The details of the function used to create the co-routine for add in
Figure~\ref{add}. 
\begin{verbatim}
q['add'] = function (arg1, arg2)
  return coroutine.create(
    function()
    end
  )
\end{verbatim}

\begin{figure}
\centering
\fbox{
\begin{minipage}{20cm}
\begin{tabbing} \hspace*{0.25in} \= \hspace*{0.25in} \= \hspace*{0.25in} \= \kill
{\bf if}  ( typeof(arg1) == "vector" ) {\bf then} arg1 = wrap(arg1) {\bf endif} \\
{\bf if}  ( typeof(arg2) == "vector" ) {\bf then} arg2 = wrap(arg2) {\bf endif} \\
assert( typeof(arg1) == "coroutine" ) \\
assert( typeof(arg2) == "coroutine" ) \\
f3sz, f3type, cfn = somefn(arg1, arg2) \TBC \\
f3 = \{\}; f3.sz = f3sz; f3.fldtype = f3type;  \\
f3.cdata = C.malloc(f3sz \(\times\) chunksize) \\
{\bf do} forever \+ \\ 
c1 = coroutine.resume(arg1) \\
c2 = coroutine.resume(arg1) \\
assert((c1 and c2) or (not c1 and not c2)) \\
{\bf if} ( not c1 ) {\bf then} return {\bf endif} \\ 
C.cfn(c1.cdata, c2.cdata, f3cdata) \\
coroutine.yield(f3) \- \\
{\bf endfor} \\
\end{tabbing}
\end{minipage}
}
\label{add}
\caption{Function used to create co-routine for add(x, y)}
\end{figure}




\section{Data Generators --- Pitfalls and Solutions}
\label{pitfalls}

We have described the representation of data as vectors
and how those vectors can be ``wrapped'' into co-routines which would offer
up chunks of data at a time to consumers (Section~\ref{coro_from_vec}).

A key thing to note is that when we write \(z = \mathrm{add}(x, y)\),
no computation actually occurs. What does happen is that the function
``add'' returns a coroutine which is stored in \(z\). If we were to
say \(w = \mathrm{eval}(z)\), this would invoke the code described in
Figure~\ref{eval} and create a variable called \(w\) with the fields
described in Section~\ref{data_vector}. There are two key points to
note
\be
\item the data for the vector has been ``materialized'' and stored in a contiguous
block. Whether this has been persisted to disk is not relevant.
\item the co-routine in \(z\) is dead. Subsequent calls to {\tt
coroutine.resume(z, \ldots)} will fail.
\ee
In other words, reads are {\bf destructive}. This 
poses a problem {\bf only} when there is more than one
consumer. As an example, if we followed the eval with \(x = sum(z)\),
this would throw an error. So, this breaks a pattern that we are used
to in programming which is that when something is on the left hand
side of an equals sign, we expect it to change. But if it is on the
right of the equals sign, we do not expect it to change. Our
expectation is that we can continue to use it on the right hand side
until it is destroyed by some other action e.g., going out of
scope, \ldots In Section~\ref{sticky_wicket}, we outline some very
real examples where this is a problem.

\subsection{Examples of tricky situations}
\label{sticky_wicket}

One can concoct several examples where destructive reads would land one in
trouble. We provide a few below. 
\be
\item Normalizing a vector. 

A common operation is to normalize a set of values so that they add up
to 1.  e.g, \(y = \frac{x}{sum(x)}\). Note that \(x\) needs to be
consumed twice, first to find the sum, a scalar, and then to divide
each element of \(x\) by that scalar.

\item Computing min, max, sum simultaneously.

Assume we want to find basic statistics on a the data offered by
coroutine \(v\). Each of the operators --- min, max, sum --- is a
consumer of the data.

The function {\tt mss} takes in a coroutine and returns 5 scalars,
(i) minimum (ii) maximum (iii) sum (iv) number of non-null values, and (v) 
total number of values seen. The simplest implementation would be to call, in
sequence, {\tt min, max, sum}, on the same generator, \(x\).

\item an even simpler example is {\tt y = add(x, x)}
\ee

\subsection{A proposed solution}

One solution would be to introduce a new assignment operator, 
\verb+:=+, for the Q programmer to use. 
Then a \verb+:=+ b is the same as a = eval(b), Figure~\ref{eval}. In other words,
we {\bf always} materialize a coroutine as a vector when it is given a name. 
Under what conditions does a coroutine {\bf not} have a name? Take the examle
{\tt x := add(sub(a, b), c)}. The co-routine returned by {\tt sub} is anonymous
and is not materialized. However, the co-routine returned by {\tt add} is
assigned to {\tt x} and hence materialized. 

We believe that this addresses the problem of destructive reads since one can
only read something that has been assigned and the act of assignment causes
materialization after which destruction is not possible. {\em Please feel free
to disagree with this claim.}


The advantage of the \verb+:=+ operator is that it makes life predictable and
understandable for the basic programmer. 
The disadvantage of the \verb+:=+ operator is that it often causes
materialization when that is not necessary. 

So, what does the advanced programmer do? For one, they could use the regular
equals operator but this does not fully solve the problem. It requires us to
introduce a contract on the writer of a data generator.

\subsubsection{The clock contract}

A data generator is a coroutine that offers up consecutive chunks of
data from a vector. Our example, wrap (Section~\ref{coro_from_vec}),
is one such generator.  Locally, the data generator is required to
support the following
\be
\item a local clock, \(C_l\), which is initialized to -1
\item a local buffer which stores the last chunk offered up. Note that it is
possible for a co-routine to {\tt yield} more than one chunk of data e.g., roup
by which returns 2 vectors, one containing the values and the other containing
the number of times that value occurs. 
\ee

The generator is required to support a clock input argument upon resume. 
When the generator is resumed, we have two options, depending on the input
clock, \(C_i\)
\be
\item 
\(C_i = \bot\). When no clock is provided, the next chunk is generated and the \(C_l\) is incremented by one. The chunk is stored in the local buffer and
{\tt yielded} to the resumer.
\item \(C_i \neq \bot\) has four possibilities
\be
\item \(C_i < C_l\) or \(C_i > C_l+1\) --- error
\item \(C_i = C_l\) --- the generator has nothing to do. 
It {\bf yields} the data in the local buffer
\item \(C_i = C_l+1\) --- the next chunk is generated, stored in the local
buffer and {\bf yielded} to the resumer. \(C_l \leftarrow C_i\)
\ee
\ee

\subsection{Advanced evaluation}

We are now in a position to describe how the advanced programmer can get greater
efficiencies. Consider the mms example described earlier. 

\begin{verbatim}
mms(x)
  clock = 0, Initialize partial results for min, max, sum
  p = coroutine.resume(x, clock)
  Compute min(p), max(p), sum(p) and update partial results
  Increment clock
\end{verbatim}


Need more details here, especially for other examples \TBC
